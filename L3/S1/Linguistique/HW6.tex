\documentclass{article}
\usepackage[a4paper, margin = 3cm]{geometry}

\title{Homework Assignment 6}
\date{\today}
\author{Matthieu Boyer}

\markright{Matthieu Boyer}
\pagestyle{myheadings}

\begin{document}
\maketitle
\begin{enumerate}
    \item Predicate 1 : \textsl{Difficult} \\\\
    It seems that \textsl{difficult} has a free variable of the form ${\left[for\ X\right]}$ after it indicating the person it refers to, much like \textsl{delicious}. In the sentence \textsl{skateboarding is difficult}, one can reply it is the easiest thing they've ever done and lead to \textsl{Well, skateboarding is difficult \texttt{for me}}.
    Then, the values it might take are things that can experience difficultness, such as humans, dogs or living creatures. In the following, the italic text symbolises the value of the free variable :
    \begin{itemize}
        \item \textsl{This is a difficult \texttt{for me} exam}
        \item \textsl{Opening nuts is difficult \texttt{for birds}}
    \end{itemize}

    Sometimes, it also seems that \textsl{difficult} can take a second free variable of the form ${\left[to\ X\right]}$, representing the thing that is complemented :
    \begin{itemize}
        \item A : \textsl{I absolutely love linguistics\footnote{This was a sentence pronounced during the writing of this homework.} !}\\
        B : \textsl{Me too, but it is so difficult \texttt{for me,} \texttt{to do linguistics}...}
        \item A : \textsl{I can't believe he was late.}\\
        B : \textsl{He is so difficult \texttt{for me,} \texttt{to live with him}}
    \end{itemize}
    This second variable can take as an input anything that can be found difficult, mainly actions such as skateboarding, cooking, or doing linguistics.

\item Predicate 2 : \textsl{Legal} \\\\
    It seems that \textsl{legal} has a free variable of the form ${\left[in\ X\right]}$ which indicates the location in which everything it is true : a county, a state, a country, a region and so on : 
    \begin{itemize}
        \item \textsl{Cannabis is legal \texttt{in the Netherlands}}
        \item \textsl{Going 300 miles an hour in a car is legal \texttt{on the Highway in Germany}}
    \end{itemize}

    And, as for \textsl{difficult}, we can say that \textsl{legal} takes another free variable of the form $\left[to \  X\right]$ representing what the illegal action is. 
    \begin{itemize}
        \item A : \textsl{Man, I really do love to smoke cannabis ! What about you ?}\\
        B : \textsl{Dude, it's illegal \texttt{to smoke cannabis,} \texttt{in Britain} !}
    \end{itemize}

\item Predicate 3 : \textsl{Of the utmost importance} \\\\
    Following the previous question, it seems that \textsl{Of the utmost importance} takes a free variable of the form $\left[to \ X\right]$ which indicates to whom the considered object if \textsl{of the utmost importance}. The values it may take are things that can have a link to the importance of a thing, such as people, groups of people, or something with a changing value (see examples 3 and 4) :
    \begin{itemize}
        \item \textsl{Linguistics are of the utmost importance \texttt{to me}}
        \item \textsl{Whether Michael slept well or not is of the utmost importance \texttt{to the students he grades}}
        \item \textsl{This exam is of the utmost importance \texttt{to your scolarity}}
        \item \textsl{The pluviometry is of the utmost importance \texttt{to the crops}}
    \end{itemize}

    Again, as in the two previous examples, we can say that \textsl{of the utmost importance} takes another free variable of the form $\left[to \ X\right]$ which indicates the thing it refers to : 
    \begin{itemize}
        \item A : \textsl{Do you think I need to do my linguistics homework ?}\\
        B : \textsl{Yes, it is of the utmost importance \texttt{to do your homework,} \texttt{to your success} !}
    \end{itemize}

    Then, we may add a third free variable right after \textsl{utmost}, of the form $\left[compared\ to \ X\right]$. It indicates, relatively to the context, on what scale is the importance considered : 
    \begin{itemize}
        \item A : \textsl{I am not going to complete my linguistics homework...}\\
        B : \textsl{Well, you have to, it is of the utmost \texttt{compared to your other projects} importance}
        \item A : \textsl{I need to eat today}\\
        B : \textsl{Yes, it is of the utmost \texttt{compared to anything that can be done} importance !}
    \end{itemize}

\end{enumerate}
\end{document}