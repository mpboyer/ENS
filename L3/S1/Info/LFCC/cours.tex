\documentclass{cours}

\newtheorem{example}{Exemple}[section]

\title{Langages Formels, Calculabilité, Complexité}
\author{Mickaël Thomazo \\ \small{Lucas Larroque}}
\date{\today}

\begin{document}
\part{Cours 1 28/09}
\section{Langages, Automates, RegExp, Monoïdes finis}
\begin{definition}
    On appelle \emph{alphabet} un ensemble fini $\Sigma$ de lettres. \\
    On appelle \emph{mot} une suite finie de lettres. \\
    On appelle \emph{langage} un ensemble de mots
\end{definition}
\begin{definition}
    On appelle \emph{automate sur l'alphabet $\Sigma$} un graphe orienté dont les arêtes sont étiquetées par les lettres de l'alphabet $\Sigma$\\
    Formellement, c'est un quadruplet $\mathcal{A} = (Q, \Sigma, I, F, \delta)$ ou : \begin{itemize}
        \item $Q$ est un ensemble fini d'états
        \item $\Sigma$ est un alphabet
        \item $I \subseteq Q$
        \item $F \subseteq Q$
        \item $\delta : Q \times \Sigma \rightarrow 2^{Q}$
    \end{itemize}

    Un calcul de $\mathcal{A}$ sur $w = a_{0}\ldots a_{n}$ est une séquence $q_{0}\ldots q_{n}$ telle que $q_{0} \in I, \ \forall i \geq 1,\ q_{i} \in \delta(q_{i-1}, a_{i})$\\
    On appelle Langage reconnu par $\mathcal{A}$ l'ensemble $\mathcal{L}(\mathcal{A}) =  \left\{w \in \Sigma^{*} \mid \exists q_{0}\ldots q_{n} \text{ calcul de } \mathcal{A} \text{ sur } w \text{ où } q_{n} \in F\right\}$\\
    On dit que $\mathcal{A}$ est déterministe si :\begin{itemize}
        \item $\forall q, a, \left| \delta(q, a)\right| \leq 1$
        \item $\left| I\right| = 1$
    \end{itemize} 
\end{definition}
\begin{definition}
    Une expression régulière est de la forme : 
    \begin{itemize}
        \item $a \in \Sigma$ 
        \item $\emptyset$
        \item $r + r$ (+ désigne l'union : $L_1 + L_2 = \left\{w \in L_{1} \cup L_{2} \right\}$)
        \item $r \cdot r$ ($\cdot$ désigne la concaténation : $L_1 \cdot L_2 = \left\{w_{1}w_{2} \ | \ w_{1} \in L_{1}, \ w_{2} \in L_{2} \right\}$)
        \item $r^{*}$ ($*$ désigne l'étoile de Kleene, $L^{*} = \left\{ \bigodot\limits_{w \in s} w \ \mid \ s \in \bigcup\limits_{n \in \mathbb{N}} L^{n} \right\}$)
    \end{itemize}
\end{definition}

\begin{definition}[Automate des Parties]
    On pose, si $\mathcal{A} = (Q, \Sigma, I, F, \delta)$ est un automate : 
    \begin{itemize}
        \item $\hat{Q} = 2^{Q} = \left\{q_{S} \mid S \subset Q\right\}$
        \item $\hat{I} = \left\{q_{I}\right\}$
        \item $\hat{F} = \left\{q_{S} \mid S \cap F \neq \emptyset \right\}$
        \item $\hat{\delta}(q_{S}, a) = \left\{q_{S^{'}}\right\}$ avec $S^{'} = \bigcup\limits_{q \in S}\delta(q, a)$
    \end{itemize}
    Alors, $\hat{\mathcal{A}} = (\hat{Q}, \Sigma, \hat{I}, \hat{F}, \hat{\delta})$ est un automate déterministe reconnaissant $\mathcal{L}(\mathcal{A})$
\end{definition}
\begin{proof}
    On procède par double inclusion :
    \begin{itemize}
        \item $(\subset)$ On introduit un calcul de $w \in \mathcal{L}(\mathcal{A})$ sur $\hat{\mathcal{A}}$ et on vérifie par récurrence que son dernier état est final.
        \item On procède de même pour la réciproque.
    \end{itemize}
\end{proof}

\begin{definition}
    Un monoïde est un magma associatif unifère. \\
    Un morphisme de monoïde est une application $\phi : (N, \cdot_{N}) \rightarrow (M, \cdot_{M})$ telle que: \begin{itemize}
        \item $\phi(1_{N}) = 1_{M}$
        \item $\phi(n_{1}n_{2}) = \phi(n_{1})\phi(n_{2})$
    \end{itemize}
    Un langage $L$ est reconnu par $(M, \times)$ ssi il existe $P \subset M$ tel que $L = \phi^{-1}(P)$ où $\phi$ est un morphisme de $\Sigma^{*}$ dans $M$
\end{definition}

\begin{proposition}
    $L\subseteq \Sigma^{*}$ est reconnu par un automate ssi $L$ est reconnu par un monoïde fini.
\end{proposition}
\begin{proof}
    \begin{itemize}
        \item Soit $L$ reconnu par un monoïde fini $(M, \times)$. Soit $\phi$ un morphisme tel que $L = \phi^{-1}(P), \ P\subset M$. On pose $\mathcal{A} = (M, \Sigma, \left\{1\right\}, P, \delta)$ où $\delta(q, a) = q \times \phi(a)$. Alors, $\mathcal{A}$ reconnaît $L$.
        \item Soit $\mathcal{A}$, déterministe, complet, reconnaissant $L$. Pour $a \in \Sigma$, $a \rightarrow \phi_{a} : q\in Q \mapsto \delta(q, a)$ induit par induction un morphisme de $(\Sigma^{*}, \cdot)$ dans $(Q^{Q}, \circ)$. Alors, avec $P = \left\{f \in Q^{Q}\ \mid \ f(i) \in F_{\mathcal{A}}\right\}$. On a défini le monoïde des transitions de $\mathcal{A}$.
    \end{itemize}
\end{proof}

\part{Cours 2 - 5/10}

\section{Lemme de Pompage}
\begin{theorem}[Lemme de Pompage/Lemme de l'Etoile]
    Si $L$ est un langage régulier, $\exists n \in \N$ 
    $\forall w \in L, \abs{w} \geq n \Rightarrow \exists x, y, z$ tels que : 
        \begin{itemize}
            \item $w = xyz$
            \item $\abs{xy} \leq n$
            \item $y \neq \epsilon$
            \item $\forall n \geq 0, xy^{n}z \in L$
        \end{itemize}
\end{theorem}
\begin{proof}
    Faire un calcul de $\mathcal{A}$ sur $w$ tel que $\abs{w} \geq n$. Celui-ci passe deux fois par le même état.
\end{proof}

\section{Langages Quotients}
\subsection{Quotients d'un Langage à Gauche}
\begin{definition}[Quotient à Gauche]
    Soit $L, K \subseteq \Sigma^{*}, u \in \Sigma^{*}$. \\
    Le quotient à gauche de $L$ par $u$ noté $u^{-1}L$ est : $\left\{v \in \Sigma^{*} \mid uv \in L\right\}$\\
    Le quotient à gauche de $L$ par $K$, $K^{-1}L$ est $\bigcup_{u \in K}u^{-1}L$
\end{definition}
\begin{proposition}
    \begin{itemize}
        \item $w^{-1}(K+L) = w^{-1}K + w^{-1}L$
        \item $(wa)^{-1}L = a^{-1}(w^{-1}L)$
        \item $w^{-1}(KL) = (w^{-1}K \cdot L) + \sum\limits_{u \in L, v \in \Sigma^{*}\\ w = uv} v^{-1}L$
    \end{itemize}
\end{proposition}

\subsection{Quotient d'un Automate à Gauche}
\begin{definition}
    On définit le quotient à gauche d'un automate par un mot $u$ comme celui obtenu en remplaçant les états initiaux par les résultats d'un calcul de l'automate sur $u$.
\end{definition}
\begin{proposition}
    $L$ est régulier si et seulement si il a un nombre fini de quotients à gauche.
\end{proposition}
\begin{proof}
    \begin{itemize}
        \item Un automate reconnaissant $L$ a au plus un quotient par état.
        \item Posons $A_{L} = \left(\Sigma, \left\{u^{-1}L \mid u \in \Sigma^{*} \right\}, I = L = \epsilon^{-1}L, F = , \delta(w^{-1}L, a) = a^{-1}(w^{-1}L)\right)$\\
        Par récurrence, le calcul de $A_{L}$ sur $w$ termine en $w^{-1}L$
    \end{itemize}
\end{proof}

\subsection{Construction de l'Automate Minimal}
\begin{definition}
    Deux états $q_{1}, q_{2}$ sont distingables si : $\exists w \in \Sigma^{*}, \delta(q_{1}, w) \in F, \delta(q_{2} \notin F)$. 
\end{definition}
\begin{proposition}
    $q_{1}$ et $q_{2}$ sont distingables s'ils n'ont pas même quotient à gauche. Si $\delta(q, a)$ est distingable $\delta(q^{'}, a)$, $q, q^{'}$ sont distingables.\\
    La relation $q, q'$ sont distingables est une relation d'équivalence. 
\end{proposition}

\part{Cours 3 - 12/10}
\section{Langages et Logique}
\subsection{Objectif}
On associe à $w \in \Sigma^{*}$ une structure $D_{w}$ et à $L \subseteq \Sigma^{*}$ une structure $\phi_{L}$ telles que : $w \in L\setminus \left\{\epsilon\right\} \Leftrightarrow D_{w} \vdash \phi_{L}$. On se place dans le cadre de la logique du premier ordre et de la monadique du second ordre.\\

\begin{definition}
    On définit $\textsf{pos}(w) = \left\{0, \ldots, \abs{w} - 1\right\}$. On définit une signature i.e. un ensemble de relations : 
\[
    \begin{aligned}
        \forall a \in \Sigma,& \ L_{a} \text{ d'arité } 1\\
        \leq, & \text{ l'ordre strict sur } \textsf{pos}(w)\\
    \end{aligned}   
\]    
On définit alors la structure $D_{w} = \left(\textsf{pos}(w), \left\{L_{a}^{D_{w}}\right\}_{a \in \Sigma}, <_{w}\right)$
\end{definition}
\begin{remark}
    On aurait pû remplacer $<_{w}$ par $succ_{w}$, mais on perd en expressivité. 
\end{remark}

\subsection{Logique du Premier Ordre et Monadique du Second Ordre}
\begin{definition}[Logique du Premier Ordre]
    On définit par induction la logique du premier ordre.
    \begin{itemize}
        \item Constantes
        \item Variables
        \item Si $R$ est une relation d'arité $n$, $t_{i}$ des termes : $R(t_{1}, \ldots, t_{n})$ 
        \item $\lnot \phi$, $\phi_{1} \land \phi_{2}$, $\phi_{1} \lor \phi_{2}$
        \item $\forall x, \phi, \exists \phi$
    \end{itemize}
\end{definition}
On cherche à associer à $\phi : \exists x,\ (L_{0}(x) \land \forall y (y < x \rightarrow (L_{1}(y))))$, un langage $L_{\phi} = \left\{w \mid D_{w} \vdash \phi \right\}$. 
\begin{definition}{Monadique du Second Ordre}
    Sont des formules : 
    \begin{itemize}
        \item $\forall X,\ \phi$ avec $X$, variable du second ordre qui a une arité associée.
        \item $\exists X, \ \phi$ avec $X$, variable du second ordre qui a une arité associée.
        \item $(x_{1}, \ldots, x_{n}) \in X$ avec $X$ d'arité $n$. On trouve aussi une formule pour les graphes qui mettent en relation deux sommets $s, t$ :
    \end{itemize}
    On se restreint dans la suite à des variables d'arité $1$
\end{definition}

On considère un vocabulaire qui contient une relation $E$ ("arêtes d'un graphe"). On représente un graphe par $D_{G}$ l'ensemble de ses sommets et $E^{D_{G}}$ l'ensemble des arêtes de ce graphe.
\begin{example}
    On trouve alors une formule pour représenter tous les graphes $3$-coloriables : 
    \begin{equation}
        \begin{split}
            \exists X_{1}, \exists X_{2}, \exists X_{3} & \left(\forall x \left(X_{1}(x) \lor X_{2}(x) \lor X_{3}(x)\right) \right) \\ 
            & \land \left(\forall x \forall y \left(E(x, y) \rightarrow \left(\lnot \left(X_{1}(x) \land X_{1}(y)\right)\land \lnot \left( X_{2}(x) \land X_{2}(y)\right) \land \lnot \left( X_{3}(x) \land X_{3}(y)\right)\right)\right)\right)
        \end{split}
    \end{equation}    
\end{example}
    
\begin{example}
    On trouve aussi une formule pour les graphes qui mettent en relation deux sommets $s, t$. Il s'agit de trouver une relation close par successeur qui contient $s$ : 
    \[
        \forall R \left(\left[s \in R \land \forall x, y, \left(R(x) \land E(x, y)\right) \rightarrow R(y)\right] \rightarrow t \in R\right)    
    \]
\end{example}

Ainsi, on peut en déduire une méthode pour reconnaître le langage d'un automate $\mathcal{A} = \left(\left\{0, \ldots, k\right\}, \Sigma, 0, \Delta, F\right)$ avec une formule $\phi_{\mathcal{A}}$ de la monadique du second ordre.

\begin{theorem}
    Un langage $L = L(\mathcal{A})$ est régulier, si et seulement si il existe une formule $\phi = \phi_{\mathcal{A}}$ telle que $\forall w \in L, D_{w} \vdash \phi$.
\end{theorem}
\begin{proof}
    \begin{itemize}
        \item $(\Rightarrow)$ : On peut obtenir le premier élément d'un mot par la formule $\textsf{first}(x) = \forall y ((x = y)\lor x < y)$. On peut faire de même pour savoir si un couple est composé d'une paire successeur-successeuse de l'automate, ou si $x$ est la dernière lettre.\\
        On sépare les positions d'un mot selon l'état de l'automate depuis lequel on part. Il faut alors vérifier que le premier élément est bien dans un état initial, que toute transition est bien valide, qu'on est dans au moins un état avant chaque lettre, et que la dernière position est bien écrite depuis une transition vers un état acceptant. \\
        On obtient alors, en notant $k$ le nombre d'états, et $0$ l'état initial : 
        \begin{equation}
            \begin{split}
                \phi_{\mathcal{A}} : \exists X_{0}, \ldots, \exists X_{k} & \left(\bigwedge_{i \neq j} \forall x, \ \lnot \left(X_{i}(x) \land X_{j}(x)\right)\right)\\
                & \forall x \ \left(\textsf{first}(x) \rightarrow X_{0}(x)\right)\\
                & \forall x, \forall y \ \left(\textsf{succ}(x, y) \rightarrow \bigvee_{(i, a, j) \in \Delta}\left(X_{i}(x) \land L_{a}(x) \land X_{j}(y)\right)\right)\\
                & \forall x \ \left(\textsf{last}(x) \rightarrow \bigvee_{\exists j \in F \mid (i, a, j) \in \Delta} \left(X_{i}(x) \land L_{a}(x)\right)\right)
            \end{split}
        \end{equation}
        \item $(\Leftarrow)$ : On procède par induction. 
        \begin{itemize}
            \item Initialisation : On peut facilement exhiber des automates qui reconnaissent les formules atomiques : $\textsf{Sing}(X), X \subseteq Y, X \subseteq L_{a}$.
            \item Hérédité : On raisonne sur les connecteurs, et on vérifie aisément, par les propriétés de cloture des langages réguliers le résultat. Pour ce qui est de la quantification existentielle, si le langage $L$ défini par $\psi(X_{1}, \ldots, X_{n})$ sur $\Sigma\times\left\{0, 1\right\}^{n}$ est reconnu par $\mathcal{A}$. On exhibe un automate reconnu par $\phi(X_{1}, \ldots, X_{n-1}) = \exists X_{n}\psi(X_{1}, \ldots, X_{n})$, il n'a alors plus qu'a trouver une suite de $0-1$ qui définit la $n$-ième composante additionnelle et fonctionne sur $\Sigma\times\left\{0, 1\right\}^{n}$ comme $\mathcal{A}$. Pour le $\forall$, il suffit de prendre la négation du $\exists$
        \end{itemize}
    \end{itemize}
    
    
\end{proof}


    


\end{document}