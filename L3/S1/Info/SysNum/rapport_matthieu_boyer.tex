\documentclass[12pt]{article}

\title{\textbf{Rapport de Rendu du Simulateur Netlist\\ \small SysNum 2023-2024}}
\author{\textbf{Matthieu Boyer}}
\date{\textbf{8 Novembre 2023}}
\usepackage[a4paper]{geometry}
\usepackage{lmodern}
\usepackage[T1]{fontenc} 
\usepackage[dvipsnames]{xcolor}
\usepackage{booktabs}
\usepackage{array}


\begin{document}
\definecolor{ulmj}{RGB}{255, 255, 20}
\definecolor{ulmv}{RGB}{54, 1, 63}
\pagecolor{ulmv}
\color{ulmj}
\maketitle
\textbf{Les opérations à réaliser par le simulateur ont été implémentée de la manière détaillée dans le tableau ci-dessous :}
\begin{table}
    \caption{\textbf{\color{ulmj}Opérations à Réaliser par le Simulateur}}
    \begin{tabular}{>{\bfseries\color{ulmj}}p{6cm}>{\bfseries\color{ulmj}}p{8cm}}
        \toprule
        Opération & Détails sur l'Implémentation\\
        \midrule
        Stocker les Variables & Une référence contenant une instance du module environnement, modifiée à chaque équation.
        \\
        Lire les Entrées & On lit l'entrée standard avec \textmd{read\_int} et on modifie le contexte de l'étape précédente \textmd{ctx} en accord.
        \\
        Résoudre les Equations de la Netlist & Pour chaque équation, on a une méthode dédiée de résolution, détaillée plus bas. On ajoute au contexte courant \textmd{env} le résultat de l'équation.
        \\
        
        \bottomrule
    \end{tabular}

\end{table}

\end{document}options