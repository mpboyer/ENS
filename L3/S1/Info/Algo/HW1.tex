\documentclass{cours}

\title{Homework Assignment 1}
\author{Matthieu Boyer}
\date{\today}
\usepackage[british]{babel}

\begin{document}

\section{Exercise 1 - [Edit Distance/Levenshtein Distance]}
\subsection{Question 1}
\begin{algorithm}
    \caption{Question 1 - Levenshtein Distance with $f$}
    \begin{algorithmic}
        \Input $\ S, T, f, t$ \Comment Two Strings, the function $f$ computing the values and the step $t$
        \EndInput\\
        \State $\mathbf{D} = \text{zeros}(n + 1, n + 1)$ \Comment $len(S) = len(T) = n$
        \For {$i \gets 0 $to $n + 1$} 
            \State $\mathbf{D}[i][0] \gets i$
        \EndFor\\
        \For {$j \gets 0$ to $n + 1$ } 
            \State $\mathbf{D}[0][j] \gets j$
        \EndFor\\
        \State $\text{up}, \text{left} \gets 0, 0$
        \While {$\text{up} < n$}
            \State $\text{left} \gets 0$
            \While {$\text{left} < n$}
                \State $down \gets \min(n-\text{up}, t)$
                \State $right \gets \min(n-\text{left}, t)$\\
                \State $b \gets \mathbf{D}[\text{up}][\text{left}]$
                \State $a \gets \mathbf{D}[\text{up} + 1 \rightarrow \text{up} + 1 + down][\text{left}]$
                \State $c \gets \mathbf{D}[\text{up}][\text{left} + 1 \rightarrow \text{left} + 1 + right]$\\
                \State $f(a, b, c, d, e)$ \Comment{\parbox[t]{.7\linewidth}{We can suppose here that $f$ modifies only the last line and column of $F$ in $\mathbf{D}$ with side-effect.}}
                \State $\text{left} \gets \text{left} + right$
                \For {$i \gets 1 \text{ to } down - 1$}
                    \State $\mathbf{D}[\text{up} + i][\text{left}] \gets \min{\begin{cases} &\mathbf{D}[\text{up} + i][\text{left} - 1] + 1 \\ &\mathbf{D}[\text{up} + i - 1][\text{left}] + 1\\ &\mathbf{D}[\text{up} + i - 1][\text{left} - 1] + \mathds{1}_{\left\{S[up + i] = T[left]\right\}} \end{cases}}$ \\ \Comment{We update the first Column of the block we consider.}
                \EndFor
            \EndWhile
            \State $\text{up} \gets \text{up} + d$
            \For {$i \gets 1 \text{ to } n$}
                    \State $\mathbf{D}[\text{up}][i] \gets \min{\begin{cases}&\mathbf{D}[\text{up}][i - 1] + 1 \\ &\mathbf{D}[\text{up} - 1][ + i] + 1\\ &\mathbf{D}[\text{up} - 1][i - 1] + \mathds{1}_{\left\{S[up + i] = T[left]\right\}} \end{cases}}$ \\ \Comment{We update the first line of the blocks we will consider.}
                \EndFor
        \EndWhile\\

        \Return $\mathbf{D}[n][n]$
    
    \end{algorithmic}
\end{algorithm}
\begin{proposition}[Complexity and Correction]
    If we denote $C_{f}$ the complexity of $f$, this algorithm has time complexity $\O{\left(C_{f}\left(\frac{n}{t}\right)^{2}\right)}$. This algorithm is correct.
\end{proposition}
\begin{proof}
    \begin{itemize}
        \item Moreover, it is clear this algorithm is correct as it only just applies the dynamic programming algorithm for the Levenshtein distance by steps.
        \item This algorithm complexity comes from the fact it has two while loops for which the commands are executed at most $ n/t $ times. The commands in both \emph{while} loops are executed in $\O\left(C_{f}\right)$. The \emph{for} loops inside the \textit{left} \emph{while} loop are equivalent to loops for $i$ between left and $\text{left} + t - 1$ and thus are disjoint. The sum of their complexity over the \textit{left} \emph{while} loop is then $n$. The number of operations inside the \textit{up} \emph{while} loop is then in $\O\left(C_{f}\frac{n}{t}\right)$ and thus the total complexity is, as announced, in $\O{\left(C_{f}\left(\frac{n}{t}\right)^{2}\right)}$
    \end{itemize}
\end{proof}


\subsection{Question 2}
By the recurrence formula : $\mathbf{D}[i][j] = \max 
\begin{cases}
    &\mathbf{D}[i-1][j] + 1\\
    &\mathbf{D}[i][j-1] + 1\\
    &\mathbf{D}[i-1][j-1] + 1 if S[i] \neq T[j] \text{ else } 0\\
\end{cases}$
we see that $\mathbf{D}[i][j]$ is at most $1$ plus one of its left neighbour, upper neighbour or upper left corner neighbour.


\subsection{Question 3}
By recurrence formula, if we substract from all the values in $A, B, C$ a certain integer $k$, then we get for the new matrix $F$, the one we would have gotten with $A, B, C$ with $k$ substracted to all values. Thus, if the values in $A, A'$, $B, B'$ and $C, C'$ all differ from a common integer, the resulting values after applying $f$ will differ from this same value. Thus, $F' = F +(A^{'} - A)$ and $A^{'} - A$ is a matrix with all values equal.

\subsection{Question 4}
We will show here that we can pre-compute all $t \times t$ matrices in $\O\left(3^{2t}\sigma^{2t}t^{2}\right)$.\\
First, as $\mathbf{D}[i][j]$ here is the minimal number of elementary operations to go from string $S[0:i]$ to string $T[0:j]$. We can thus interpret the submatrix of $\mathbf{D}$ between $(i, j)$ and $(i + t, j + t)$ as the dynamic programming matrix for minimum number of operations to go from string $S[i : i + t+ 1]$ to string $T[j : j + t + 1]$ to which we added a first line and a first column.\\
We thus see that those matrices can be fully determined by their first line, first column and by two words.\\
As the values in $\mathbf{D}$ are bounded by $0$ and $2n$ (we can always remove all letters in $S$ and add all letters in $T$), and by question 2., as values along a line or a column differ by an integer in ${-1, 0, 1}$, the number of first lines and columns is $\O{(n3^{2t})}$. However, from question 3., if we allow negative values for $f$ (which we have no reason not to do), we can always substract from the first line and column the value in the top left corner, and re-add it in $\O(t^2)$ after pre-processing.\\
Moreover, there are $\sigma^{t}$ words of length $t$ over the alphabet so the number of submatrices is $\O({3^{2t}\sigma^{2t}})$. \\
We can then use the recurrence equation to derive the values on the submatrix in $\O{(t^{2})}$. \\
Finally, we can pre-compute all $t\times t$ submatrices in $\O{(3^{2t}\sigma^{2t}t^{2})}$.\\
Then, to access the values of the submatrix from $(up, left)$ to $(up + t, left + t)$, we need to identify the preprocessed corresponding matrix and thus we need to go through $S[up:up+t]$, $T[left:left+t]$, the first column and row of this submatrix to which we substracted the upper left value, which all are done in $\O(t)$.

\subsection{Question 5}
From Question 4, we have got an algorithm that allows us to compute the result in $\O{(3^{2t}\sigma^{2t}t^{2} + \left(\frac{n^{2}}{t}\right))}$. Indeed, after pre computing, we only need to check in $\O(1)$ the $\O\left(\left(\frac{n}{t}\right)^{2}\right)$ $t\times t$ submatrices in $\mathbf{D}$.\\
Thus, if we take $t = \log_{(3\sigma)}(\sqrt{n})$, we have complexity in $\O\left((3\sigma)^{\log_{3\sigma}(n)}\log_{3\sigma}(\sqrt{n}) + \left(\frac{n^{2}}{\log_{3\sigma}(\sqrt{n})}\right)\right)$. As $\log_{3\sigma}(\sqrt{n}) = \Theta(\log(n))$ and $(3\sigma)^{\log_{3\sigma}(n)}\log_{3\sigma}(\sqrt{n}) = \O(n \log(n)^{2}) = o(\frac{n^{2}}{\log(n)})$, this algorithm has complexity in $\O\left(\frac{n^{2}}{\log{n}}\right)$

\newpage
\section{Exercice 2}
In this exercise, we will denote by $\textmd{lg}(n)$ the log in base $2$ of $n$
\subsection{Question 1}
We will try to precompute arrays containing part of the answer by dividing the bitvector $B$ into multiple blocks, of smaller and smaller size. We will use two intermediate arrays of size $\O\left(\frac{n}{\textmd{lg}(n)}\right) = o(n)$ bits and a final array (meaning lowest level array) of size $\O(n)$.\\

We build a first array $A_{0}$ which will contain blocks of size $s_{0} = \textmd{lg}^{2}(n)$ (we will justify this value later). The blocks of this array contain the rank of the first position in $B$ that is compacted in the block, i.e., $A[i] = \textsf{rank}_{1}(i \times s_{0})$ where $\textsf{rank}$ designates the rank operation in $B$. As each entry in $A$ takes at most $\textmd{lg}(n)$ bits to store, $A$ takes $\O\left(\frac{n}{\textmd{lg}^{2}(n)}\textmd{lg}(n)\right) = \O\left(\frac{n}{\textmd{lg}(n)}\right) = o(n)$ bits to store, thus justifying the value chosen for $s_{0}$.\\

Then we build a second array $A_{1}$ which will store the ranks of smaller blocks of size $s_{1} = \textmd{lg}(n)/2$ (as for $s_{0}$, we will justify this value later). Here, the items in the array only need at most $\textmd{lg}\left(\textmd{lg}^{2}(n)\right)$ bits to be stored. This table thus takes $\O\left(\frac{n}{s_{1}}\textmd{lg}\left(\textmd{lg}^{2}(n)\right)\right) = \O\left(4n\frac{\left(\textmd{lg}\textmd{lg}(n)\right)}{\textmd{lg}(n)}\right) = o(n)$ bits to store, the equality coming from the properties of $\textmd{lg}$, thus explaining the general form of $s_{1}$ as $\textmd{lg} * c$ but not yet why $c = 1/2$\\

To finally answer the query, we need to maintain a third array $A_{2}$ which will contain correspondances between every possible $\textsf{rank}_{1}(i)$ query on a bitvector of length $s_{1}$ with $i < s_{1}$ and their answers. There are $2^{s_{1}}$ such bitvectors, and storing all answers on each can be done in $\O(s_{1} \textmd{lg}(s_{1}))$. $A_{2}$ thus needs $\O(2^{s_{1}}\textmd{lg}(s_{1})s_{1}) = \O(\sqrt{n}\textmd{lg}(n)\textmd{lg}{\left(\textmd{lg}(n)\right)}) \leq n$, as $\textmd{lg}(n)/2 = \textmd{lg}(\sqrt{n})$ and from the properties of $\textmd{lg}$, thus explaining the coefficient of $\textmd{lg}$ in $s_{1}$. \\

Using these arrays we can compute $\textsf{rank}_{1}(i)$ in $\O(1)$ time by searching in blocks $i/s_{0}$, $(i \mod s_{0})/s_{1}$ and $(i \mod s_{0}) \mod s_{1}$ in $A_{0}, A_{1} \text{ and } A_{2}$, and as : $\textsf{rank}_{0}(i) = i - \textsf{rank}_{1}(i)$, we obtain the wanted time complexity. The total number of bits used by these arrays is at most $n + o(n)$ so we have the wanted space complexity, and this solution is one. 

\subsection{Question 2}
The solution proposed before can lead to a $\O(\textmd{lg}(n))$ time complexity, so it is not sufficient to answer the $\textsf{select}$ queries. We will thus refine the latter to get a new structure, by using three levels of intermediate arrays using $o(n)$ bits and two final arrays using $\O(n)$ bits.\\

We define our first array $C_{0}$ which records the positions of the $\textmd{lg}^{2}(n)$-th (again, this value will be explained later) $1$ bits. Storing a value in this array costs at most $\textmd{lg}(n)$ bits and there are at most $\frac{n}{\textmd{lg}^{2}(n)}$ values stored in the array. So it only uses : $\frac{n}{\textmd{lg}(n)}$ bits.\\

Then, we will repeat the operation. Let $s_{1}$ be the size of a block in our first array. We want to use at most $\frac{s_{1}}{\textmd{lg}(n)} \leq \textmd{lg}(n)$ bits on this block in our second array $C_{1}$. We can calculate $s_{1}$ on the fly in $\O(1)$ during calculation, and using this, we can deduce the location in $B$ of this block.
However, there are $\textmd{lg}^{2}(n) $ answers in that range, and it so requires $\textmd{lg}(n)^{3} $ bits to store. 
Then, if we have $s_{1} \geq \textmd{lg}(n)^{4}$, we would have sufficient space to store the values in $o(n)$ bits, but, in the other case, we need to redivide the block in the same way.\\

We denote by $s_{2}$ the size of the considered block in $C_{1}$. We would need $\textmd{lg}(s_{2})\textmd{lg}(s_{1})\textmd{lg}(n)$ bits to store all the answers. We obtain a similar inequality on $s_{2}$ as on $s_{1}$ before, by the same methods. If we have $s_{2} \geq \textmd{lg}(s_{2})\textmd{lg}(s_{1})\textmd{lg}(n)$ then we have sufficient space not to need to go further in the construction and keep $o(n)$ space. Else, we have : 
\[
    \begin{aligned}
        s_{2} &<\textmd{lg}(s_{2})\textmd{lg}(s_{1})\textmd{lg}(n)\\
        s_{1} &< (\textmd{lg}(n))^{4}
    \end{aligned}    
\]
So we have : $\textmd{lg}(s_{1}) < 4(\textmd{lg}(\textmd{lg}(n)))$. 
Then, by growing of $\textmd{lg}$ and as $s_{2} < s_{1}$ we get : $\textmd{lg}(s_{1}) < 4 \textmd{lg}(\textmd{lg}(n))$ and thus : \[s_{2} < 16 \textmd{lg}(n) \textmd{lg}^{2}(\textmd{lg}(n))\].

Using the same idea as in question 1., we answer $\textsf{select}$ queries using final arrays, storing the values we need. Indeed, for each of the $2^{s_{3}}$ bit pattern of length $s_{3} = \frac{\textmd{lg}(n)}{2}$, we can record the position of the $i$-th 1 bit in the pattern in an array $C_{3}$, and store the number of $1$ in the pattern in an array $C_{4}$. Then to compute the value of $\textsf{select}_{1}(i)$ we can scan the range using $C_{4}$ to know which subrange contains the answer and use $C_{3}$ to get the answer, all in time $\O(1)$.\\

Finally, $\textsf{select}_{1}(i)$ can be computed by finding the block in $C_{0}$ in which $i$ is. From here, we compute $s_{1}$ and then, based on the case disjunction detailed earlier, we either get the correct answer from $C_{1}$ or compute $s_{2}$ and start over, using the final arrays. Thus, this data structure supports $\textsf{select}$ queries in $\O(1)$ time. 
For storing the auxiliary directories $C_{0}, C_{1}$ and $C_{2}$, we need at most $\frac{3n}{\textmd{lg}(n)}$ bits (the simplicity of this expression justifies the values chosen for $s_{1}$ and $s_{2}$), and for the final arrays we use $\sqrt{n}\left(\textmd{lg}(n) + \frac{1}{2}\textmd{lg}(n)\textmd{lg}(n)\right)$, by the same calculation as in question 1, thus justifying the choice for the size of the patterns in $C_{4}$. The extra storage is in $\O(n)$, and this data structure is compatible with our requirements.

\subsection{Question 3}
Here, we model our set $S$ of items as a bitvector $B$ of length $n = \max S$ with coefficients : \[B[i] = 1 \text{ if } i \in S \text{ else } 0\]
Then, solving, the predecessor problem can be done using the data structures introduced in questions $1$ and $2$. Indeed, finding the predecessor of $j$ in $S$ can be done by calling $\textsf{rank}_{1}(j)$, getting the number of elements smaller than $j$ in $S$, then calling $\textsf{select}_{1}$ on the resulting value, giving us the greatest element in $S$ that is smaller than $j$. 

\subsection{Question 4}


\end{document}