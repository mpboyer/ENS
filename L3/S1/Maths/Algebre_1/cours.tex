\documentclass{cours}
\title{Algèbre 1}
\author{Gaëtan Chenevier}
\date{\today}


\begin{document}
\section{Ensembles Quotients}
\section{Généralités sur les Groupes}
\subsection{Isomorphismes et Morphismes}
\begin{theorem}[Transfert de Structure]
    Soient $G$ un groupe, $X$ un ensemble en bijection par $\phi$ avec G. Il existe une unique loi de groupe sur $X$ telle que $X$ et $G$ soient isomorphes par $\phi$.
\end{theorem}
\begin{corollary}
    Tout ensemble peut être muni d'une loi de groupe : 
    \begin{itemize}
        \item Si $X$ est fini, on prend une bijection de $X$ dans $\mathbb{Z}/n\mathbb{Z}$.
        \item Sinon, il est en bijection avec $\left(\mathbb{Z}/2\mathbb{Z}\right)^{(X)}$ (Note, parties finies)
    \end{itemize}
\end{corollary}
\begin{remark}
    Si on connaît toutes les lois de groupe sur $\left\{1, \ldots, n\right\}$, on connaît tous les groupes d'ordre $n$.
\end{remark}
\begin{definition}
    On note $\text{Aut}_{G}$ l'ensemble des automorphismes de $G$.
\end{definition}
\begin{remark}
    L'ensemble des automorphismes intérieurs de $G$ est un sous-groupe de $\text{Aut}_{G}$
\end{remark}
\begin{remark}
    Tous deux isomorphismes $\phi$ et $\psi$ de $G$ dans $G^{'}$ s'obtiennent par la composé de l'un par un automorphismes de $G$.
\end{remark}
\end{document}