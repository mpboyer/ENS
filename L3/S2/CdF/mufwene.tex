\documentclass{cours}
\usepackage{qtree}

\title{Mouvements de Populations, Contacts des Langues, Évolution Linguistique\\ \small Chaire Anuelle Mondes Francophones}
\author{Salikoko S. Mufwene}

\begin{document}
\section*{Introduction}
Dans la bible, la diversité des langues est une punition divine. Faut-il rêver d'une langue unique ? \og Les langues n'ont pas de vie indépendante de leur locuteur. Comme les virus, nous les transmettons d'une personne à l'autre. \fg
Il y a un fort lien entre l'évolution des langues et le colonialisme.
La question de l'avenir du français fait débat. Faut-il lutter pour une langue française pure ? Les emprunts à d'autres langues sont pourtant des signes de vitalité.\\
Salikoko Mufwene est né en RdC puis a étudier à l'université du Zaïre et à celle de Chicago. Les variables historiques et sociales sont fortement liées à l'évolution des langues.
Mufwene met en parallèle l'écologie et la mise en danger des langues, au sein d'un concept de vitalité des langues.
La distinction créole/non-créole des langues n'est pas selon lui liée à des questions structurelles mais plus à des questions socio-historiques. \\
\textit{The Great Human Diaspora}: Nous sommes une population de migrants. Dans l'histoire de l'\textit{Homo Sapiens}, on trouve plusieurs couches de co-colonisation, d'abord au sens génétique, puis aussi au sens politique, avec la relocalisation d'une population sur un territoire où une population est déjà présente. 
Il faut négocier les termes de la cohabitation.\\
Dans la suite, on entend \textit{Mouvements de Populations} comme migrations. 


\section{L'Avant Dernière Vague de l'Expansion Indo-Européenne}
Avec l'invention de la caravelle, le Portugal, l'Espagne, la France, l'Angleterre et la Hollande ont rejoints l'aventure coloniale. 
Durant cette vague, il y a en particulier eu la création de la Nouvelle-France et de l'Afrique dite francophone\footnote{La majorité de la population ne parlant pas français}.\\
L'une des particularités de cette vague est l'émergence de langues appelées créoles et pidgins. 
La dernière vague d'expansion Indo-Européenne a eu lieu au 20ème siècle quand les portugais dits brésiliens ont envahi l'Amazonie.
La carte de l'émergence des créoles et des pidgins montre que les créoles et les pidgins forment des parles complémentaires. 
À Hawaii, on trouve que les pidgins émergent dans les plantations et les créoles en ville, à l'inverse d'ailleurs dans le monde. 
Ailleurs, il y a eu un mélange des populations serviles et engagées, ce qui a causé cette différence. Les engagés n'apprennent pas la population du pays colonisateur, mais des gens avec qui ils travaillent.
Mufwene a appris l'anglais au Congo avec des professeurs Belges par exemple.\\
Au 19ème siècle, les colons plus cultivés qui vont aux colonies ont découvert les créoles. Leur erreur fût de comparer le créole à la variété standard de leur pays, plutôt qu'à la version de leur langue parlée sur place.
Leurs explications s'inscrivent dans le \og \textit{White Man's Burden}\fg des anglais. Pour certains, comme Julien Vinson, les esclaves sont incapables de reproduire les sons raffinés.\\
Toutefois, certains comme Hugo Schuchardt et Adolpho Coelho pensent que les créoles nous invitent à revisiter l'explication de l'époque de l'évolution des langues romaines. 
Selon eux, les processus qui ont aboutits à la création des langues créoles sont les mêmes que ceux qui ont amené à la formation des langues romaines.\\
Ils ont été suivis au 20ème siècle par d'autres linguistes. Suzanne Sylvain concluait était que les créoles sont des langues hybrides nées par le processus de création/de sélection de langues en contact.
Les créoles ont beaucoup maintenu la prononciation du français. Il est rare que les gens apprennent la phonologie s'en prendre attention aux particularités gramaticales.\\
Selon Brigitte Schlieben-Lange, les créoles sont des langues romanes. Toutefois, comme Schuchardt, elle a déduit que les éléments qui ont mené aux créoles ont menés aux langues romanes. Mervyn Alleyne a eu le même avis.\\
Mufwene propose lui de s'inspirer de l'évolution biologique pour étudier l'évolution des langues. Si on prend le point de vue de la propagation des virus, on veut observer qui interagit avec qui, et qui modifie quoi. 
On trouve une restructuration continuelle. Selon Robert Chaudenson, les processus de création des créoles sont similaires aux processus de variantes populaires, de par de nombreuse ressemblance grammaticale et distinction avec la grammaire standard.\\
Selon Rebecca Posner et R.L.Trask, on devrait envisager les créoles comme de nouveaux parlers indo-européens. Pourquoi avait on même désavoué les créoles comme appartenant à la même famille phylogénétique?
Pour des questions de races, qui ont entouré fortement l'évolution des créoles.\\

\section{Modèle des Colonies de Traite et Contact des Langues}
Il y avait des institutions politiques bien fondées en Afrique, pour l'installation des comptoirs (qui ont menés au début des villes), il a fallu négocier. 
Les vieilles villes européennes étaient entourées de murs, et on a donc fait de même en Afrique. \\
Alors, le commerce n'est pas accessible à tous, et il y a eu la fondation des compagnies des indes. 
Pour plusieurs raisons, notamment médicales (malaria) et linguistiques (connaissance des langues africaines), les européens ne pouvaient pas entrer dans les terres africaines.
Des femmes faisant du trait sur l'Afrique l'ouest ont fait des alliances avec les européens. Les enfants métis bilingues ont alors joué un rôle crucial dans le commerce.\\
Les colonies de trait ont survécues comme quartiers portugais dans certaines villes d'Asie. Les populations dites \og eurasiennes \fg, ont par la suite réclamé la nationalité.
Il y a eu la création d'une nouvelle ethnique de Portugais d'Asie qui parlent un portugais dit \og créole\fg par les linguistes.\\
Certaines companies des indes ont encouragé leurs agents à apprendre les langues locales et ont poussé pour la fondation des écoles bilingues.\\
La raison pour laquelle les pidgins ont émergés tardivement dans les colonies de traite vient de la valeur des marchandises. En effet, il est impossible de négocier dans une langue \og cassée\fg ni dans deux langues différentes.\\
Il y a eu la formation de nombreux interprètes/intermédiaires qui parlaient tous la même langue, le créole souvent. Durant des temps où le capitalisme se mettait déjà en oeuvre, lorsque les plantations ont grandi, alors que les européens devenaient minoritaires, les africains ont commencé à parler entre eux dans une langue européenne qui s'est modifiée.
C'est le même système que ce qui s'est passée à la chute de l'empire romain. A l'apogée de l'empire, il y avait 1 millions d'habitants. Si les fonctionnaires avaient dûs être romains, Rome serait restée vide. 
Les romains ont établis des partenariats avec les chefs locaux/indigènes, les ont romanisés et leurs ont appris le latin, permettant à leurs enfants de rentrer dans l'institution romaine. 
Le latin ecclésisatique a aussi été diffusé fortement par les missionnaires et les érudits ont développé le latin scolaire et chaque province a modifié son propre latin. Cicéron s'en plaignait déjà au 1er siècle!\\
Le latin était fortement diffusé dans l'empire, mais les variétés régionales n'était pas continues spatialement. Par exemple, sous le règne arabe, avant la reconquista, le latin existait encore.\\
Après la chute de l'empire, les populations germaniques sont venus coloniser à leur tour la Gaule. Au lieu d'imposer leur langue (le francique), ils ont adoptés le latin. De la même manière que les Normands ont d'abord fonctionné en français en Angleterre avant de l'abandonné. 
L'administration a appris le latin vulgaire émergent. On ne peut pas nier l'impact du contact des langues dans leur évolution. On parle d'influence abstratique.\\
Par exemple, le latin n'avait pas d'articles mais l'allemand en a, ce qui peut expliquer l'existence d'articles en français. 
On peut également comparer l'isolement géographique de la Gaule par rapport à Rome comme la ségrégation spatiale dans les territoires où les créoles. 
On découvre ici le phénomène d'\og Approximation de l'Approximation de l'Approximation \fg.

\section{Evolution Naturelle}
Compte tenu des français populaires, on ne peut pas dire que le français standard a évolué d'une façon entièrement naturelle. 
La réintroduction du latin, la régulation institutionelle du français (l'Académie Française), le rôle de la scolarisation de masse, la concentration des grandes maisons d'éditions, etc\ldots sont des exemples de tentatives par des personnes pensant avoir autorité d'influer le français.\\
Les langues romanes ont en partie eu une évolution naturelle, même si certain·e·s locuteurices s'érigent en gardien·ne·s de l'uniformité et de la pureté de leur langue comme patrimoine national.\\
Les variétés standards des langues romanes reflètent des interventions. Les créoles n'étant pas (encore) standardisés, ils reflètent aussi l'évolution naturelle des parlers non standards. 
Dans les variétés standardisées présentent une combinaison de sélections naturelles et artificielles, la conséquence d'interventions des institutions régulatrices, dont l'école. 
Les pressions et processus de sélection qui ont favorisé les variétés dites normées par rapport aux variétés populaires sont les mêmes que celles qui ont avantagées le français moderne par rapport à l'occitan et au provençal.\\
Encore une fois, \og Un langage est un dialecte avec une armée et une marine\fg. 


\end{document}
