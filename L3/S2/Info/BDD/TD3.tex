\documentclass{cours}

\begin{document}
\section{Minimisation Algorithm}
\begin{enumerate}
    \item $q(x_{1}) \gets R(x_{1}, 42, 12)$.
    \item $q(x_{1}) \gets \exists z_{2} \left(\exists y, z_{1}, x_{2} R(x_{1}, y, z_{1}) \land R(x_{2}, y, z_{2})\right) \land \exists y_{2} \left(\exists z, x_{3}, y_{3}, R(x_{1}, y_{3}, z) \land R(x_{3}, y_{2}, z)\right)$. On minimise en : 
    \[
        q_{x_{1}, z_{2}} \gets \exists y_{1}, z_{1}, x_{2}, R(x_{1}, y_{1}, z_{1}) \land R(x_{2}, y_{1}, z_{2})
    \]
    \item $q_{z_{1,2}, y_{3}} \gets 
    \exists x_{1,1} \left(\exists y_{1,1}, z_{1,1}, x_{1,2}, R(x_{1,1}, y_{1,1}, z_{1,1}) \land R(x_{1, 2}, y_{1,1}, z_{1,2})\right) 
    \land x_{2} \left(\exists z_{2}, R(x_{2}, 8, z_{2})\right) 
    \land y_{3}, z_{3}, R(5, y_{3}, z_{3})$
    On minimise en : 
    \[
        q_{z} \gets \exists x_{1}, y_{2}, R(x_{1}, 8, z) \land R(5, y_{2}, z)
    \]
    \item $\exists x_{1}, y_{1}, R(x_{1}, y_{1}, 3) \land \left(R(x_{1}, y_{1}, 1)\right)$
\end{enumerate}

% \section{PostgreSQL and Query Minimisation}
% On lit le code. Flemme.

\section{Minimisation of Unions of Conjunctive Queries}
On montre d'abord que $q \sqsubseteq q' \Longleftrightarrow \forall i, \exists j, q_{i} \sqsubseteq q_{j}'$
\begin{itemize}
    \item[$\Leftarrow$] On se donne une instance $I$ et $t \in q(I)$. Alors, $\exists i, t\in q_{i}(I)$ et puisque $q_{i} \sqsubseteq q_{j(i)}'$ alors $t \in q_{j(i)}'(I)$ et donc $t \in q'(I)$.
    \item[$\Rightarrow$] On considère l'instance canonique $I_{q}$. On a : $q_{i}\left(I_{q_{i}}\right) \subseteq q\left(I_{q_{i}}\right) \subseteq q'\left(I_{q_{i}}\right)$. Donc pour un certain $q_{j(i)}'$, les variables libres $\left(a_{x_{i, 1}},\ldots, a_{x_{i, m}}\right)$ sont dans $q'_{j(i)}(I_{q_{i}})$ donc $q_{i} \sqsubseteq q_{j}'$ par théorème d'homéomorphisme.
\end{itemize}
Pour la minimisation, on procède comme suit : 
\begin{itemize}
    \item On minimise chaque clause
    \item Pour tout $i$, s'il existe un $j$ tel que $q_{i} \sqsubseteq q_{j}$, on supprime $q_{i}$. On conserve bien les équivalences par le premier point. 
\end{itemize}
Par le premier point et le théorème d'homéomorphisme, on a bien une UCQ minimale (à isomorphisme près) puisque si $q' \sqsubseteq q$, on a pour toute clause $q_{i}'$ l'inclusion $q_{i}'\sqsubseteq q_{j(i)}$, mais par minimalité de chacune des clauses, il y a un isomorphisme. En particulier, on a donc strictement moins de clauses dans $q'$ que dans $q$, et donc il y a des clauses redondantes dans $q$. 
\end{document}