\documentclass{cours}

\begin{document}
\section{Exercice 1}
L'évaluation de circuits booléens est $P$-complète. On considère la réduction à la logique de Horn.  
On considère $H \gets$ et $H\gets B_{1} \land B_{2}$. 
On considère la DB qui pour chaque $H\gets$ a un tuple $Fact(H)$ et pour chaque $H \gets B_{1} \land B_{2}$ on a un tuple $Rule(H, B_{1}, B_{2})$. 
On prend comme programme Datalog le suivant $True(x) \gets Fact(x)$ et $True(x) \gets Rule(x, y, z) \land True(y) \land True(z)$. On vérifie trivialement qu'on a $True(A)$ si et seulement si $A$ est une conséquence du programme originel. 
On encode bien en log space. 

\section{Exercice 2}
Datalog ne peut pas vérifier le nombre pair d'éléments d'un ensemble (FO ne peut pas mdr)

\section{Exercice 3}
Même argument que pour Datalog normal. 

\section{Exercice 4}
$initial_{k}(y) \gets \lnot \exists x, succ_{k}(x, y)$;

$final_{k}(y) \gets \lnot \exists x, (succ_{k}(y, x) \land \lnot succ_{k}(x, y))$;

$succ_{k}(x, y) \gets \lnot \exists z, (succ_{k}(x, z) \land succ_{k}(z, y) \land \lnot succ_{k}(y, z))$;

\section{Exercice 5}
cf TD5 exercice 2

\section{Exercice 6}


\end{document}
