\documentclass{cours}

\begin{document}
\section{Exercice 1}
\subsection{Question 1}
\begin{itemize}
    \item Relation $\emptyset \vdash \bot$, $S\vdash (a, b) \Leftrightarrow \forall x \in S, a\leq x \leq b$
    \item Concrétisation : $\gamma(\bot) = \emptyset$, $\gamma((a, b)) = \left\{a, \ldots, b\right\}$ et on a bien, si $c \vdash a$, $c\subseteq \gamma(a)$.
    \item Abstraction : $\alpha(\emptyset) = \bot$, $\alpha(S) = (\inf S, \sup S)$ et on a bien, si $c \vdash a$, $\alpha(c) \preceq a$.
    \item On a: $\gamma(C) \sqsubseteq \gamma(C)$ donc $\alpha(\gamma(C)) \preceq C$. De même, $\gamma(\alpha(S)) \sqsubseteq S$.
\end{itemize}

\subsection{Question 2}
On considère les opérations au sens de Minkowski. \\
On définit déjà $(a, b) \cup (c, d) = \min (a, c), \max(b, d)$
\begin{itemize}
    \item Pour l'addition on prend la convention $(-\infty, a) + (b, +\infty) = (-\infty, +\infty)$ où, $a, b$ sont quelconques et $\bot + S = \bot$. Il est clair qu'on a bien la sûreté par une sur-estimation puisqu'on sur-estime les ensembles par des intervalles au départ et qu'il y a une correspondance exacte pour les intervalles.
    \item Pour la multiplication $(-\infty, a < 0) \times (b > 0, c) = (-\infty, b \times a)$, $(-\infty, a) \times (b, c < 0) = (a\times b, +\infty)$, $(a < 0, b > 0) \times (c < 0, d > 0) = (\min(bc, ad), \max(ac, bd))$.
    \item Pour la division euclidienne, si $(a, b) \div (c < 0, d > 0) = a \div (c, -1) \cup a \div (1, d) \cup (+\infty, +\infty) \cup (-\infty, -\infty)$ avec les valeurs positives si $b > 0$, négatives si $a < 0$. Sinon, on a $(a, b) \div (c > 0, d > 0) = (a\div d, b \div c)$, et $(a, b) \div (c < 0, d < 0) = (a\div d, b \div c)$ 
    \item Pour le modulo, $(a, b) \mod (c < 0, d > 0) = (a, b) \mod (c, - 1) \cup (a, b) \mod (1, d) \cup (0, 0)$. Sinon, on considère $\left\{x \mod y\ \middle|\ a \leq x \leq b, c \leq y \leq d\right\}$.
\end{itemize}

\subsection{Question 3}
On définit déjà $(a, b) \cup (c, d) = \min (a, c), \max(b, d)$. Il n'est pas exact car $(1, 2) \cup (4, 5) = \{1, 2, 4, 5\} \neq (1, 5)$.\\
De plus \[(a, b) \cap (c, d) = \begin{cases}
    \bot & \text{si } b < c \lor a > d\\
    (\max(a, c), \min(b, d)) & \text{sinon}
\end{cases}\]
Cet opérateur est exact.

\section{Exercice 3}
\subsection{Question 1}
Oui car $\gamma((a, b)) = \left\{a, \ldots, b\right\}$ et $\alpha(\left\{a, \ldots, b\right\}) = (a, b)$. De plus $\gamma((-\infty, a)) = \left\{\ldots, a\right\}$ et $\alpha(\left\{\ldots, a\right\}) = (-\infty, a)$.

\subsection{Question 2}
Oui, en effet, $\gamma$ est surjective, il suffit de prendre pour 


\end{document}