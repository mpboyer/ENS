\documentclass[a4paper]{article}

\usepackage[left=1cm, right=1cm, top=2cm, bottom=2cm]{geometry}
\usepackage{booktabs}
\usepackage[french]{babel}
\usepackage{longtable}
\usepackage{hyperref}

\title{Descriptif des Enseignements Suivis}
\author{Matthieu Boyer}
\date{}

\def\cours#1#2#3#4#5{
  \midrule
  #1 & #2 & #3 & #4\\
\cmidrule(r){2-2}
     & #5 & & \\
}

\begin{document}
\maketitle

\begin{center}
	\begin{longtable}{p{.1\textwidth}|p{.5\textwidth}|p{.25\textwidth}|l}
		\toprule
		Volume Horaire & Intitulé et Descriptif & Enseignants & Cursus \\
		\midrule
		\multicolumn{4}{l}{\bf Enseignements en Informatique}          \\
		\cours{48h}{Algorithmique}{Tatiana STARIKOVSKAYA, Pierre ABOULKER}{L3/DENS}{Le but de ce cours est d’acquérir une maîtrise des concepts de base et de certains concepts avancés de l’algorithmique et des structures de données, ainsi que de se familiariser à l’implémentation de ces concepts dans un langage de programmation. Le cours consiste en des leçons hebdomadaires, suivies de sessions de travaux dirigés. Des devoirs à la maison, avec une partie théorique et une partie implémentation, sont à réaliser chaque semaine. Le cours se conclut par un examen.
			Les concepts abordés sont les suivants : introduction à l’algorithmique et aux structures de données diviser pour régner ; programmation dynamique et algorithmes gloutons ; algorithmes de tri ; ensembles et tableaux associatifs ; algorithmes de texte ; ensembles disjoints ; arbre couvrant minimal ; recherche en profondeur ; plus courts chemins ; réseaux de flots.}
		\cours{48h}{Bases de Données}{Pierre SENELLART, Michaël THOMAZO, Paul BONIOL, Lucas LARROQUE}{L3/DENS}{Ce cours couvre les grands principes des systèmes de gestion de données (SGBD). Les SGBD sont des logiciels génériques permettant le stockage et la manipulation efficace de données pour une très large gamme d’applications. Du point de vue pratique, les SGBD sont des logiciels sophistiqués, très largement utilisés, omniprésents dans le monde industriel. Du point de vue théorique, la conception de ces systèmes repose sur des fondements conceptuels, logiques, algorithmiques, en lien avec d’autres domaines de la science informatique. Le cours ira des aspects théoriques aux aspects systèmes des SGBD, en particulier ceux basés sur le modèle relationnel.}
		\cours{34h}{Informatique Pratique}{Pierre SENELLART, Timothy BOURKE, Jean-Christophe FILLIÂTRE, Nicolas GEORGE}{L3/DENS}{Cette semaine de pré-rentrée du DI est une semaine d'initiation intensive aux aspects pratiques de l'informatique : utilisation d'un systèmes Linux, réseaux, travail collaboratif et à distance, utilisation avancée d'éditeurs de texte, programmation en C et OCaml, chaîne de compilation, LaTeX\ldots}
		\cours{48h}{Langages de Programmation et compilation}{Jean-Christophe FILLIÂTRE, Jérôme BOILLOT}{L3/DENS}{Ce cours présent les principaux concepts des langages de programmation au travers de l’étude de leur compilation, c’est-à-dire de leur traduction vers le langage machine. Les TDs ont pour objectif de programmer certaines des notions vues en cours. L’évaluation comprend un projet consistant en la réalisation d’un petit compilateur.}
		\cours{48h}{Langages formels, calculabilité et complexité}{Michaël THOMAZO, Lucas LARROQUE}{L3/DENS}{1. Langages réguliers, leurs propriétés et leur caractérisation par automates, expressions régulières, formules logiques, monoïdes. Langages sans étoile. Premières notions sur les langages de mots infinis.
			2. Grammaires et hiérarchie de Chomski. Langages hors contexte, leurs propriétés, leur caractérisation par automates à pile
			3. Calculabilité (fonctions récursives et Machines de Turing). Problèmes décidables, indécidables, semi-décidables.
			4. Complexité en temps et espaces. Bornes de complexité. Classes de complexité (NP, Pspace) et problèmes complets.}
		\cours{48h}{Systèmes d'exploitation}{Timothy BOURKE, Marc POUZET}{L3/DENS}{Le cours présente les concepts fondamentaux des systèmes d’exploitation, leur utilisation et leur mise en œuvre dans le système Unix. La première partie étudie le cas d’Unix : organisation de la mémoire, systèmes de fichiers, gestion des processus lourds et léger (“threads”), signaux, communication entre processus, interruption, ordonnancement préemptif, pipes, sockets. La seconde partie étudie les problèmes classiques : inter blocage et famine entre processus, courses critiques, prise en compte des temps de calcul, etc. Le cours aborde la modélisation de ces questions et comment les techniques de vérification formelle automatiques permettent de définir des implémentations prouvées correctes. Un projet de programmation est présenté en début du cours. Il est réalisé en groupe (typiquement en binôme) et donne lieu à une soutenance. Une feuille de TD est distribuée à chaque cours.}
		\cours{48h}{Systèmes numériques}{Sylvain GUILLEY, Hadrien BARRAL, Théophile WALLEZ}{L3/DENS}{Le cours théorique présente la composante matérielle du monde informatique. Des principes de conception et de réalisation des circuits, à diverses applications du calcul numérique haute performance : en physique, électronique, algèbre et télécommunication. Chaque application va de l’algorithme (logiciel) au circuit (matériel) : mêmes opérations, autres performances. La partie pratique du cours est un projet, à réaliser par groupes : chaque groupe doit entièrement concevoir un microprocesseur, et le réaliser au moyen de portes logiques élémentaires ; il faut ensuite simuler les portes en fonctionnement, et programmer le microprocesseur pour en faire une montre numérique, simulée en temps-réel.}
		\cours{48h}{Sémantique et applications à la vérification de programmes}{Xavier RIVAL, Jérôme FÉRET, Sylvain CONCHON}{L3/DENS}{Dans ce cours, nous étudierons les techniques permettant de raisonner sur les programmes, afin de vérifier des propriétés de correction. Nous nous intéresserons tout d’abord aux fondements de la sémantique des langages de programmations, et à la notion de preuve de programmes à l’aide de triplets "à la Hoare".  Ensuite, nous formaliserons les différents types de propriétés intéressantes (sûreté, vivacité, sécurité). Enfin, nous aborderons plusieurs approches permettant de vérifier des programmes de manière automatique (analyse statique par interprétation abstraite, vérification de modèles de systèmes finis, résolution modulo théorie) : l’inférence des étapes de la preuve est alors confiée à un autre programme informatique.}
		\cours{48h}{Théorie de l'information et codage}{Bartek BLASZCZYSZYN}{L3/DENS}{Ce cours présente la théorie de l’information et du codage dans un cadre discret. On s’intéresse à la quantité d’information contenue dans un message et aux moyens de transmettre ce message à travers un canal bruité. On s’intéresse donc à la fois aux méthodes de compression des données et aux méthodes de détection et de correction d’erreurs. I - Compression des données : taux de compression et entropie ; algorithme de Huffman, ZivLempel et optimalité II - Canal de transmission : capacité d’un canal, théorème de Shannon III - Codes correcteurs d’erreurs : codes linéaires, codes cycliques, code de Hamming, codes BCH.}
		\midrule
		\cours{48h}{Lambda-calcul et catégories}{Paul-André MELLIÈS, Quentin ARISTOTE}{M1/DENS}{Ce cours s’intéresse à la syntaxe et à la sémantique des langages de programmation, à partir du lambda-calcul. On rappellera les principaux théorèmes syntaxiques du lambda-calcul : confluence, standardisation, résultats de terminaison. Puis on étudiera les modèles du lambdacalcul : pour ce faire, le langage de la théorie des catégories sera utilisé.  Plus généralement, les catégories servent à interpréter bien des extensions du lambda-calcul (avec références, exceptions, etc.), ainsi qu’à comprendre et structurer des notions de concurrence (notamment la notion de bisimulation). Le cours fournit une introduction assez générale et complète au formalisme catégorique, et l’applique à la sémantique des langages de programmation.  Interpréter un langage dans un modèle s’apparente à une compilation, et les modèles offrent ainsi des occasions de retour sur la syntaxe : machines abstraites pour l’exécution des programmes, preuves de propriétés de programmes. Dans le même ordre d’idées, ce sont des observations sur un modèle du lambda-calcul qui ont conduit Girard à la logique linéaire, munie de connecteurs exprimant un contrôle sur l’usage des hypothèses vues comme ressources, ou bien plus récemment Thomas Ehrhard au lambda-calcul différentiel, qui relie de manière originale substitution et... formule de Taylor.}
		\cours{48h}{Modèles et langages pour la programmation des systèmes réactifs}{Marc POUZET, Timothy BOURKE}{M1/DENS}{Modèles et langages pour la programmation des systèmes réactifs.
			La programmation réactive touche désormais tous les domaines de l'informatique: controle en temps réel de systèmes embarqués (avions, voitures, train), interfaces graphiques, réseaux de capteurs, etc. Ce nouveau cours choisit de placer la correction et la modularité comme objectifs prioritaires.
			L’objectif de ce cours est l’étude et la mise en oeuvre des principes et modèles fondamentaux utilisés pour concevoir et réaliser les systèmes réactifs: les différents modèles de composition et de temps, le parallélisme déterministe, l’expression de modèles mathématiques dans des langages de haut niveau (modèle synchrone, modèle flot-de-données, modèle hybride), la spécification de propriétés temporelles (logique temporelle, observateurs), leur vérification formelle (model checking), la compilation de langages de haut niveau vers une cible séquentielle et parallèle dont on peut montrer la correction vis-à-vis du modèle.
			Le cours sera illustré à l'aide d'exemples écrits dans les langages ReactiveML (\url{http://rml.lri.fr}), Lustre (\url{http://www-verimag.imag.fr/DIST-TOOLS/SYNCHRONE/reactive-toolbox/}) et Zelus (\url{https://zelus.di.ens.fr}) et les outils de vérification formelle tels que Kind2 (\url{https://github.com/tinelli/kind2}) et Cubicle (\url{http://cubicle.lri.fr}).
			Le cours comporte un aspect fortement pratique dans les TD/TPs pour mettre en oeuvre des points du cours. Un projet de programmation ambitieux sera proposé (par ex., la programmation d'un drone, d'un arduino ou l'écriture d'un compilateur d'un langage dédié).}
		\cours{24h}{Optimication combinatoire}{Chien-Chung HUANG}{M1/DENS}{Ce cours est une introduction aux problèmes et concepts en optimisation combinatoire. Le but est d’apprendre à reconnaitre, transformer et résoudre ces problèmes d’optimisation. Nous regarderons de manière plus approfondie les notions de théorie des graphes.
			Plusieurs applications illustreront les techniques vues dans ce cours.}
		\cours{24h}{Optimisation convexe}{Adrien TAYLOR}{M1/DENS}{L'optimisation convexe est une branche relativement mature de l'optimisation continue, dans laquelle un grand nombre de problèmes peuvent être résolus de manières efficaces, avec une théorie forte.
			Ce cours est une introduction aux problèmes et concepts en optimisation convexe. Nous aborderons d'une part les propriétés et les familles importantes de problèmes convexes, afin de mieux les reconnaitre et de les résoudre. D'autre part, nous parcourrons les techniques algorithmiques classiques de résolution de tels problèmes, ainsi que l'étude de leurs complexités.
			Parmi les termes abordés: problèmes convexes, notamment linéaires, quadratiques, et semi-définis. Algorithmes du premier ordre: notamment (sous)gradients, proximaux, Frank-Wolfe, accélération, localisation, et stochasticité. Algorithmes du second ordre: Newton et points intérieurs.}
		\cours{48h}{Planification de mouvement en robotique et en animation graphique}{Julien CARPENTIER, Stéphane CARON, Yann DUBOIS DE MONT-MARIN}{M1/DENS}{La planification de mouvement s’intéresse au calcul automatique de chemins sans collision pour un système mécanique (robot mobile, bras manipulateur, personnage animé...) évoluant dans un environnement encombré d’obstacles. Les méthodes consistent à explorer l’espace des configurations du système : une configuration regroupe l’ensemble des paramètres permettant de localiser le système dans son environnement. Aux obstacles de l’environnement correspondent des domaines à éviter dans l’espace des configurations. La planification de mouvement pour le système mécanique se trouve ainsi ramenée au problème de la planification de mouvement d’un point dans une variété non simplement connexe.}
		\cours{72h}{Projet de recherche encadré}{Pierre SENELLART}{M1/DENS}{Les équipes de recherches du DI ENS proposent aux normaliens de réaliser un projet de recherche encadré en leur sein, sur une de leurs thématiques de recherche. Les sujets sont proposés par les équipes ou sollicités par les étudiants et sont réalisés sous la responsabilité de l’un des chercheurs de l’équipe, permanent ou non permanent. Les étudiants travaillent pendant l’ensemble du premier semestre de M1 sur le sujet, en parallèle des autres enseignements suivis. Les étudiants sont invités à participer à la vie de l’équipe (en particulier séminaires) pendant cette période, dans la limite de ce qui est compatible avec leur emploi du temps. Des réunions régulières (au minimum toutes les 2 semaines) de suivi de l’avancement sont mises en place. Le projet de recherche est évalué sur la base d’une soutenance, réalisée en présence de l’encadrant et d’un autre membre du département informatique, extérieur à l’équipe.
			Encadré par Mathieu DEHOUCK au Laboratoire LATTICE.}
		\midrule
		\multicolumn{4}{l}{\bf Enseignements en Mathématiques}         \\
		\cours{70h}{Algèbre 1}{Gaëtan CHENEVIER}{L3/ENS}{Cours de théorie des groupes, et particulièrement des groupes finis. Programme disponible sur \url{http://gaetan.chenevier.perso.math.cnrs.fr/AlgebreI.html}}
		\cours{70h}{Intégration et probabilités}{Anne-Laure DALIBARD}{L3/ENS}{Nous présenterons dans un premier temps la théorie de Lebesgue qui propose une construction de l’intégrale plus souple, plus générale et mieux adaptée aux passages à la limite, que l’intégrale de Riemann.
			Dans cette théorie, la notion de mesure et d’espace mesuré y joue un rôle central. Kolmogorov remarque en 1930, que ces notions peuvent être utilisées pour donner une bonne axiomatique de la théorie des probabilités.
			Ainsi, les probabilités pourraient être formellement vues comme un appendice de la théorie de l’intégration, mais nous verrons dans la deuxième partie de ce cours, que cette discipline développe ses propres questions et ses propres intuitions, notamment avec la notion d’indépendance d’événements, et l’étude des suites de variables aléatoires indépendantes (ou pas…).
			Les fameux théorèmes limites que sont la loi des grands nombres et le théorème central limite seront le point d’orgue du cours.}
		\cours{70h}{Topologie et calcul différentiel}{Djalil CHAFAÏ}{L3/ENS}{Cours de topologie générale, de théorie du calcul différentiel en dimension finie, et de théorie des équations différentielles ordinaires. Programme disponible sur \url{https://www.math.ens.psl.eu/formations/topologie-et-calcul-differentiel/}. Non-Validé}
		\cours{70h}{Analyse complexe}{Ariane MÉZARD}{L3/ENS}{Cours d'analyse complexe, programme disponible sur \url{https://www.math.ens.psl.eu/formations/analyse-complexe/}}
		\midrule
		\cours{63h}{Mathématiques des données}{Gabriel PEYRÉ}{M1/ENS}{Ce cours passe en revue les méthodes mathématiques et numériques fondamentales en sciences des données. La première partie du cours couvre les bases de la représentation et du traitement des données, en particulier la théorie de
			Shannon, le filtrage et les ondelettes. La deuxième partie présente l’optimisation convexe et non-convexe, dans la perspective de son utilisation en apprentissage automatique et en particulier pour les réseaux de neurones. Le cours est validé par un petit projet et un examen.}
			\cours{63h}{Topologie algébrique}{Muriel LIVERNET}{M1/ENS}{Ce cours est une introduction à la topologie algébrique. On associera aux espaces topologiques des invariants algébriques (groupe fondamental, groupes d’homologie, anneau de cohomologie, groupes d’homotopie supérieurs), et on donnera des applications de l’étude et du calcul de ces invariants à des problèmes de topologie.}
		\midrule
		\multicolumn{4}{l}{Enseignements dans d'autres disciplines.}   \\
		\cours{48h}{Turc Moderne}{Marc AYMES}{DENS}{Le cours niveau A1 (programme annuel) s’adresse aux débutants, ou aux locuteurs n’ayant pas entrepris d’apprentissage systématisé. Il vise à l’acquisition d’une maîtrise des principes grammaticaux du turc de Turquie. Le format du cours permet de s’initier au système phonétique et à la pratique orale.}
		\cours{24h}{Linguistics 101}{Salvador MASCARENHAS}{DENS}{This course is an introduction to the principled study of human natural language. Our chief interest and goal is to examine mathematically rigorous theories of (fragments of) the human cognitive capacity for language. To this end, we will introduce the fundamental concepts and theories in phonology, morphology, syntax, semantics, and pragmatics. We will discuss in some detail the philosophical foundations for the study of human language from this cognitive and mathematically intelligible perspective: Are all of the formal symbols that occur in our theories somehow in the head?
			In less detail due to time constraints, we will also introduce classics and in some cases illustrate recent work on the neurobiological bases of human language (neurolinguistics), the specifics of how the human capacity for language is deployed by humans in language production and comprehension (psycholinguistics), how children learn their native language effortlessly and with no substantive instruction (language acquisition), and what the points of contact and divergence are between the human faculty for language and the impressive behavioral achievements of modern large language models.}
		\cours{30h}{Introduction au droit public ; Introduction au droit privé}{Jean-Louis HALPÉRIN}{DENS}{Cours d'introduction au droit français. Je n'ai pas pu suivre entièrement le cours d'introduction au droit privé puisque je suis parti en stage. Enseignements disponibles sur \url{https://www.droit.ens.fr/Enseignements-de-droit-a-l-ENS-27.html}}
	\end{longtable}
\end{center}
\end{document}
