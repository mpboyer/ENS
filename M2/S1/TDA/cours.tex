\documentclass[info, math]{mpb-cours}

\title{Topological Data Analysis}
\author{D'après Julien Tierny et Frédéric Chazal}

\DeclareMathOperator{\St}{St}
\DeclareMathOperator{\Lk}{Lk}
\def\mK{\mathcal{K}}

\begin{document}
\bettertitle
\url{mailto:julien.tierny@sorbonne-universite.fr}
\url{mailto:frederic.chazal@inria.fr}
\url{https://julien-tierny.github.io/topologicalDataAnalysisClass.html}

\section*{Introduction}
Méthodes algorithmiques d'analyse topologique de données, particulièrement en science et en ingénierie.

Le but est de partir de données, sous forme de maillages et maillables, et de retrouver des
structures au sein de jeux de données.
Partant d'une carte (considérée comme jeu de données brutes), avec des features intéressantes,
pour pouvoir raisonner sur l'espace, on passe à une représentation abstraite, par exemple
comme un graphe, et c'est sur cette structure de données sous-jacente qu'on va raisonner.
Ici, on peut ajouter des filtres pour redéfinir le maillage et donc redéfinir le résultat
du raisonnement.
Plus généralement, on veut construire une carte à partir d'un jeu de données.
En astrophysique, par exemple, on modélise la croissance de l'univers à une grande échelle, on
la simule par une grille de voxel, on estime la densité de matière noire sur chaque voxel,
et on découvre une sorte de géométrie ressemblant à des neurones lorsqu'on trouve aussi des groupes
de galaxies, formant une "toile cosmique".
On peut calculer les connexions avec des complexes simpliciaux dits de Morse-Smale, dont on peut extraire
une structure de graphes.

Ainsi, on extrait de la structure d'un ensemble de données, de manière robuste et indépendante de l'échelle, par comparaison et extraction de propriétés.
Sous le capot, on fait:
\begin{itemize}
	\item Homologie Simpliciale
	\item Théorie de Morse
	\item Homologie Persistente
\end{itemize}

Pour des données numériques, étant données un échantillon de points dans un espace euclidien, par exemple, on peut les représenter et objectiver des représentations géométriques apparaissant.
On a des manières de mailler l'ensemble (triangulation de Delaunay, par exemple) qui amènent à des
indicateurs qui nous expliquent où sont répartis les données, par exemple avec des noyaux pour estimer la
densité.
Avec une fonction scalaire sur un maillage, on définit une filtration, et on regarde les propriétés de la fonction, comme les optima locaux et on en extrait une structure algébrique (complexe de Morse-Smale) qui nous donne une structure algébrique.
On obtient des générateurs, et des composantes "connexes".

On a ce genre de densité de pixels, par exemple la hauteur de surface de la mer qui permet
de remarquer les vortexs, en chimie quantique ou des spectrogrammes d'enregistrement vocaux.
On part d'un domaine géométrique et d'un signal sur ce domaine, signal qui exhibe des patternes géométriques
qu'on souhaite quantifier.
Ceci permet l'extraction de propriétés, la segmentation, la réduction de dimension et autres.
Dans le cas de points en grande dimension, on a une unique théorie qui s'applique très généralement.

En terme de logiciels, on a le TTK (ParaView >= 5.10) et Gudhi (bibliothèque python).

\section{Homologie Simplicale}
Les données reçues, parfois, vont contenir explicitement la géométrie avec une construction combinatoire.
On supposera qu'on aura une donnée d'entrée linéaire par morceau sur un complexe simplical.

\subsection{Complexe Simpliciaux}
\begin{definition}
	Un $d$-simplexe est l'enveloppe convexe $\sigma$ de $d + 1$ points affinement indépendants dans l'espace euclidien $\R^{n}$ avec $0 \leq d \leq n$.
	On dit que $d$ est la dimension du simplexe.
\end{definition}

Une ligne est un $1$-simplexe, un triangle un $2$-simplexe et un tétrahèdre un $3$-simplexe.

\begin{definition}
	Une face $\tau$ d'un simplexe $\sigma$ est un simplexe construit par un ensemble non vide des $d + 1$ poits définissant $\sigma$.
	On note $\tau \leq \sigma$ et $\tau_{i}$ une face de dimension $i$.
	On dit aussi que $\sigma$ est une coface de $\tau$.
\end{definition}
Selon la définition, on a $\sigma \leq \sigma$.

\begin{definition}
	Un complexe simplicial $\mathcal{K}$ est une collection finie non-vide de simplexes $\{\sigma_{i}\}$, telle que:
	\begin{enumerate}
		\item $\tau \leq \sigma \Rightarrow \tau \in \mathcal{K}$;
		\item $\sigma_{i} \cap \sigma_{j}$ est soit une face, soit vide.
	\end{enumerate}
\end{definition}

\begin{definition}
	L'étoile d'un simplexe $\sigma \in \mathcal{K}$ est l'ensemble des simplexes de $\mathcal{K}$ qui contiennent $\sigma$:
	\begin{equation*}
		\mathrm{St}(\sigma) = \left\{\tau \in \mathcal{K} \middle| \sigma \leq \tau \right\}
	\end{equation*}
	On note $\mathrm{St}_{d}(\sigma)$ les $d$-simplexes de $\mathrm{St}(\sigma)$.
\end{definition}
C'est l'ensemble des cofaces de $\sigma$ dans $\mK$. C'est le plus petit voisinage combinatoire autour d'un simplexe.

\begin{definition}
	Le lien d'un simplexe $\sigma$ est l'ensemble des faces de $\St(\sigma)$ disjointes de $\sigma$:
	\begin{equation*}
		\Lk(\sigma) = \left\{\tau \leq \sigma' \suchthat \sigma' \in \St(\sigma) \land \tau \cap \sigma = \emptyset \right\}
	\end{equation*}
	On définit de même le $d$-lien $\Lk_{d}(\sigma)$ en remplaçant $\St$ par $\St_{d}$ dans la définition
\end{definition}
C'est en quelque sorte la bordure du voisinage combinatoire de lui-même.

\medskip

En réalité on va considérer que les sommets (ou $0$-simplexes) sont des points, et que les $d$-simplexes sont des ensembles de points.
Ceci définit une notion de complexe simplicial abstrait, utile lorsqu'on n'a pas d'immersion dans un espace euclidien, ou alors dans un espace euclidien en trop grande dimension.
On relâche ici la condition d'intersection, puisqu'on n'a plus de structure géométrique de l'espace.

Un exemple de complexe simplicial abstrait est le complexe de Rips ou complexe de Vietori-Rips. Étant donné un nuage de points avec une métrique:
\begin{itemize}
	\item Le diamètre d'un ensemble $P$ est défini par: $\emptyset(P) = \sup\{d(x, y) \mid x, y \in P\}$
	\item On construit un complexe simplicial $p \leq p_{max}$ de sorte que tous $p + 1$ points dont le diamètre est plus petit qu'une valeur seuil $d_{max}$.
\end{itemize}
Le complexe de Rips est une généralisation de la notion de graphe de voisinage.

\begin{definition}
	L'espace sous-jacent à un complexe simplicial est l'union des simplexes du complexe.
\end{definition}

\begin{definition}
	La triangulation $\mathcal{T}$ d'un espace topologique $X$ est un complexe simplicial $\mK$ dont l'espace sous-jacent $\abs{\mK}$ est homéomorphe à $X$.
\end{definition}
Une triangulation d'un espace est donc un complexe simplicial abstrait.


\begin{definition}
	Une $d$-variété $M$ est un espace topologique dans lequel tout élément $m$ a un voisinage ouvert homéomorphe à une $d$-boule euclidienne.
\end{definition}

\begin{definition}
	La triangulation d'une $d$-variété est appelée $d$-variété linéaire par morceaux.
\end{definition}

Représenter en mémoire un complexe simplicial est très couteux: il faut, pour chaque dimension, une liste des hyperarêtes (ou $d$-simplexes) en les représentant par un indice de sommet.

\subsection{Homologie simpliciale}
\subsubsection{Rappels d'homotopie}
\begin{definition}
	Un chemin $p$ dans $C$ est un homéomorphisme d'un intervalle réel vers l'objet $C$.
	On dit que $C$ est connexe (par arcs) si pour tous deux points il existe un chemin dans $C$ les reliant.
\end{definition}

\begin{definition}
	Une composante connexe d'un objet est un sous-ensemble connexe (par arcs) maximal de l'objet.
\end{definition}

\begin{definition}
	Une homotopie entre deux fonctions continues $f$ et $g$ de $X$ vers $Y$ est une fonction continue $H: X \to  [0, 1] \to Y$ du produit d'un espace topologique $X$ par l'intervalle unité vers un espace topologique $Y$ de sorte que $H(x, 0) = f(x)$ et $H(x, 1) = g(x)$ pour tout $x \in X$.
	S'il existe une homotopie entre $f$ et $g$ on dit que $f$ et $g$ sont homotopes.
\end{definition}

\begin{definition}
	Si dans un espace $X$, tous les chemins entre tous deux points sont homotopes, on dit que $X$ est simplement connexe.
\end{definition}
Le disque est simplement connexe, mais pas le disque privé de $0$.

\begin{definition}
	La caractéristique d'Euler d'une triangulation $T$ d'un espace topologique est la somme alternée des nombres des $i$-simplexes:
	\begin{equation*}
		\chi(T) = \sum_{i=0}^{d} (-1)^{i}\abs{\sigma_{i}}
	\end{equation*}
\end{definition}

\begin{proposition}
	La caractéristique d'Euler est invariante par homéomorphisme.
\end{proposition}

\subsubsection{Groupes d'homologie}
\begin{definition}
	Une $p$-chaine est une somme (formelle) de $p$-simplexes. On suppose que l'opérateur somme est défini modulo $2$.
\end{definition}
Ici, la somme est réellement la différence symmétrique (ou la disjonction exclusive) sur $\mathbb{F}_{2}^{\abs{\sigma_{p}}}$, et définit le groupe $C_{p}$ des $p$-chaînes.
Le rang de $C_{p}$ est $\abs{\sigma_{p}}$ et son ordre est $2^{\abs{\sigma_{p}}}$


On peut généraliser à des coefficients plus généraux.
Informatiquement, il faut voir la notion de chaîne comme un masque binaire sur l'ensemble des $p$-simplexes.

\begin{definition}
	L'opérateur de bordure $\partial$ d'un $p$-simplexe renvoie la $(p-1)$-chaîne des $(p-1)$-faces du simplexe.
	On l'étend aux $p$-chaînes comme un morphisme de $C_{p} \to C_{p - 1}$.
\end{definition}
Pour un triangle, c'est l'ensemble de ses arêtes. Pour une arête, c'est l'ensemble des extrémités.

\begin{definition}
	Un $p$-cycle est une $p$-chaîne dont la bordure est vide.
	On définit $Z_{p}$ le groupe des $p$-cycles comme sous-groupe de $C_{p}$.
\end{definition}

\begin{definition}
	Le groupe $B_{p}$ des $p$-bordures est l'image de $C_{(p + 1)}$ par $\partial$.
\end{definition}

\begin{lemme}
	Pour tout $x \in C_{p}, p \geq 2$, $\partial \partial x = 0$.
\end{lemme}
\begin{proof}
	Il suffit de vérifier le résultat sur les $p$-simplexes et d'étendre par somme.
	Puisque pour tout $(p-2)$-faces $\tau$ on a exactement $2$ $(p-1)$-co-faces de $\tau$ dans un $p$-simplexe $\sigma$, on a le résultat.
\end{proof}

On obtient directement:
\begin{proposition}
	$B_{p}$ est un sous-groupe de $Z_{p}$.
\end{proposition}
Si on a trois $1$-simplexes $e_{1}, e_{2}, e_{3}$ mais pas leur coface commune $\{e_{1}, e_{2}, e_{2}\}$, on a un exemple d'inclusion stricte.
Ceci nous amène à définir la notion de trou, en isolant les cycles:

\begin{definition}
	Le $p$-ème groupe d'homologie $H_{p}$ est le quotient de $Z_{p}$ par $B_{p}$.
\end{definition}
\begin{proof}
	$B_{p}$ étant un sous-groupe de $Z_{p}$, $H_{p}$ est bien défini.
\end{proof}

Géométriquement, on peut étendre un $p$-cycle à un autre $p$-cycle lorsqu'ils encapsulent le même "trou", c'est-à-dire lorsque qu'on peut "étendre" le premier cycle en encapsulant un $(p + 1)$-simplexe.
Une classe d'homologie est un élément de $H_{p}$, ou plutôt sa classe d'équivalence dans $Z_{p}$.

Pour calculer $\abs{H_{p}}$, on énumère $C_{p}$, on élimine les chaînes de bordure non-vide pour calculer $Z_{p}$, et on peut ensuite énumérer les classes d'homologie.

\begin{definition}
	On définit le $p$-ème nombre de Betti $\beta_{p}$ comme le rang du groupe $H_{p}$. Ici, c'est $\log_{2}\abs{H_{p}}$.
\end{definition}
La formule logarithmique pour $\beta_{p}$ vient du calcul modulo $2$ dans notre opération de groupe.

\begin{proposition}
	La caractéristique d'Euler d'une triangulation $T$ d'un espace topologique $X$ de dimension $d$ vérifie:
	\begin{equation*}
		\chi(T) = \sum_{i = 0}^{d} (-1)^{i}\beta_{i}(T)
	\end{equation*}
\end{proposition}

\section{Homologie Persistente}

\section{Fonctions de Morse}

\section{Inférence Topologique}

\section{Théorie de Morse Discrète}

\section{Noyaux et Statistiques}

\end{document}
