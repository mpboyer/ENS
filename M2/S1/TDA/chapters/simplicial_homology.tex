\section{Homologie Simplicale}
Les données reçues, parfois, vont contenir explicitement la géométrie avec une construction combinatoire.
On supposera qu'on aura une donnée d'entrée linéaire par morceau sur un complexe simplical.

\subsection{Complexe Simpliciaux}
\begin{definition}
	Un \define{$d$-simplexe} est l'enveloppe convexe $\sigma$ de $d + 1$ points affinement indépendants dans l'espace euclidien $\R^{n}$ avec $0 \leq d \leq n$.
	On dit que $d$ est la \define{dimension} du simplexe.
\end{definition}

Une ligne est un $1$-simplexe, un triangle un $2$-simplexe et un tétrahèdre un $3$-simplexe.

\begin{definition}
	Une \define{face} $\tau$ d'un simplexe $\sigma$ est un simplexe construit par un ensemble non vide des $d + 1$ poits définissant $\sigma$.
	On note $\tau \leq \sigma$ et $\tau_{i}$ une face de dimension $i$.
	On dit aussi que $\sigma$ est une \define{coface} de $\tau$.
\end{definition}
Selon la définition, on a $\sigma \leq \sigma$.

\begin{definition}
	Un \define{complexe simplicial} $\mathcal{K}$ est une collection finie non-vide de simplexes $\{\sigma_{i}\}$, telle que:
	\begin{enumerate}
		\item $\tau \leq \sigma \Rightarrow \tau \in \mathcal{K}$;
		\item $\sigma_{i} \cap \sigma_{j}$ est soit une face, soit vide.
	\end{enumerate}
\end{definition}

\begin{definition}
	\define{L'étoile} d'un simplexe $\sigma \in \mathcal{K}$ est l'ensemble des simplexes de $\mathcal{K}$ qui contiennent $\sigma$:
	\begin{equation*}
		\mathrm{St}(\sigma) = \left\{\tau \in \mathcal{K} \middle| \sigma \leq \tau \right\}
	\end{equation*}
	On note $\mathrm{St}_{d}(\sigma)$ les $d$-simplexes de $\mathrm{St}(\sigma)$.
\end{definition}
C'est l'ensemble des cofaces de $\sigma$ dans $\mK$. C'est le plus petit voisinage combinatoire autour d'un simplexe.

\begin{definition}
	Le \define{lien} d'un simplexe $\sigma$ est l'ensemble des faces de $\St(\sigma)$ disjointes de $\sigma$:
	\begin{equation*}
		\Lk(\sigma) = \left\{\tau \leq \sigma' \suchthat \sigma' \in \St(\sigma) \land \tau \cap \sigma = \emptyset \right\}
	\end{equation*}
	On définit de même le \define{$d$-lien} $\Lk_{d}(\sigma)$ en remplaçant $\St$ par $\St_{d}$ dans la définition
\end{definition}
C'est en quelque sorte la bordure du voisinage combinatoire de lui-même.

\medskip

En réalité on va considérer que les sommets (ou $0$-simplexes) sont des points, et que les $d$-simplexes sont des ensembles de points.
Ceci définit une notion de complexe simplicial abstrait, utile lorsqu'on n'a pas d'immersion dans un espace euclidien, ou alors dans un espace euclidien en trop grande dimension.
On relâche ici la condition d'intersection, puisqu'on n'a plus de structure géométrique de l'espace.

Un exemple de complexe simplicial abstrait est le complexe de Rips ou complexe de Vietori-Rips. Étant donné un nuage de points avec une métrique:
\begin{itemize}
	\item Le diamètre d'un ensemble $P$ est défini par: $\emptyset(P) = \sup\{d(x, y) \mid x, y \in P\}$
	\item On construit un complexe simplicial $p \leq p_{max}$ de sorte que tous $p + 1$ points dont le diamètre est plus petit qu'une valeur seuil $d_{max}$.
\end{itemize}
Le complexe de Rips est une généralisation de la notion de graphe de voisinage.

\begin{definition}
	\define{L'espace sous-jacent} à un complexe simplicial est l'union des simplexes du complexe.
\end{definition}

\begin{definition}
	La \define{triangulation} $\mathcal{T}$ d'un espace topologique $X$ est un complexe simplicial $\mK$ dont l'espace sous-jacent $\abs{\mK}$ est homéomorphe à $X$.
\end{definition}
Une triangulation d'un espace est donc un complexe simplicial abstrait.


\begin{definition}
	Une \define{$d$-variété} $M$ est un espace topologique dans lequel tout élément $m$ a un voisinage ouvert homéomorphe à une $d$-boule euclidienne.
\end{definition}

\begin{definition}
	La triangulation d'une $d$-variété est appelée \define{$d$-variété linéaire par morceaux}.
\end{definition}

Représenter en mémoire un complexe simplicial est très couteux: il faut, pour chaque dimension, une liste des hyperarêtes (ou $d$-simplexes) en les représentant par un indice de sommet.

\subsection{Homologie simpliciale}
\subsubsection{Rappels d'homotopie}
\begin{definition}
	Un \define{chemin} $p$ dans $C$ est un homéomorphisme d'un intervalle réel vers l'objet $C$.
	On dit que $C$ est \define{connexe (par arcs)} si pour tous deux points il existe un chemin dans $C$ les reliant.
\end{definition}

\begin{definition}
	Une \define{composante connexe} d'un objet est un sous-ensemble connexe (par arcs) maximal de l'objet.
\end{definition}

\begin{definition}
	Une \define{homotopie} entre deux fonctions continues $f$ et $g$ de $X$ vers $Y$ est une fonction continue $H: X \to  [0, 1] \to Y$ du produit d'un espace topologique $X$ par l'intervalle unité vers un espace topologique $Y$ de sorte que $H(x, 0) = f(x)$ et $H(x, 1) = g(x)$ pour tout $x \in X$.
	S'il existe une homotopie entre $f$ et $g$ on dit que $f$ et $g$ sont \define{homotopes}.
\end{definition}

\begin{definition}
	Si dans un espace $X$, tous les chemins entre tous deux points sont homotopes, on dit que $X$ est \define{simplement connexe}.
\end{definition}
Le disque est simplement connexe, mais pas le disque privé de $0$.

\begin{definition}
	La \define{caractéristique d'Euler} d'une triangulation $T$ d'un espace topologique est la somme alternée des nombres des $i$-simplexes:
	\begin{equation*}
		\chi(T) = \sum_{i=0}^{d} (-1)^{i}\abs{\sigma_{i}}
	\end{equation*}
\end{definition}

\begin{proposition}
	La caractéristique d'Euler est invariante par homéomorphisme.
\end{proposition}

\subsubsection{Groupes d'homologie}
\begin{definition}
	Une \define{$p$-chaine} est une somme (formelle) de $p$-simplexes. On suppose que l'opérateur somme est défini modulo $2$.
\end{definition}
Ici, la somme est réellement la différence symmétrique (ou la disjonction exclusive) sur $\mathbb{F}_{2}^{\abs{\sigma_{p}}}$, et définit le groupe $C_{p}$ des $p$-chaînes.
Le rang de $C_{p}$ est $\abs{\sigma_{p}}$ et son ordre est $2^{\abs{\sigma_{p}}}$


On peut généraliser à des coefficients plus généraux.
Informatiquement, il faut voir la notion de chaîne comme un masque binaire sur l'ensemble des $p$-simplexes, et l'addition comme la disjonction exclusive.

\begin{definition}
	\define{L'opérateur de bordure} $\partial$ d'un $p$-simplexe renvoie la $(p-1)$-chaîne des $(p-1)$-faces du simplexe.
	On l'étend aux $p$-chaînes comme un morphisme de $C_{p} \to C_{p - 1}$.
\end{definition}

Pour un triangle, c'est l'ensemble de ses arêtes. Pour une arête, c'est l'ensemble des extrémités.

\begin{proposition}
	L'opérateur de bordure définit une suite exacte appelée le complexe de chaîne associé au complexe $K$ de dimension $d$:
	\begin{equation*}
		\{0\} \to C_{d}(K) \xrightarrow{\partial} C_{d - 1}(K)
		\xrightarrow{\partial} \cdots
		\xrightarrow{\partial} C_{k + 1}(K)
		\xrightarrow{\partial} C_{k}(K)
		\xrightarrow{\partial} \cdots
		\xrightarrow{\partial} C_{1}(K)
		\xrightarrow{\partial} C_{0}(K)
		\xrightarrow{\partial} O
	\end{equation*}
\end{proposition}

\begin{definition}
	Un \define{$p$-cycle} est une $p$-chaîne dont la bordure est vide.
	On définit \define{$Z_{p}$} le groupe des $p$-cycles comme sous-groupe de $C_{p}$.
\end{definition}

\begin{definition}
	Le groupe \define{$B_{p}$} des $p$-bordures est l'image de $C_{(p + 1)}$ par $\partial$.
\end{definition}

\begin{lemme}
	Pour tout $x \in C_{p}, p \geq 2$, $\partial \partial x = 0$.
\end{lemme}
\begin{proof}
	Il suffit de vérifier le résultat sur les $p$-simplexes et d'étendre par somme.
	Puisque pour tout $(p-2)$-faces $\tau$ on a exactement $2$ $(p-1)$-co-faces de $\tau$ dans un $p$-simplexe $\sigma$, on a le résultat.
\end{proof}

On obtient directement:
\begin{proposition}
	$B_{p}$ est un sous-groupe de $Z_{p}$.
\end{proposition}
Si on a trois $1$-simplexes $e_{1}, e_{2}, e_{3}$ mais pas leur coface commune $\{e_{1}, e_{2}, e_{2}\}$, on a un exemple d'inclusion stricte.
Ceci nous amène à définir la notion de trou, en isolant les cycles:

\begin{definition}
	Le \define{$p$-ème groupe d'homologie $H_{p}$} est le quotient de $Z_{p}$ par $B_{p}$.
\end{definition}
\begin{proof}
	$B_{p}$ étant un sous-groupe de $Z_{p}$, $H_{p}$ est bien défini.
\end{proof}

Géométriquement, on peut étendre un $p$-cycle à un autre $p$-cycle lorsqu'ils encapsulent le même "trou", c'est-à-dire lorsque qu'on peut "étendre" le premier cycle en encapsulant un $(p + 1)$-simplexe.
Une classe d'homologie est un élément de $H_{p}$, ou plutôt sa classe d'équivalence dans $Z_{p}$.

Pour calculer $\abs{H_{p}}$, on énumère $C_{p}$, on élimine les chaînes de bordure non-vide pour calculer $Z_{p}$, et on peut ensuite énumérer les classes d'homologie.

\begin{definition}
	On définit le \define{$p$-ème nombre de Betti $\beta_{p}$} comme le rang du groupe $H_{p}$. Ici, c'est $\log_{2}\abs{H_{p}}$.
\end{definition}
La formule logarithmique pour $\beta_{p}$ vient du calcul modulo $2$ dans notre opération de groupe.

\begin{proposition}
	La caractéristique d'Euler d'une triangulation $T$ d'un espace topologique $X$ de dimension $d$ vérifie:
	\begin{equation*}
		\chi(T) = \sum_{i = 0}^{d} (-1)^{i}\beta_{i}(T)
	\end{equation*}
\end{proposition}

On a des interprétentations des nombres de Betti en faible dimension.
Par exemple, $\beta_{0}(K)$ est le nombre de composantes connexes de $K$.

