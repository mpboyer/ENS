\section{Homologie Persistente}
\subsection{Invariance Topologique et Filtrations}
\subsubsection{Homologie Singulière}
On a le résultat important suivant, qui va nous permettre de simplifier la manière de définir l'homologie,
notamment dans la représentation des simplexes:
\begin{thm}
	Si $K$ et $K'$ sont des complexes simpliciaux de supports homéomorphes, leurs groupes d'homologie sont
	isomorphes et leurs nombres de Betti sont égaux.
\end{thm}
\begin{proof}
	\og Débrouille-toi, normalien.\fg
\end{proof}

On va donc définir l'homologie singulière, sur tout espace topologique, en considérant à homotopie près. On note $\Delta_{k}$ le $k$-simplexe standard dans $\R^{k}$.
\begin{definition}
	Un \define{$k$-simplexe singulier} dans un espace topologique $X$ est une fonction continue $\sigma: \Delta_{k}\to X$.
\end{definition}

On reprends les constructions de l'homologie simpliciale pour les complexes singuliers: c'est la définition de l'homologie singulière.

\begin{proposition}
	L'homologie singulière est définie pour tout espace $X$. Si $X$ est homotopiquement équivalent au support
	d'un complexe simpliciel, les homologies singulières et simpliciales coïncident.
\end{proposition}

\begin{proposition}
	Si $f: X \to Y$ est continue, et $\sigma: \Delta_{k} \to X$ est un simplexe dans $X$, alors $f\circ \sigma$ est un simplexe dans $Y$, et $f$ définit donc une application linéaire entre les groupes d'homologie:
	\begin{equation*}
		f_{\sharp}: H_{k}(X) \to H_{k}(Y)
	\end{equation*}
	Si $f$ est un homéomorphisme ou une équivalence d'homotopie, alors $f_{\sharp}$ est un isomorphisme.
\end{proposition}

\begin{remarque}
	En particulier, si $X \subset Y$, l'application d'inclusion induit une application linéaire d'homologie.
\end{remarque}

\subsubsection{Filtrations}
\begin{definition}
	Un \define{complexe simplicial filtré (ou une filtration)} $\K$ construit sur un ensemble $X$ est une famille
	$\left\{K_{a} \suchthat a \in T\right\}$, où $T \subseteq \R$, de sous-complexes d'un
	certain complexe simplicial fixé $K$ avec ensemble de sommets $X$ de sorte que $K_{a} \subseteq K_{b}$
	pour tout $a \leq b$.
\end{definition}

L'homologie persistente d'un complexe simplical filtré encode l'évolution de l'homologie des sous-complexes.

\begin{definition}
	Une \define{filtration d'un complexe simplicial fini} $K$ est une séquence de sous-complexes telle que:
	\begin{enumerate}
		\item $\emptyset = K^{0} \subset K^{1} \subset \cdots \subset K^{m} = K$
		\item $K^{i + 1} = K^{i} \cup \sigma^{i + 1}$ où $\sigma^{i + 1}$ est un simplexe de $K$.
	\end{enumerate}
\end{definition}
La famille des ensembles de sous-niveau pour une fonction est un exemple de filtration.
On verra plus bas comment définir de telles filtrations dans plus de cas.
On a toutefois un algorithme pour calculer itérativement l'homologie simpliciale d'un complexe étant donné
une filtration de celui-ci:
\begin{algorithm}
	\caption{Calcul d'homologie (simpliciale)}
	\label{alg:betti_nop}
	\begin{algorithmic}
		\Input{Une filtration $\left(K^{i}\right)_{i \leq m}$ d'un complexe simplical $K$ en dimension $d$}
		\EndInput
		\State {$\beta_{i} \gets 0$}
		\For{$i \in \left\{1, \ldots, m\right\}$}
		\State {$k \gets \dim \sigma^{i} - 1$}
		\If{$\sigma^{i}$ est dans un $\left(k + 1\right)$-cycle de $K^{i}$}
		\State{$\beta_{k + 1} \gets \beta_{k + 1} + 1$}
		\Else
		\State{$\beta_{k} \gets \beta_{k} - 1$}
		\EndIf
		\EndFor
		\Return{$\left(\beta_{0}, \ldots, \beta_{d}\right)$}
	\end{algorithmic}
\end{algorithm}

\begin{definition}
	Un $(k + 1)$-simplexe $\sigma^{i}$ est dit \define{positif} s'il est contenu dans un $(k + 1)$-cycle de $K^{i}$.
	Il est dit \define{négatif} sinon.
\end{definition}
Un $(k + 1)$-simplexe positif crée un $(k + 1)$-cycle dans $K^{i}$.
Un $(k + 1)$-simplexe négatif détruit un $k$-cycle dans $K^{i}$.

On veut donc vérifier:
\begin{equation*}
	\beta_{k}(K) = \abs{\text{$k$-simplexes positifs}} - \abs{\text{$(k - 1)$-simplexes négatifs}}
\end{equation*}

On a:
\begin{lemme}
	\label{lem:filter_cycle}
	Si $\sigma^{i}$ est un $k$-simplexe positif, il existe un unique $k$-cycle $c_{\sigma}$ tel que:
	\begin{enumerate}
		\item $c_{\sigma}$ n'est pas une bordure dans $K^{i}$
		\item $c_{\sigma}$ contient $\sigma^{i}$ mais pas d'autre $k$-simplexe positif.
	\end{enumerate}
\end{lemme}
\begin{proof}
	Par induction sur l'ordre d'apparition des simplexes dans la filtration.
\end{proof}

\begin{proof}[Preuve de correction de l'algorithme \ref{alg:betti_nop}]
	Si $\sigma^{i}$ est contenu dans un $(k + 1)$-cycle $c$ de $K^{i}$, alors ce cycle n'est pas une bordure dans $K^{i}$.
	De plus, $c$ ne peut pas être homologue à un cycle dans $K^{i - 1}$, et donc $\beta_{k + 1}\left(K^{i}\right) \geq \beta_{k + 1}(K^{i - 1}) + 1$.
	Si $\sigma^{i}$ n'est contenu dans aucun $(k + 1)$-cycle $c$ de $K^{i}$, alors $\partial\sigma^{i}$ n'est pas une bordure dans $K^{i - 1}$ et donc $\beta_{k}(K^{i}) \leq \beta_{k}\left(K^{i - 1}\right) - 1$.
\end{proof}

Cela pose quelques questions, qui vont nous conduire à introduire l'homologie persistente.

\subsection{Homologie Persistente}
\subsubsection{Sur les fonctions}
Pour définir les diagrammes de persistence pour une fonction $f: X \to \R$, on étudie ses
ensembles de sous-niveau.
On représente, pour chaque dimension $d$ d'homologie, la "durée de vie" d'une propriété topologique de dimension $d$.
Ces propriétés topologiques sont, entre autres, observées par la variation du $d$-ème nombre de Betti.
On représente alors sur un graphe $2$-D, en abscisse, la valeur $x$ pour laquelle une propriété apparaît, et en ordonnée la valeur $x'$ pour laquelle la propriété disparaît.
On a nécessairement $x' > x$.
On notera $D_{f, d}$ le diagramme défini ci-dessus, comme son ensemble de points du plan $\R_{+}^{2}$.

On définit alors une distance sur deux tels diagrammes:
\begin{definition}
	La \define{distance infinie de Wasserstein, ou distance du goulot} entre deux diagrammes $D_{1}$ et $D_{2}$ est définie par:
	\begin{equation*}
		d_{B}\left(D_{1}, D_{2}\right) = \inf_{\gamma \in \Gamma}\sup_{p \in D_{1}} \ninf{p - \gamma(p)}
	\end{equation*}
	où $\Gamma$ est l'ensemble des bijections entre $D_{1}$ et $D_{2}$.
\end{definition}
On note que, les normes étant équivalentes sur $\R^{2}$, la distance de Wasserstein $2$ est, à un facteur près, la distance infinie.

\begin{remarque}
	Pour pouvoir obtenir des bijections, on doit souvent \emph{augmenter} les diagrammes, en projetant les points de l'un sur la diagonale de l'autre et réciproquement.
\end{remarque}

On aura alors un théorème important de stabilité:
\begin{thm}
	Pour toutes fonctions \emph{dociles} $f, g: \mathbb{X}\to \R$, $d_{B}(D_{f}, D_{g}) \leq \ninf{f - g}$.
\end{thm}


Cette définition, avec les mains, va être précisée et étendue plus bas.

\subsubsection{Sur les filtrations}
On va maintenant étendre la notion de diagrammes de persistences aux filtrations de complexes simpliciaux.
On a une relation fondamentale:
\begin{proposition}
	Si $t \leq t'$, $f^{-1}\left(\left]-\infty, t\right]\right) \subseteq f^{-1}\left(\left]-\infty, t'\right]\right)$.
	Si $f$ est définie sur les sommets d'un complexe simplicial $K$, et étendue de sorte que
	\begin{equation*}
		f\left(\sigma = \left[v_{0}, \ldots, v_{k}\right]\right) = \max f\left(v_{i}\right),
	\end{equation*}
	alors les ensembles de sous-niveau de $f$ définissent une filtration du complexe simplcial $K$.
\end{proposition}

Il suffit maintenant d'adapter l'algorithme \ref{alg:betti_nop} ci-dessus pour maintenir une base d'homologie et les paires naissance-mort d'une propriété.
On notera $H_{k}^{i} = H_{k}\left(K^{i}\right)$, et on va construire des bases par récurrence (les $H_{k}$ sont des $\mathbb{F}_{2}$-espaces vectoriels).

La base de $H_{k}^{0}$ est vide, puisque l'ensemble est vide.
Si on a construit une base de $H_{k}^{i - 1}$, on a deux cas:
\begin{enumerate}
	\item Si $\sigma^{i}$ est un $k$-simplexe positif, alors on ajoute la classe d'homologie du cycle
	      $c^{i}$ associé à $\sigma^{i}$ par le Lemme \ref{lem:filter_cycle} à la base de
	      $H_{k}^{i - 1}$ pour obtenir une base de $H_{k}^{i}$.
	\item Si $\sigma^{i}$ est un $(k + 1)$-simplexe négatif:
	      \begin{itemize}
		      \item On dénote $c^{j_{1}}, \ldots, c^{j_{p}}$ les cycles associés aux simplexes
		            positifs $\sigma^{j_{1}}, \ldots, \sigma^{j_{p}}$ de la base de $H_{k}^{i - 1}$.
		      \item On pose $d = \partial \sigma^{j} = \sum_{k = 1}^{p}\epsilon_{k}c^{j_{k}} + b$
		      \item On pose $l(i) = \max \left\{j_{k} \suchthat \epsilon_{k} = 1\right\}$
		      \item On enlève la classe d'homologie de $c^{l(i)}$ pour obtenir une base de $H_{k}^{i}$.
	      \end{itemize}
\end{enumerate}

Ceci explique comment modifier l'algorithme \ref{alg:betti_nop} pour calculer les diagrammes de persistence.
Cependant, avant de réécrire l'algorithme, on va s'intéresser à un test algorithmique pour vérifier que $\sigma^{j}$ est positif ou négatif.
Pour ce faire, on introduit la matrice de l'opérateur de bordure. On rappelle qu'on se donne une filtration:
d'un complexe simplicial fini $d$-dimensionnel
$\emptyset = K^{0} \subset K^{1} \subset \cdots \subset K^{m} = K$ telle que
$K^{i + 1} = K^{i} \cup \sigma^{i + 1}$ où $\sigma^{i + 1}$ est un simplexe de $K$.

\begin{definition}
	On pose $M = \left(m_{i, j}\right)_{1\leq i, j \leq m}$ telle que $m_{i, j} = 1 \text{ si, et seulement si } \sigma^{i} \text{ est une face de } \sigma^{j}$ et vaut $0$ sinon.
	C'est la \define{matrice de l'opérateur d'inclusion.}
\end{definition}
Pour toute colonne $C_{j}$, on définit donc $l(j)$ par:
\begin{equation*}
	(i = l(j)) \Leftrightarrow (m_{i, j} = 1 \land m_{i', j} = 0, \forall i' > i)
\end{equation*}

On obtient une version matricielle de l'algorithme de persistence:
\begin{algorithm}
	\caption{Algorithme de Persistence, version Matricielle}
	\label{alg:persistence_mat}
	\begin{algorithmic}
		\Input{Une filtration $\emptyset = K^{0} \subseteq \cdots \subseteq K^{m} = K$ d'un complexe simplical $d$-dimensionnel de sorte que $K^{i + 1} = K^{i} \cup \sigma^{i + 1}$ où $\sigma^{i + 1}$ est un simplexe de $K$}
		\EndInput
		\State{Calculer la matrice $M$ de l'opérateur de bordure.}
		\For{$j \in \left\{0, \ldots, m\right\}$}
		\While{$\exists j' < j, l(j') = l(j)$}
		\State{$C_{j} \gets C_{j} + C_{j'} \mod 2$}
		\EndWhile
		\EndFor
		\Return{Paires $\left(l(j), j\right)$}
	\end{algorithmic}
\end{algorithm}

Dans le pire des cas, on a un algorithme en $\O(m^{3})$.

\begin{proof}
	À chaque étape de l'algorithme, la colonne $C_{j}$ représente une chaîne de la forme:
	\begin{equation*}
		\partial \left(\sigma^{j} + \sum_{i < j} \epsilon_{i}\sigma^{i}\right), \epsilon_{i}\in \{0, 1\}
	\end{equation*}
	À la fin de l'algorithme, si $j$ est tel que $l(j)$ est défini, alors $\sigma^{l(j)}$ est un simplexe positif.
	Donc si à la fin de l'algorithme, $C_{j}$ est nulle alors $\sigma^{j}$ est positif.
	Donc, si $C_{j}$ n'est pas nulle, alors $(\sigma^{l(j)}, \sigma^{j})$ est une paire de persistence.
\end{proof}

\begin{definition}
	On représente sur un \define{diagramme de persistence} les \define{paires de persistence} $\left(\sigma^{l(j)}, \sigma^{j}\right)$ par $\left(l\left(j\right), j\right)$ ou $\left(f\left(\sigma^{l(j)}\right), f\left(\sigma^{j}\right)\right)$.
	On ajoute au diagramme la diagonale $\{y = x\}$ et, pour chaque simplexe positif qui n'est pas dans une paire $\sigma^{i}$, le point $(i, + \infty)$.
\end{definition}

\begin{definition}
	Si $D_{1}, D_{2}$ sont deux diagrammes (potentiellement augmentés pour avoir le même cardinal):
	\begin{description}
		\item[La Distance du Goulot] est définie par:
		      \begin{equation*}
			      d_{B}^{\infty}(D_{1}, D_{2}) = \inf_{\gamma \in \Gamma} \sup_{p \in D_{1}} \ninf{p - \gamma(p)}
		      \end{equation*}
		\item[La Distance $p$-Wasserstein] est définie, pour $p \geq 1$ par:
		      \begin{equation*}
			      W_{p}(D_{1}, D_{2}) = \inf_{\gamma \in \Gamma} (\sum_{\rho \in D_{1}} \norm{\rho - \gamma(\rho)}_{p}^{p})^{\frac{1}{p}}
		      \end{equation*}
	\end{description}
	Dans les deux cas, $\Gamma$ est l'ensemble des bijections entre $D_{1}$ et $D_{2}$.
\end{definition}

\begin{remarque}
	Ces deux définitions peuvent être vues comme le coût du transport optimal pour la norme infinie et la norme $p$.
	Ces deux définitions sont par ailleurs équivalentes à un facteur près, ce qui est important pour les théorèmes de stabilité.
\end{remarque}

\begin{thm}
	Si $f, g : X \to \R$ sont \emph{dociles}, on a:
	\begin{equation*}
		d_{B}^{\infty}(D_{f}, D_{g}) \leq \ninf{f - g}
	\end{equation*}
	où $D_{\phi}$ est le diagramme de persistence de la filtration associée aux ensembles de sous-niveau de $\phi$ sur $X$.
\end{thm}
On reviendra plus tard sur la notion de docilité.

\subsection{Calcul de filtrations}
\subsubsection{Complexes de \v{C}ech, de Vietoris-Rips, et autres}
\begin{definition}
	On considère un recouvrement $\mU$ par des ouverts d'un espace topologique $X$.
	Le \define{complexe de \v{C}ech} $C\left(\mU\right)$ associé au recouvrement $\mU$ vérifie:
	\begin{itemize}
		\item L'ensemble de sommets de $C\left(\mU\right)$ est l'ensemble $\mU$.
		\item $\left[U_{0}, \ldots, U_{k}\right]$ est un $k$-simplexe dans $C\left(\mU\right)$ si et seulement si $\cap U_{j} \neq \emptyset$.
	\end{itemize}
\end{definition}

\begin{thm}[Nerveux (Leray)]
	Si toutes les intersections entre les ouverts de $\mU$ sont soit vides soit contractibles, alors $C(\mU)$ et $X$ sont homotopiquement équivalents.
\end{thm}

Si on se donne plutôt un nuage de point $V$ dans un espace métrique $(X, d)$ et un réel $\alpha$.
\begin{definition}
	Le \define{complexe de \v{C}Čech} $\cech\left(V, \alpha\right)$ est le complexe simplicial
	filtré indexé par $\R$ dont l'ensemblde sommets est $V$ et tel que:
	\begin{equation*}
		\sigma = \left[p_{0}, \ldots, p_{k}\right] \in \cech\left(V, \alpha\right) \Leftrightarrow \bigcap_{i = 0}^{k} B(p_{i}, \alpha) \neq \emptyset
	\end{equation*}
\end{definition}

\begin{definition}
	Le \define{complexe de Vietoris-Rips} $\rips(V)$ est le complexe simplicial filtré indexé par $\R$ dont l'ensemble de sommets est $V$
	et est défini par:
	\begin{equation*}
		\sigma = \left[p_{0}, \ldots, p_{k}\right] \in \cech\left(V, \alpha\right) \Leftrightarrow \forall i, j \in \left\{0, \ldots, k\right\}, d\left(p_{i}, p_{j}\right) \leq \alpha
	\end{equation*}
\end{definition}

\begin{proposition}
	On a, pour tout $\alpha > 0$:
	\begin{equation*}
		\cech\left(L, \frac{\alpha}{2}\right) \subseteq \rips\left(L, \alpha\right) \subseteq \cech\left(L, \alpha\right)
	\end{equation*}
\end{proposition}

\begin{definition}
	Si $V = \{p_{1}, \ldots, p_{n}\} \subseteq \R^{d}$, on définit la \define{cellule de Voronoï} associée à $p_{i}$ par:
	\begin{equation*}
		\mathcal{Vor}(p_{i}) = \left\{x \in \R^{d}\suchthat \forall j, \norm{x - p_{i}} \leq \norm{x - p_{j}}\right\}
	\end{equation*}
	Le \define{complexe de Delaunay} $\mathcal{D}(P)$ est le nerf de la couverture faite par les cellules de Voronoï.
	\define{L'alpha complexe} $\A(P, \alpha)$, pour $\alpha \geq 0$ est le nerf de la famille:
	\begin{equation*}
		(\mathrm{Vor}(p_{i}) \cap B(p_{i}, \sqrt{\alpha}))_{i = 1, \ldots, n}
	\end{equation*}
\end{definition}

\begin{thm}
	$\A(P, \alpha)$ est homotopie équivalent à $\bigcup_{i = 1}^{n} B(p_{i}, \sqrt{\alpha})$.
\end{thm}

\subsubsection{Stabilité}

On va utiliser ci-dessous la distance de Hausdorff:
\begin{equation*}
	d_{H}(A, B) = \max \left\{\sup_{b\in B} d(b, A), \sup_{a \in A} d(a, B)\right\}
\end{equation*}
et la distance de Gromov-Hausdorff:
\begin{equation*}
	d_{GH}(\X, \Y) = \inf_{\Z, \gamma_{1}, \gamma_{2}}(\X, \Y)
\end{equation*}
l'infimum étant pris pour $\Z$ un espace métrique, et $\gamma_{1}, \gamma_{2}$ des immersions isométriques de $\X, \Y$ dans $\Z$.


\begin{thm}
	Si $\mathbb{X}$ et $\mathbb{Y}$ sont des espaces métriques pré-compacts:
	\begin{equation*}
		d_{\infty}(\rips(\X), \rips(\Y)) \leq d_{GH}(\X, \Y)
	\end{equation*}
\end{thm}

Ceci est notamment utile lorsqu'on considère la classification de formes non-rigides, puisqu'alors celles
ci sont presque isométriques, mais que calculer leur distance de Gromov-Hausdorff est très coûteux.
On va désormais essayer de démontrer ces résultats de stabilité:
\begin{definition}
	Un \define{module de persistence} $\V$ est une famille d'espaces vectoriels $\left(V_{a}\right)_{a\in \R}$ et une famille $v_{a}^{b}: V_{a} \to V_{b}, a \leq b$ qui se compose bien et de sorte que $v_{a}^{a}$ soit l'identité.
\end{definition}
\begin{itemize}
	\item Si $\S$ est un complexe simplicial filtré, les familles $V_{a} = H(\S_{a})$ et $v_{a}^{b}: H(\S_{a}) \to H(\S_{b})$ l'application linéaire induites par l'inclusion $\S_{a} \hookrightarrow \S_{b}$ forment un module de persistence.
	\item Étant donné un espace métrique $\X$, $H(\rips(\X))$ est un module de persistence.
	\item La filtration par les sous-niveaux de $f$ induit un module de persistence au niveau de l'homologie.
\end{itemize}
\begin{remarque}
	Il faut voir un module de persistence comme un foncteur de la (petite) catégorie associée au poset d'indexation, vers la catégorie des modules sur un anneau $A$.
	Ici, c'est donc un foncteur de $\R$ vers $\mathrm{Vect}_{\F_{2}}$, puisqu'on considère notre homologie dans $\F_{2}$.
\end{remarque}

\begin{definition}
	On dit qu'un module de persistence est dit \define{$q$-docile} si pour tout $a < b$, $v_{a}^{b}$ est de rang fini.
\end{definition}
Si $\X$ est pré-compact métrique, alors $H(\rips(\X))$ et $H(\cech(\X))$ sont $q$-dociles.

Cette condition apporte de forts théorèmes:
\begin{thm}\label{thm:docile-defini}
	Les modules de persistence $q$-dociles ont des diagrammes de persistence bien définis.
\end{thm}
Il faut ici entendre la notion de diagrammes de persistence comme définis par les bases des espaces.

\begin{definition}
	Un \define{homomorphisme de degré $\epsilon$} entre deux modules de persistence est une collection $\Phi$ d'application linéaire vérifiant:
	\begin{category}
		U_{a}\ar[r, "u_{a}^{b}"]\ar[dr, "\phi_{a}"']& U_{b}\ar[dr, "\phi_{b}"]&\\
		& V_{a + \epsilon}\ar[r, "v_{a + \epsilon}^{b+\epsilon}"'] & V_{b + \epsilon}
	\end{category}
	Un \define{$\epsilon$-intercalaire} entre $\mathbb{U}$ et $\mathbb{V}$ est défini par deux homomorphismes de degré $\epsilon$ $\Phi: \U \to \V$ et $\Psi: \V \to \U$ de vérifiant:
	\begin{category}
		\cdots\ar[r]&  U_{a}\ar[dr, "\phi_{a}"]\ar[rr, "u_{a}^{a + 2\epsilon}"] & & U_{a + 2\epsilon}\ar[dr, "\phi_{a + 2\epsilon}"]\ar[r] & \cdots\\
		\cdots\ar[ur]\ar[rr] & & V_{a + \epsilon}\ar[ur, "\psi_{a + \epsilon}"]\ar[rr, "v_{a + \epsilon}^{a + 3\epsilon}"'] & & V_{a + 3\epsilon}
	\end{category}
\end{definition}

\begin{thm}\label{thm:docile-distance}
	Si $\U$ et $\V$ sont $q$-dociles et $\epsilon$-intercalés pour un certain $\epsilon \geq 0$, alors:
	\begin{equation*}
		d_{\infty}(\diag(\U), \diag(\V)) \leq \epsilon
	\end{equation*}
\end{thm}

On va donc chercher à construire des filtrations qui induisent par leurs groupes d'homologie des modules de persistence $q$-dociles,
et qui sont $\epsilon$-intercalés quand les espaces/fonctions considérées sont $O(\epsilon)$-proches.
Plus particulièrement, on va démontrer la docilité des complexes de Rips et de \v{C}ech.


\subsubsection{Théorèmes de Stabilité}
\begin{definition}
	Une \define{application multivaluée} $C$ de $\X$ dans $\Y$ est une partie de $\X \times \Y$ qui se projette surjectivement sur $\X$ par la projection $\pi_{\X}$ définissant le produit.
	\define{L'image} $C(\sigma)$ de $\sigma \subseteq \X$ est la projection canonique sur $\Y$ de la préimage de $\sigma$ par $\pi_{\X}$.
	La \define{transposée} $\transpose{C}$ de $C$ est l'image de $C$ par la symétrie $\X \times \Y \to \Y \times \X$.
	Si $\transpose{C}$ est aussi une application multivaluée, on dit que $C$ est une \define{correspondance.}
\end{definition}

\begin{definition}
	Si $(\X, \rho_{\X}$ et $(\Y, \rho_{\Y})$ sont des espaces métriques compacts, une correspondance $C$ de $\X$ dans $\Y$ est une \define{$\epsilon$-correspondance} si:
	\begin{equation*}
		\forall (x, y), (x', y') \in C, \abs{\rho_{X}(x, x') - \rho_{\Y}(y, y')} \leq \epsilon
	\end{equation*}
\end{definition}

\begin{proposition}
	Avec les hypothèses de la définition ci-dessus:
	\begin{equation*}
		d_{GH}(\X, \Y) = \frac{1}{2}\inf \left\{\epsilon \geq 0\suchthat \exists C, C \text{ est une $\epsilon$-correspondance}\right\}
	\end{equation*}
\end{proposition}

\begin{definition}
	Si $\S$, $\T$ sont des complexes simpliciaux filtrés avec des ensembles de sommets $\X$ et $\Y$ respectivement, une application multivaluée $C$ de $\X$ dans $\Y$ est dite \define{$\epsilon$-simpliciale}
	de $\S$ dans $\T$ si pour tout $a \in \R$ et tout simplexe $\sigma \in \S_{a}$, chaque partie finie de $C(\sigma)$ est un simplexe de $\T_{a + \epsilon}$.
\end{definition}

\begin{proposition}\label{prop:intercalaire-correspondance}
	Si $C$ est une correspondance telle que $C$ et $\transpose{C}$ sont $\epsilon$-simplicales de $\S$ dans $\T$, elle permette de construire un $\epsilon$-intercalaire entre $H(\S)$ et $H(\T)$.
\end{proposition}
\begin{proof}
	Il suffit pour ça de voir qu'une fonction $\epsilon$-simpliciale définit immédiatement un homomorphisme de degré $\epsilon$ sur les modules de persistence définis par les filtrations considérées, par continuation linéaire.
\end{proof}

\begin{proposition}
	Si $\X, \Y$ sont des espaces métriques, pour tout $\epsilon > 2\d_{GH}\left(\X, \Y\right)$, alors $H\left(\rips\left(\X\right)\right)$ et $H\left(\rips\left(\Y\right)\right)$ sont $\epsilon$-intercalés.
\end{proposition}
\begin{proof}
	Si $C$ est une correspondance de $\X$ dans $\Y$ avec une distortion d'au plus $\epsilon$.
	Si $\sigma \in \rips\left(\X, a\right)$ alors $\rho_{\X}\left(x, x'\right) \leq a$ pour tout $x, x'\in \sigma$.
	Soit $\tau \subseteq C\left(\sigma\right)$ fini. Pour tout $y, y' \in \tau$, il existe $x, x'\in \sigma$ tels que:
	$y \in C\left(x\right)$ et $y'\in C\left(x'\right)$ donc:
	\begin{equation*}
		\rho_{\Y}\left(y, y'\right) \leq \rho_{\X}\left(x, x'\right) + \epsilon \leq a + \epsilon
	\end{equation*}
	Ainsi, $\tau \in \rips\left(\Y, a + \epsilon\right)$.
	De même $\transpose{C}$ est $\epsilon$-simpliciale de $\rips\left(\Y\right)$ dans $\rips\left(\Y\right)$.
	On conclut par la Proposition \ref{prop:intercalaire-correspondance}.
\end{proof}

\begin{proposition}
	Si $\X, \Y$ sont des espaces métriques, pour tout $\epsilon \geq 2\d_{GH}\left(\X, \Y\right)$, alors $H\left(\cech\left(\X\right)\right)$ et $H\left(\cech\left(\Y\right)\right)$ sont $\epsilon$-intercalés.
\end{proposition}
La preuve est similaire à celle d'avant.

\begin{thm}
	Soit $\X$ un espace métrique compact. Les modules de persistence associés à l'homologie des complexes $\rips(\X)$ et $\cech(\X)$ sont $q$-dociles.
\end{thm}
\begin{proof}
	On veut montrer que $I_{a}^{b}: H\left(\rips\left(\X, a\right)\right) \to H\left(\rips\left(\X, b\right)\right)$ sont de rang finis quand $a < b$.
	On pose $\epsilon = (b - a) / 2$ et $F \subseteq \X$ un ensemble fini tel que $d_{H}(\X, F) \leq \epsilon / 2$.
	Alors:
	\begin{equation*}
		C = \left\{\left(x, f\right) \in X \times F \suchthat d\left(x, f\right) \leq \frac{\epsilon}{2}\right\}
	\end{equation*}
	définit une $\epsilon$-correspondance.
	Utilisant l'application d'intercalage de $X$ et $F$, $I_{a}^{b}$ se factorise en:
	\begin{equation*}
		H(\rips(X, a))\longrightarrow \underset{\textcolor{vulm}{\text{de dimension finie}}}{H(\rips(F, a + \epsilon))}\rightarrow H(\rips(X, a + 2\epsilon)) = H(\rips(X, b))
	\end{equation*}
\end{proof}

\begin{thm}
	Si $\X$, $\Y$ sont des espaces métriques compacts, alors:
	\begin{align*}
		\dinf\left(\diag\left(H\left(\cech\left(\X\right)\right)\right), \diag\left(H\left(\cech\left(Y\right)\right)\right)\right) \leq & 2d_{GH}\left(\X, \Y\right) \\
		\dinf\left(\diag\left(H\left(\rips\left(\X\right)\right)\right), \diag\left(H\left(\rips\left(Y\right)\right)\right)\right) \leq & 2d_{GH}\left(\X, \Y\right) \\
	\end{align*}
\end{thm}
La preuve des deux derniers théorèmes n'utilise pas l'inégalité triangulaire on pourrait donc étendre les résultats précédents à des espaces munies d'une similarité.

\medskip

Cependant, on a des problèmes avec la dimension de nos espaces: pour tout $0 < \alpha \leq \beta \in \R$, il existe un
espace métrique compact $X$ (immersible dans $\R^{4}$) tel que pour tout $a \in [\alpha, \beta]$, $H_{k}(\rips(X, a))$ est de dimension indénombrable.
Toutefois:
\begin{itemize}
	\item Si $X$ est compact, $\dim H_{1}(\cech(X, a)) < + \infty$
	\item Si $X$ est géodésique, $\dim H_{1}(\rips(X, a)) < + \infty$ pour $a > 0$ et $\diag(H_{1}(\rips(X)))$ est contenu dans la ligne $x = 0$
	\item Si $X$ est un espace géodésique $\delta$-hyperbolique, $\diag(H_{2}(\rips(X)))$ est contenu dans une bande verticale de largeur $\O(\delta)$.
\end{itemize}

\subsection{Calculabilité et Bruit}
Le complexe de Vietoris-Rips et ses filtrations se calculent en $\O\left(\abs{\X}^{d}\right)$, ce qui rend le calcul de persistence quasi impossible en pratique.
Par ailleurs, les filtrations et la distance de Gromov-Hausdorff sont très sensibles au bruit et aux anomalies.

\subsubsection{Calcul statistique}
On va s'intéresser à un espace métrique $(\mathbb{M}, \rho)$ et à une mesure de probabilité $\mu$ à support compact $X_{\mu}$ dans $\mathbb{M}$.
On échantillonne $m$ points selon $\mu$, ce qui nous donne un nuage de point $\hatx{m}$, et une filtration
$\filt{\hatx{m}}$.
On a alors:
\begin{proposition}
	Si $\epsilon > 0$:
	\begin{equation*}
		\P\left(\dinf\left(\diag\left(\filt\left(\X_{\mu}\right)\right), \diag\left(\filt\left(\hatx{m}\right)\right)\right) > \epsilon\right) \leq \P\left(d_{GH}\left(\X_{\mu}, \hatx{m}\right) > \frac{\epsilon}{2}\right)
	\end{equation*}
\end{proposition}
\begin{proof}
	Conséquence directe du Théorème \ref{thm:docile-distance} de stabilité.
\end{proof}

On obtient quasi immédiatement des inégalités de déviation:
\begin{definition}
	Pour $a, b > 0$, on dit que $\mu$ vérifie \define{la supposition $(a, b)$-standard} si pour $x \in \X_{\mu}$ et $r > 0$, on a:
	\begin{equation*}
		\mu(B(x, r)) \geq \min(ar^{b}, 1)
	\end{equation*}
	On note $\mP(a, b, \mathbb{M})$ \define{l'ensemble des distributions de probabilité $(a, b)$-standard} sur $\M$.
\end{definition}
\begin{thm}
	Si $\mu$ vérifie la supposition $(a, b)$-standard, pour tout $\epsilon > 0$:
	\begin{equation*}
		\P\left(\dinf\left(\diag\left(\filt\left(\X_{\mu}\right)\right), \diag\left(\filt\left(\hatx{m}\right)\right)\right) > \epsilon\right) \leq \min \left(\frac{8^{b}}{a\epsilon^{b}}\exp{\left(-ma\epsilon^{b}\right)}, 1\right)
	\end{equation*}
	De plus:
	\begin{equation*}
		\P\left(\dinf\left(\diag\left(\filt\left(\X_{\mu}\right)\right), \diag\left(\filt\left(\hatx{m}\right)\right)\right) \geq C_{1} \left(\frac{\log{m}}{m}\right)^{1 / b}\right) \xrightarrow[m \to \infty]{} 1
	\end{equation*}
	où $C_{1}$ est une constante qui ne dépend que de $a$ et $b$.
\end{thm}
\begin{proof}
	On commence par majorer $\P\left(\d_{GH}\left(\X_{\mu}, \hatx{m}\right) > \frac{\epsilon}{2}\right)$ puis,
	on obtient par la supposition $(a, b)$-standard une borne supérieure explicite pour la couverture de $\X_{\mu}$ par des boules de rayon $\epsilon / 2$.
	On peut alors conclure en prenant l'union des bornes.
\end{proof}

\begin{thm}
	On a:
	\begin{equation*}
		\sup_{\mu \in \mP(a, b, \mathbb{M})} \E[\dinf(\diag(\filt(\X_{\mu})), \diag(\filt(\hatx{m})))] \leq C(\frac{\ln m}{m})^{1 / b}
	\end{equation*}
	où $C$ ne dépend que de $a$ et $b$.
	Si de plus il y a un point non isolé $x$ dans $\mathbb{M}$, et si $x_{m} \in \mathbb{M} \setminus \{x\}$,
	telle que $\rho(x, x_{m}) \leq (am)^{-1/b}$, pour tout estimateur $\hat{\diag}_{m}$ de $\diag(\filt(\X_{\mu}))$:
	\begin{equation*}
		\liminf_{m \to \infty}\rho(x, x_{m})^{-1}\sup_{\mu \in \mP(a, b, \mathbb{M})}\E[\dinf(\diag(\filt(\X_{\mu})), \hat{\diag}_{m})] \geq C'
	\end{equation*}
	où $C'$ est une constante absolue.
\end{thm}

\subsubsection{Paysages de persistence}
\begin{definition}
	Si on a un diagramme de persistence $(b_{i}, d_{i})$, son \define{paysage de persistence} est obtenu en y ajoutant à chaque point les deux projections orthogonales sur la diagonale par rapport aux axes, puis en plaçant à l'horizontale sa diagonale.
	Formellement, c'est l'union pour $p = (\frac{b + d}{2}, \frac{d - b}{2})$ des graphes:
	\begin{equation*}
		\Lambda_{p}(t) = \begin{cases}
			t - b & t \in [b, \frac{b + d}{2}] \\
			d - t & t \in [\frac{b + d}{2}, d] \\
			0     & \text{sinon}
		\end{cases}
	\end{equation*}
\end{definition}

C'est un encodage de la persistence comme un élément d'un espace fonctionnel, comme fait ci-dessous:
\begin{definition}
	On définit le \define{$k$-ème paysage} d'un diagramme $D$ par:
	\begin{equation*}
		\lambda_{D}(k, t) = \underset{p \in D}{\mathrm{kmax}} \Lambda_{p}(t), t \in \R, k \in \N
	\end{equation*}
	où $\mathrm{kmax}$ désigne le $k$-ème plus grand élément d'un ensemble.
\end{definition}

\begin{proposition}
	\begin{itemize}
		\item Pour $t \in \R$ et $k \in \N$, $0 \leq \lambda_{D}(k, t) \leq \lambda_{D}(k + 1, t)$
		\item Pour $t \in \R$, et $k \in \N$, $\abs{\lambda_{D}(k, t) - \lambda_{D'}(k, t)} \leq \dinf(D, D')$
	\end{itemize}
\end{proposition}

Dans la suite, on note $\mL_{T}$ les paysages dont le support est dans $[0, T]$, on prend $P$ une distribution de probabilité
sur $\mL_{T}$ et $\lambda_{1}, \ldots, \lambda_{n} \sim P$ i.i.d.
On note $\mu(t) = \E[\lambda_{i}(t)]$ le paysage moyen et on l'estime par la moyenne échantillonnée:
\begin{equation*}
	\bar{\lambda}_{n}(t) = \frac{1}{n}\sum \lambda_{i}(t)
\end{equation*}
$\bar{\lambda}_{n}$ est un estimateur point à point non biaisé de $\mu$, qui converge point à point.

\begin{definition}
	Soit $\mF$ la famille des applications d'évaluation $f_{t} : \mL_{T} \to \R$.
	Le \define{processus empirique} indexé par les $f_{t}$ est défini par:
	\begin{equation*}
		\mathbb{G}_{n}(t) = \sqrt{n}(\bar{\lambda}_{n}(t) - \mu_{t}) = \sqrt{n}(P_{n} - P)(f_{t})
	\end{equation*}
\end{definition}

\begin{thm}
	Soit $\mathbb{G}$ un pont Brownien avec fonction de covariance:
	\begin{equation*}
		\kappa(s, t) = \int f_{t}(\lambda)f_{s}(\lambda)\d P(\lambda) - \int f_{t}(\lambda)\d P(\lambda)\int f_{s}(\lambda)\d P(\lambda).
	\end{equation*}
	On a alors:
	\begin{equation*}
		\mathbb{G}_{n} \to \mathbb{G}
	\end{equation*}
	pour la convergence faible.
\end{thm}

Si de plus on note $\sigma(t)$ l'écart-type de $\sqrt{n}\bar{\lambda}_{n}(t)$:
\begin{thm}
	Si $\sigma(t) > c > 0$ sur un intervalle $I = [t_{*}, t^{*}] \subseteq [0, T]$ pour une constance $c$, avec
	$W = \sup_{t \in I} \abs{\mathbb{G}(f_{t})}$ on a:
	\begin{equation*}
		\sup_{z \in \R}\abs{\P\left(\sup_{t \in [t_{*}, t^{*}]}\abs{\mathbb{G}_{n}\left(t\right)} \leq z\right) - \P\left(W \leq z\right)} = \O\left(\frac{\left(\log n\right)^{7 / 8}}{n^{1 / 8}}\right)
	\end{equation*}
\end{thm}

\subsubsection{Bruit et échantillonnage}
