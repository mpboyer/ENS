\section{Fonctions de Morse}

\subsection{Fonction Lisses}
Dans la suite on considère des fonctions réelles sur une $d$-variété différentielle $\M$ munie d'un atlas.

\begin{definition}
	Un \define{point critique} $p$ de $f: \M \to \R$ est un point de $\M$ tel que $\nabla f(p) = 0$.
	Il est dit \define{dégénéré} si $\abs{H(p)} = 0$.
\end{definition}

\begin{lemme}[Lemme de Morse]
	Si on se donne un point critique non-dégénéré $p^{*}$ pour $f: \M \to \R$, il existe une carte $\phi$ de
	$\M$ telle que $\phi(p^{*}) = 0$ et tel que sur l'ouvert associé à $\phi$, $f$ s'écrit
	\begin{equation*}
		f(p) = f(p^{*}) - \sum_{i = 1}^{q}x_{i}^{2} + \sum_{i = q + 1}^{d}x_{i}^{2},
	\end{equation*}
	dans son développement de Taylor autour de $p^{*}$.
\end{lemme}

\begin{definition}
	\define{L'indice} d'un point critique non-dégénéré $p^{*}$ est le nombre $q$ de coefficients négatifs
	du développement du Lemme de Morse.
	C'est le nombre de valeurs propres négatives de $H(p^{*})$.
\end{definition}

\begin{definition}
	Une fonction $f: \M \to \R$ est une \define{fonction de Morse} si elle n'a pas de point critique dégénéré
	et des valeurs critiques distinctes.
\end{definition}

\begin{proposition}
	\begin{itemize}
		\item Les fonctions de Morse forment un sous-ensemble dense des fonctions lisses.
		\item Les points critiques d'une fonction de Morse sont isolés (et $f$ est localement polynomiale).
	\end{itemize}
\end{proposition}

Dans la suite, on prend $f$ une fonction de Morse et on note $\M^{i} = f^{-1}(\left]-\infty, i\right])$
l'ensemble de sous-niveau de $f$ associé à $i$.

\begin{proposition}
	\begin{itemize}
		\item Si on se donne un intervalle $[i, j]$ qui ne contient pas de valeur critique de $f$,
		      $\M^{i}$ et $\M^{j}$ sont difféomorphes.
		\item Autour d'une valeur critique $f(p)$ d'indice $q$, $\M^{f(p) - \epsilon}$ auquel on recolle
		      $\mathbb{D}^{q}$ a même homotopie que $\M^{f(p) + \epsilon}$.
	\end{itemize}
\end{proposition}

\begin{thm}[Relation de Morse-Euler]
	On a toujours
	\begin{equation*}
		\chi(\M) = \sum_{i = 0}^{d} (-1)^{i}\mu_{i}(f),
	\end{equation*}
	ou $\mu_{i}(f)$ est le nombre de points critiques d'indice $i$ de $f$.
\end{thm}

\subsection{Représentation des fonctions de Morse}
\begin{definition}
	Soit $p \in \R^{n}$ et $\sigma$ un $d$-simplexe.
	Il existe des réels $\alpha_{0}, \ldots, \alpha_{d}$ tels que
	$$
		p = \sum_{i = 0}^{d}\alpha_{i}\tau_{0}^{i}, \quad \sum_{i = 0}^{d}\alpha_{i} = 1,
	$$
	où $\tau_{0}^{i}$ est la $i$-ème $0$-face de $\sigma$.
	Ces coefficients sont appelés \define{coordonnées barycentriques} de $p$ relativement à $\sigma$.
\end{definition}

\begin{definition}
	Si $\hat{f}$ est une fonction réelle sur les $0$-simplexes d'une triangulation $\mT$, on définit
	$f: \mT \to \R$ par interpolation linéaire de $\hat{f}$ par rapport aux coordonnées barycentriques, i.e.
	si $p \in \sigma \in \mT$ est un point d'un $d$-simplexe,
	\begin{equation*}
		f(p) = \sum_{i = 0}^{d}\alpha_{i}\hat{f}(\tau_{0}^{i}).
	\end{equation*}
	$f$ est un \define{champ scalaire linéaire par morceaux} (PL scalar field).
\end{definition}

\begin{definition}
	Le \define{lien inférieur} $\Lk^{-}(\sigma)$ (respectivement supérieur $\Lk^{+}$) d'un $d$-simplexe
	$\sigma$ relativement à un champ scalaire PL $f$ est la partie du lien $\Lk(\sigma)$ telle que chacune
	de ses $0$-faces a une image par $f$ strictement inférieure (respectivement supérieure) à celle de
	$\sigma$.
\end{definition}

\begin{definition}
	Pour $f: \M \to \R$ un champ scalaire PL sur une variété PL $\M$, un sommet $v$ de $\M$ est un
	\define{point régulier} si et seulement si $\Lk^{-}(v)$ et $\Lk^{+}(v)$ sont simplement connexes.
	Sinon, $v$ est un \define{point critique} de $f$ et $f(v)$ est une \define{isovaleur critique}.
\end{definition}

\begin{definition}
	Un champ scalaire PL $f$ est une \define{fonction de Morse PL} si toutes ses isovaleurs critiques sont
	distinctes et	si elle n'a pas de point critique dégénéré.
\end{definition}

La relation de Morse-Euler nous dit que les points critiques se simplifient en paires.
On va pour cela utiliser l'ordre associé à la persistence croissante pour simplifier les paires.

\begin{definition}
	On construit des sous-complexes simpliciaux $\M_{i}$ de $\M$ une variété PL par l'union des $i$ premiers
	simplexes pour l'ordre lexicographique.
	Ceci nous donne une suite de sous-complexes croissante pour l'inclusion: c'est une filtration appelée la
	\define{filtration lexicographique}.
\end{definition}

Calculer la suite des groupes d'homologies $\mathcal{H}_{p}(\M_{i})$ nous permet de calculer le diagramme de
persistence associé à $\M$.

\begin{definition}
	Si $(\tau, \sigma_{j})$ est une paire de persistence (un élément d'un diagramme de persistence), on
	définit sa \define{persistence} comme $\tilde{f}(\sigma_{j}) - \tilde{f}(\tau)$ où pour tout simplexe
	$\sigma$, sa \define{valeur scalaire} $\tilde{f}(\sigma)$ est le maximum des valeurs de $f$ sur les
	sommets ($0$-faces) de $\sigma$.
\end{definition}

Ceci nous donne l'algorithme \ref{alg:pairCells} pour calculer la persistence de chaque composante
topologique de la variété PL $\M$.
\begin{algorithm}
	\caption{Calcul des Paires de Persistence pour Morse}
	\label{alg:pairCells}
	\begin{algorithmic}
		\Input{Filtration lexicographique de $\M$ par $f$}
		\EndInput
		\Output{Diagrammes de persistence $\diag_k(f)$.}
		\EndOutput

		\For{$j \in [1, n]$}
		\State{// Traiter le $(d_i+1)$-simplexe $\sigma_j$.}
		\State{$Pair(\sigma_j) \leftarrow \emptyset$}
		\State{$Chain(\sigma_j) \leftarrow \sigma_j$}
		\State{// Propagation homologue de $\partial \sigma_j$}
		%   Search for a $(d_i-1)$-cycle in $\domain_i$ homologous
		% to $\partial \sigma_i$
		\While{$\partial\left(Chain(\sigma_j)\right) \neq 0$}
		\State{$\tau \leftarrow \max\left(\partial\left(Chain(\sigma_j)\right)\right)$}
		\If{$Pair(\tau) == \emptyset$}
		\State{// $\tau$ crée un $(d_i)$-cycle}
		\State{\textbf{break}}
		\Else
		\State{// Étendre la chaîne (avec bord homologue)}
		\State{$Chain(\sigma_j) \leftarrow Chain(\sigma_j) + Chain\left(Pair(\tau)\right)$}
		\EndIf
		\EndWhile
		\If{$\partial \left(Chain(\sigma_j)\right) \neq 0$}
		\State{// Un cycle non-trivial homologue à $\partial \sigma_j$ existe (l.10)}
		\State{$\tau \leftarrow \max\left(\partial\left(Chain(\sigma_j)\right)\right)$}
		\State{$Pair(\sigma_j) \leftarrow \tau$}
		\State{$Pair(\tau) \leftarrow \sigma_j$}
		\State{$\diag_{d_{i}}(f) \leftarrow
				\diag_{d_{i}}(f) \cup (\tau, \sigma_j)$}
		\EndIf
		\EndFor
	\end{algorithmic}
\end{algorithm}

Cet algorithme suit la même idée que \ref{alg:betti_nop}.

\subsection{Comparaison Topologique}
On a désormais une méthode pour calculer des diagrammes de persistence à partir uniquement d'une filtration
et d'une fonction de Morse PL sur cette filtration.

\subsubsection{Simplifcation et Empreinte Topologique}
L'idée va donc être d'avoir une mesure permettant de simplifier la topologie de la variété PL en ne
maintenant que les propriétés suffisamment persistentes:
\begin{category}
	\mD(f)\ar[r] & \mD(g) \subseteq \mD(f)\ar[d, "?"]\\
	f & g
\end{category}
La question est donc de savoir comment calculer la fonction $g$ a partir du diagramme objectif $\mD(g)$.

Toutefois, bien qu'en $2$ dimensions le problème soit résolu,
la simplification des ensembles de sous-niveau de paramètres $(t, d)$ (calcul d'une fonction à distance au
plus $d$ de la fonction originelle qui minimise les nombres de Betti d'isovaleur $t$) est NP-difficile
en $3$ dimensions.
Pour cela on va plutôt chercher à optimiser numériquement la fonction $g$, en se tournant vers l'analyse
géométrique de l'espace des diagrammes de persistence.

\begin{definition}
	On définit le diagramme $\mD^{*}$ \define{barycentre} de $N$ diagrammes d'entrée $\diag(f_{i})$ comme
	\begin{equation*}
		\mD^{*} = \argmin_{\mD} \sum_{i = 1}^{N} W_{2}^{2}(\mD, \diag(f_{i})).
	\end{equation*}
\end{definition}

On rappelle que le Théorème \ref{thm:persistence-stability} montre que si $f_{i}$ et $f_{j}$ sont proches,
leurs diagrammes le seront.

\medskip

En voyant les diagrammes de persistence comme des empreintes topologiques des variétés.
En particulier, on peut comparer les données, notamment via la distance de Wasserstein qui se calcule en
résolvant un simple problème d'assignation, et ceci permet de définir des géodésiques et des moyennes sur
entre les variétés, ce qui permet de construire de meilleurs groupement avec l'algorithme des $k$-moyennes
par exemple.

\subsubsection{Graphes de Reeb}
Les diagrammes de persistence sont malheureusement assez limités, notamment puisque $f$ et $ \lambda f$
auront toujours le même diagramme associé, ou par exemple si on échange les centres de gaussiennes dans une
mixture.

\begin{definition}
	Si $f: \M \to \R$ est un champ scalaire de Morse PL défini sur une variété compacte PL $\M$, on note
	$f^{-1}(f(p))_{p}$ le \define{contour} de $f$ contenant $p \in \M$.
	Le \define{graphe de Reeb} $\mR(f)$ est le complexe simplicial unidimensionnel défini comme le quotient
	de $\M\times \R$ par $(p_{1}, f(p_{1})) \sim (p_{2}, f(p_{2})) \Leftrightarrow p_{2}\in (f^{-1}(f(p_{1})))_{p} \land f(p_{1}) = f(p_{2})$.
\end{definition}

Le graphe de Reeb est calculable en temps $\O(m\log m)$ où $m$ est le nombre de $2$-simplexes de $\M$.

\begin{proposition}
	Si on note $\phi: \M\to \mR(f)$ la projection de la variété sur le graphe de Reeb associé à $f$
	et $\psi: \mR(f) \to \R$ la projection sur l'isovaleur associée au point, on a $f = \psi \circ \phi$.
\end{proposition}

\begin{definition}
	La valence d'un $0$-simplexe $v \in \mR(f)$ est le nombre de $1$-simplexes dans son étoile $\St(v)$.
\end{definition}

\begin{proposition}
	Soit $f = \psi \circ \phi$ un champ scalaire de Morse PL sur une $d$-variété PL et soit $\mR(f)$ son
	graphe de Reeb.
	\begin{itemize}
		\item On caractérise les images par $\phi$ des points réguliers de $f$ comme l'intérieur de
		      $1$-simplexes de $\mR(f)$.
		\item On caractérise les images des points critiques d'indice $0$ ou $d$ de $f$ (donc ses extrema)
		      par les $0$-simplexes de $\mR(f)$ de valence $1$.
		\item Si $d = 2$, on caractérise les images des points critiques d'indice $1$ de $f$
		      (donc ses selles) comme les $0$-simplexes de $\mR(f)$ de valence $2, 3, 4$.
		\item Si $d\geq 3$, les points critiques d'indice $1$ ou $d-1$ de $f$ ont pour image par $\phi$ les
		      $0$-simplexes de valence $2$ ou $3$. La réciproque n'est pas nécessairement vraie.
		\item Si $d > 3$, les points critiques d'indice différent de $0, 1, (d- 1)$ ou $d$ ont pour image
		      par $\phi$ les $0$-simplexes de valence $2$.
	\end{itemize}
\end{proposition}

En terme d'homologie, il reste à comparer l'homologie et les graphes de Reeb.

\begin{thm}
	On a $\beta_{0}(\mR(f)) = \beta_{0}(\M)$ et $\beta_{1}(\mR(f)) \leq \beta_{1}(\M)$.
\end{thm}

On peut noter que dans le cas où $\M$ est une surface ($2$-variété), le graphe de Reeb est planaire.

\begin{thm}
	On considère $\mR(f)$ le graphe de Reeb de $f$ un champ scalaire PL sur une $2$-variété $\M$.
	Soit $b(\M)$ le nombre de composantes connexes de $\M$ et $g(\M)$ son genre.
	Le nombre de boucles de $\mR(f)$ est décrit par:
	\begin{itemize}
		\item Si $\M$ est orientable:
		      \begin{itemize}
			      \item Si $b(\M) = 0$ alors $\beta_{1}(\mR(f)) = g(\M)$
			      \item Sinon $g(\M) \leq \beta_{1}(\mR(f)) \leq 2g(\M) + b(\M) - 1$
		      \end{itemize}
		\item Sinon:
		      \begin{itemize}
			      \item Si $b(\M) = 0$ alors $0 \leq \beta_{1}(\mR(f)) \leq \frac{1}{2}g(\M)$
			      \item Sinon $0 \leq \beta_{1}(\mR(f)) \leq g(\M) + b(\M) - 1$
		      \end{itemize}
	\end{itemize}
\end{thm}

\begin{definition}
	Étant donnée $f: \M \to \R$, on note $x \sim_{M} y$ si $f(x) = f(y) = \alpha$ et $x, y$ sont dans la même
	composante connexe de $f^{-1}(\left]-\infty, \alpha\right])$.
	L'\define{arbre de fusion} est l'espace $\M$ quotienté par $\sim_{M}$.
\end{definition}

L'arbre de fusion est une relaxation du graphe de Reeb et permet de supprimer les boucles du graphe de Reeb
en supprimant la possibilité de défusionner.
Il permet de plus de récupérer un historique des fusions de composantes.

\smallskip

Le graphe de Reeb permet de faire de la segmentation par arcs en considérant les séparations et les fusions
comme des changements de segments.
Toutefois, ils perdent beaucoup d'information, on introduit donc le cartographe:
\begin{definition}
	Si $f : \M \to \R$ tire en arrière tous les intervalles en un nombre fini de composantes connexes par arc,
	on définit le cartographe $M(I, f)$ (ou mapper) de l'intervalle $I$ en $f$ comme le nerf de la famille
	des composantes connexes de $f^{-1}(I)$.
\end{definition}

Dans le cas d'un nuage de points, on remplace les composantes connexes par les clusters sous $f^{-1}$, et on
obtient un algorithme qui permet de calculer localement une alternative plus générale au graphe de Reeb.
