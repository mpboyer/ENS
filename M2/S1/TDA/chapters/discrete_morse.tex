\section{Théorie de Morse Discrète}
Puisque le graphe de Reeb (et les diagrammes de persistence) est inutilisable pour comparer des objets qui
qui sont par exemple des nappes de mixtures gaussiennes avec des points différents, il faut trouver une
manière de décrire des exemples trop similaires.
On peut aisément construire des exemples de jeux de données qui ont la même description topologique par le
graphe de Reeb mais qui sont applicativement différents.

\subsection{Complexe de Morse}
Le complexe de Morse est défini à l'aide des variétés descendantes de la surface d'une fonction de Morse.
\begin{definition}
	Si $M$ est une variété PL, une \define{courbe intégrale} sur $M$ est une courbe sur $M$ dont les vecteurs
	tangents sont colinéaires au gradient sur $M$.
\end{definition}
Les courbes intégrales vont donc toujours "monter"

\begin{definition}
	La \define{variété descendante} d'un point critique $p$ de $M$ est l'ensemble des points des courbes
	intégrales qui terminent en $p$.
\end{definition}

\begin{definition}
	Le \define{complexe de Morse} de $f$ est le complexe cellulaire formé par les variétés descendantes de la
	variété surface de $f$.
\end{definition}

\begin{remarque}
	C'est une généralisation du diagramme de Voronoï avec des fonctions autres que des fonctions de distance.
\end{remarque}

On va aussi s'intéresser à l'opposé des fonctions que l'on considère. Dans ce cas, au lieu d'avoir des
courbes intégrales qui montent, on a des courbes qui descendent, et donc, on calcule des \define{variétés
	ascendantes} sur $M$.

\begin{definition}
	Le \define{complexe de Morse-Smale} de $M$ est l'intersection des complexes de Morse de $f$ et de $-f$.
\end{definition}

Ceci revient à calculer les extrémités d'intégration (en avant, et en arrière, au sens d'Euler) de sorte que
tous les points qui ont les mêmes extrémités d'intégration appartiennent à la même cellule du complexe de
Morse-Smale.

Toutefois, il n'est pas possible de définir ce complexe pour toute fonction $f$. En effet, mais s'il semble
clair qu'il faille avoir une fonction de Morse à l'origine, avec quelques conditions en plus.
En particulier, il faut que les fonctions soient dérivables, aient tous leurs points critiques distincts,
et que les variétés ascendantes et descendantes s'intersectent transversalement.

\begin{definition}
	L'intersection de deux variétés est \define{transverse} si l'intersection des plans tangents génère le plan
	tangent de l'intersection.
	Mathématiquement, cela revient à dire que si on note $I = A \cap B$, on doit avoir
	\begin{equation*}
		\codim A + \codim B = \codim I.
	\end{equation*}
\end{definition}

Si on prend par exemple deux selles de cheval tournées l'une par rapport à l'autre de manière orthogonale,
leur intersection n'est pas transverse.

\begin{proposition}
	Le complexe de Morse-Smale transforme les $1$-cellules d'indice $d$ en $1$-cellule d'indice $d + 1$.\\
	Pour une $2$-variété:
	\begin{itemize}
		\item Les $2$-cellules sont des quadrangles (un mimimum local, un maximum local et deux points selles);
		\item Les selles simples ont valence $4$;
		\item Les extrema ont valence arbitraire.
	\end{itemize}
	Pour une $3$-variété:
	\begin{itemize}
		\item Les $3$-cellules ont un nombre arbitraire de faces;
		\item Les faces polygonales sont des quadrangles.
	\end{itemize}
\end{proposition}

Il est toutefois particulièrement complexe de calculer le complexe de Morse-Smale, notamment à cause de
la difficulté du dépliage de selles et de la difficulté de vérification des intersections transverses.

\subsection{Calcul du complexe de Morse-Smale}
On se donne dans la suite un complexe simplicial $\sigma$. Par convention, $\sigma_{i}$ sera un $i$-simplexe.

\begin{definition}
	Une fonction \define{de Morse discrète} est une fonction qui à chaque simplexe $\sigma_{i}$ associe
	un réel $f(\sigma_{i})$ telle que:
	\begin{itemize}
		\item $\abs{\left\{\sigma_{i - 1} < \sigma_{i} \suchthat f(\sigma_{i - 1}) \geq f(\sigma_{i})\right\}} \leq 1$
		\item $\abs{\left\{\sigma_{i + 1} > \sigma_{i} \suchthat f(\sigma_{i > 1}) \leq f(\sigma_{i})\right\}} \leq 1$
	\end{itemize}
	On demande donc que le nombre de faces de dimension $i- 1$ ayant une valeur supérieure au simplexe de
	dimension $i$  soit au plus $1$, et de même pour les cofaces de dimension $i + 1$.\\
	Un point critique est alors un point dont le nombre de faces supérieures et le nombres de cofaces
	inférieures sont nuls.
\end{definition}

On va donc pouvoir ainsi définir des champs de vecteurs afin de modéliser le gradient de notre fonction.

\begin{definition}
	Un \define{vecteur dicret} est une paire $\{\sigma_i < \sigma_{i + 1}\}$.
	Un \define{champ de vecteur discret} est une collection $V$ de vecteurs discrets tel que chaque simplexe
	soit présent dans au plus un vecteur.
\end{definition}

\begin{definition}
	Une \define{courbe intégrale discrète} ou \define{$v$-chemin} est une suite de vecteurs discrets
	$\left(\left\{\sigma_{i}^{j} < \sigma_{i + 1}^{j}\right\}\right)_{0 \leq j \leq n}$ telle que
	$\sigma_{i}^{j} \neq \sigma_{i}^{j + 1} < \sigma_{i + 1}^{j}$.
\end{definition}

Autrement dit, on part d'un $i$-simplexe, puis on continue d'une autre face d'une coface de ce simplexe de
départ.

\begin{definition}
	Un \define{champ de gradient discret} est un champ de vecteur discret dont tous les $v$-chemins sont sans
	boucles.
\end{definition}

\begin{remarque}
	Ceci revient à dire intuitivement que le rotationnel du champ de vecteur discret est nul, et qu'on décrit
	donc bien un potentiel.
\end{remarque}

Pour pouvoir décrire les variétés ascendantes et descendantes et construire les analogues des complexes de
Morse-Smale discrets, on passe par le complexe dual (au sens de Poincaré) de notre complexe cellulaire de
départ.
Pour cela, on a souvent besoin d'ajouter un demi-sommet à l'infini permettant de fermer le complexe.

Alors, pour représenter le gradient dual, il suffit de retourner tout vecteur discret du gradient discret.
Un $v$-chemin renversé (qui permet de remonter le gradient) est donc simplement un $v$-chemin dans le
complexe dual.

On peut ainsi calculer le champ de gradient discret à partir d'une filtration, comme présenté dans
l'Algorithme \ref{alg:graddmt}.
On peut montrer que l'algorithme est fortement parallélisable.
L'algorithme utilise l'ordre lexicographique associé à la filtration, et fait apparaître des paires
de persistence dites apparentes.

\begin{algorithm}
	\caption{Assignation par plus forte descente}
	\label{alg:graddmt}
	\begin{algorithmic}
		\For{$i \leq d$ une dimension}
		\For{$\sigma_{i}$ un simplexe}
		\State{$C^{-}(\sigma_{i}) = \left\{\sigma_{i + 1} > \sigma_{i} \suchthat \sigma_{i} = \argmax\displaystyle_{\sigma_{i}' < \sigma_{i + 1}}f(\sigma_{i}')\right\}$}\Comment{Cofaces maximisées par $\sigma_{i}$}
		\State{$\sigma_{i + 1}^{*} = \argmin_{C^{-}(\sigma_{i})}f(\sigma_{i + 1})$}\Comment{Coface minimale}
		\EndFor
		\EndFor
	\end{algorithmic}
\end{algorithm}

\begin{proposition}
	Les simplexes non-appairés sont des impasses pour les $v$-chemins, ce qui nous donne une notion de
	simplexe critique.
	Sur le domaine de la variété, on n'a donc pas de point critique dégénéré.
	\smallskip

	Pour l'intersection transverse, la dimension des $v$-chemins dépend de l'indice critique du point.
	Les variétés descendantes se retrouvent dans le complexe primal et les ascendantes dans le complexe dual.
\end{proposition}

\begin{proposition}
	Avec l'Algorithme \ref{alg:graddmt}, à tout point critique d'indice $i$ au sens PL, il existe au moins un
	point critique au sens discret.
\end{proposition}

Le défaut de tels objets est leur forte instabilité, qui se voit notamment lorsqu'on déplace continuement
le plus bas centre de l'opposé d'une mixture gaussienne, et qui donne un complexe qui n'est pas continu,
malgré des structures arbitrairement proches.


