\subsection{Calcul de filtrations}
\subsubsection{Complexes de \v{C}ech, de Vietoris-Rips, et autres}
\begin{definition}
	On considère un recouvrement $\mU$ par des ouverts d'un espace topologique $X$.
	Le \define{complexe de \v{C}ech} $C\left(\mU\right)$ associé au recouvrement $\mU$ vérifie:
	\begin{itemize}
		\item L'ensemble de sommets de $C\left(\mU\right)$ est l'ensemble $\mU$.
		\item $\left[U_{0}, \ldots, U_{k}\right]$ est un $k$-simplexe dans $C\left(\mU\right)$ si et seulement si $\cap U_{j} \neq \emptyset$.
	\end{itemize}
\end{definition}

\begin{thm}[Nerveux (Leray)]
	Si toutes les intersections entre les ouverts de $\mU$ sont soit vides soit contractibles, alors $C(\mU)$ et $X$ sont homotopiquement équivalents.
\end{thm}

Si on se donne plutôt un nuage de point $V$ dans un espace métrique $(X, d)$ et un réel $\alpha$.
\begin{definition}
	Le \define{complexe de \v{C}Čech} $\cech\left(V, \alpha\right)$ est le complexe simplicial
	filtré indexé par $\R$ dont l'ensemblde sommets est $V$ et tel que:
	\begin{equation*}
		\sigma = \left[p_{0}, \ldots, p_{k}\right] \in \cech\left(V, \alpha\right) \Leftrightarrow \bigcap_{i = 0}^{k} B(p_{i}, \alpha) \neq \emptyset
	\end{equation*}
\end{definition}

\begin{definition}
	Le \define{complexe de Vietoris-Rips} $\rips(V)$ est le complexe simplicial filtré indexé par $\R$ dont l'ensemble de sommets est $V$
	et est défini par:
	\begin{equation*}
		\sigma = \left[p_{0}, \ldots, p_{k}\right] \in \cech\left(V, \alpha\right) \Leftrightarrow \forall i, j \in \left\{0, \ldots, k\right\}, d\left(p_{i}, p_{j}\right) \leq \alpha
	\end{equation*}
\end{definition}

\begin{proposition}
	On a, pour tout $\alpha > 0$:
	\begin{equation*}
		\cech\left(L, \frac{\alpha}{2}\right) \subseteq \rips\left(L, \alpha\right) \subseteq \cech\left(L, \alpha\right)
	\end{equation*}
\end{proposition}

\begin{definition}
	Si $V = \{p_{1}, \ldots, p_{n}\} \subseteq \R^{d}$, on définit la \define{cellule de Voronoï} associée à $p_{i}$ par:
	\begin{equation*}
		\mathcal{Vor}(p_{i}) = \left\{x \in \R^{d}\suchthat \forall j, \norm{x - p_{i}} \leq \norm{x - p_{j}}\right\}
	\end{equation*}
	Le \define{complexe de Delaunay} $\mathcal{D}(P)$ est le nerf de la couverture faite par les cellules de Voronoï.
	\define{L'alpha complexe} $\A(P, \alpha)$, pour $\alpha \geq 0$ est le nerf de la famille:
	\begin{equation*}
		(\mathrm{Vor}(p_{i}) \cap B(p_{i}, \sqrt{\alpha}))_{i = 1, \ldots, n}
	\end{equation*}
\end{definition}

\begin{thm}
	$\A(P, \alpha)$ est homotopie équivalent à $\bigcup_{i = 1}^{n} B(p_{i}, \sqrt{\alpha})$.
\end{thm}

\subsubsection{Stabilité}

On va utiliser ci-dessous la distance de Hausdorff:
\begin{equation*}
	d_{H}(A, B) = \max \left\{\sup_{b\in B} d(b, A), \sup_{a \in A} d(a, B)\right\}
\end{equation*}
et la distance de Gromov-Hausdorff:
\begin{equation*}
	d_{GH}(\X, \Y) = \inf_{\Z, \gamma_{1}, \gamma_{2}}(\X, \Y)
\end{equation*}
l'infimum étant pris pour $\Z$ un espace métrique, et $\gamma_{1}, \gamma_{2}$ des immersions isométriques de $\X, \Y$ dans $\Z$.


\begin{thm}\label{thm:docile-distance-function}
	Si $\mathbb{X}$ et $\mathbb{Y}$ sont des espaces métriques pré-compacts:
	\begin{equation*}
		d_{\infty}(\rips(\X), \rips(\Y)) \leq d_{GH}(\X, \Y)
	\end{equation*}
\end{thm}

Ceci est notamment utile lorsqu'on considère la classification de formes non-rigides, puisqu'alors celles
ci sont presque isométriques, mais que calculer leur distance de Gromov-Hausdorff est très coûteux.
On va désormais essayer de démontrer ces résultats de stabilité:
\begin{definition}
	Un \define{module de persistence} $\V$ est une famille d'espaces vectoriels $\left(V_{a}\right)_{a\in \R}$ et une famille $v_{a}^{b}: V_{a} \to V_{b}, a \leq b$ qui se compose bien et de sorte que $v_{a}^{a}$ soit l'identité.
\end{definition}
\begin{itemize}
	\item Si $\S$ est un complexe simplicial filtré, les familles $V_{a} = H(\S_{a})$ et $v_{a}^{b}: H(\S_{a}) \to H(\S_{b})$ l'application linéaire induites par l'inclusion $\S_{a} \hookrightarrow \S_{b}$ forment un module de persistence.
	\item Étant donné un espace métrique $\X$, $H(\rips(\X))$ est un module de persistence.
	\item La filtration par les sous-niveaux de $f$ induit un module de persistence au niveau de l'homologie.
\end{itemize}
\begin{remarque}
	Il faut voir un module de persistence comme un foncteur de la (petite) catégorie associée au poset d'indexation, vers la catégorie des modules sur un anneau $A$.
	Ici, c'est donc un foncteur de $\R$ vers $\mathrm{Vect}_{\F_{2}}$, puisqu'on considère notre homologie dans $\F_{2}$.
\end{remarque}

\begin{definition}
	On dit qu'un module de persistence est dit \define{$q$-docile} si pour tout $a < b$, $v_{a}^{b}$ est de rang fini.
\end{definition}
Si $\X$ est pré-compact métrique, alors $H(\rips(\X))$ et $H(\cech(\X))$ sont $q$-dociles.

Cette condition apporte de forts théorèmes:
\begin{thm}\label{thm:docile-defined-module}
	Les modules de persistence $q$-dociles ont des diagrammes de persistence bien définis.
\end{thm}
Il faut ici entendre la notion de diagrammes de persistence comme définis par les bases des espaces.

\begin{definition}
	Un \define{homomorphisme de degré $\epsilon$} entre deux modules de persistence est une collection $\Phi$ d'application linéaire vérifiant:
	\begin{category}
		U_{a}\ar[r, "u_{a}^{b}"]\ar[dr, "\phi_{a}"']& U_{b}\ar[dr, "\phi_{b}"]&\\
		& V_{a + \epsilon}\ar[r, "v_{a + \epsilon}^{b+\epsilon}"'] & V_{b + \epsilon}
	\end{category}
	Un \define{$\epsilon$-intercalaire} entre $\mathbb{U}$ et $\mathbb{V}$ est défini par deux homomorphismes de degré $\epsilon$ $\Phi: \U \to \V$ et $\Psi: \V \to \U$ de vérifiant:
	\begin{category}
		\cdots\ar[r]&  U_{a}\ar[dr, "\phi_{a}"]\ar[rr, "u_{a}^{a + 2\epsilon}"] & & U_{a + 2\epsilon}\ar[dr, "\phi_{a + 2\epsilon}"]\ar[r] & \cdots\\
		\cdots\ar[ur]\ar[rr] & & V_{a + \epsilon}\ar[ur, "\psi_{a + \epsilon}"]\ar[rr, "v_{a + \epsilon}^{a + 3\epsilon}"'] & & V_{a + 3\epsilon}
	\end{category}
\end{definition}

\begin{thm}\label{thm:docile-distance-module}
	Si $\U$ et $\V$ sont $q$-dociles et $\epsilon$-intercalés pour un certain $\epsilon \geq 0$, alors:
	\begin{equation*}
		d_{\infty}(\diag(\U), \diag(\V)) \leq \epsilon
	\end{equation*}
\end{thm}

On va donc chercher à construire des filtrations qui induisent par leurs groupes d'homologie des modules de persistence $q$-dociles,
et qui sont $\epsilon$-intercalés quand les espaces/fonctions considérées sont $O(\epsilon)$-proches.
Plus particulièrement, on va démontrer la docilité des complexes de Rips et de \v{C}ech.


\subsubsection{Théorèmes de Stabilité}
\begin{definition}
	Une \define{application multivaluée} $C$ de $\X$ dans $\Y$ est une partie de $\X \times \Y$ qui se projette surjectivement sur $\X$ par la projection $\pi_{\X}$ définissant le produit.
	\define{L'image} $C(\sigma)$ de $\sigma \subseteq \X$ est la projection canonique sur $\Y$ de la préimage de $\sigma$ par $\pi_{\X}$.
	La \define{transposée} $\transpose{C}$ de $C$ est l'image de $C$ par la symétrie $\X \times \Y \to \Y \times \X$.
	Si $\transpose{C}$ est aussi une application multivaluée, on dit que $C$ est une \define{correspondance.}
\end{definition}

\begin{definition}
	Si $(\X, \rho_{\X}$ et $(\Y, \rho_{\Y})$ sont des espaces métriques compacts, une correspondance $C$ de $\X$ dans $\Y$ est une \define{$\epsilon$-correspondance} si:
	\begin{equation*}
		\forall (x, y), (x', y') \in C, \abs{\rho_{X}(x, x') - \rho_{\Y}(y, y')} \leq \epsilon
	\end{equation*}
\end{definition}

\begin{proposition}
	Avec les hypothèses de la définition ci-dessus:
	\begin{equation*}
		d_{GH}(\X, \Y) = \frac{1}{2}\inf \left\{\epsilon \geq 0\suchthat \exists C, C \text{ est une $\epsilon$-correspondance}\right\}
	\end{equation*}
\end{proposition}

\begin{definition}
	Si $\S$, $\T$ sont des complexes simpliciaux filtrés avec des ensembles de sommets $\X$ et $\Y$ respectivement, une application multivaluée $C$ de $\X$ dans $\Y$ est dite \define{$\epsilon$-simpliciale}
	de $\S$ dans $\T$ si pour tout $a \in \R$ et tout simplexe $\sigma \in \S_{a}$, chaque partie finie de $C(\sigma)$ est un simplexe de $\T_{a + \epsilon}$.
\end{definition}

\begin{proposition}\label{prop:intercalaire-correspondance}
	Si $C$ est une correspondance telle que $C$ et $\transpose{C}$ sont $\epsilon$-simplicales de $\S$ dans $\T$, elle permette de construire un $\epsilon$-intercalaire entre $H(\S)$ et $H(\T)$.
\end{proposition}
\begin{proof}
	Il suffit pour ça de voir qu'une fonction $\epsilon$-simpliciale définit immédiatement un homomorphisme de degré $\epsilon$ sur les modules de persistence définis par les filtrations considérées, par continuation linéaire.
\end{proof}

\begin{proposition}
	Si $\X, \Y$ sont des espaces métriques, pour tout $\epsilon > 2\d_{GH}\left(\X, \Y\right)$, alors $H\left(\rips\left(\X\right)\right)$ et $H\left(\rips\left(\Y\right)\right)$ sont $\epsilon$-intercalés.
\end{proposition}
\begin{proof}
	Si $C$ est une correspondance de $\X$ dans $\Y$ avec une distortion d'au plus $\epsilon$.
	Si $\sigma \in \rips\left(\X, a\right)$ alors $\rho_{\X}\left(x, x'\right) \leq a$ pour tout $x, x'\in \sigma$.
	Soit $\tau \subseteq C\left(\sigma\right)$ fini. Pour tout $y, y' \in \tau$, il existe $x, x'\in \sigma$ tels que:
	$y \in C\left(x\right)$ et $y'\in C\left(x'\right)$ donc:
	\begin{equation*}
		\rho_{\Y}\left(y, y'\right) \leq \rho_{\X}\left(x, x'\right) + \epsilon \leq a + \epsilon
	\end{equation*}
	Ainsi, $\tau \in \rips\left(\Y, a + \epsilon\right)$.
	De même $\transpose{C}$ est $\epsilon$-simpliciale de $\rips\left(\Y\right)$ dans $\rips\left(\Y\right)$.
	On conclut par la Proposition \ref{prop:intercalaire-correspondance}.
\end{proof}

\begin{proposition}
	Si $\X, \Y$ sont des espaces métriques, pour tout $\epsilon \geq 2\d_{GH}\left(\X, \Y\right)$, alors $H\left(\cech\left(\X\right)\right)$ et $H\left(\cech\left(\Y\right)\right)$ sont $\epsilon$-intercalés.
\end{proposition}
La preuve est similaire à celle d'avant.

\begin{thm}
	Soit $\X$ un espace métrique compact. Les modules de persistence associés à l'homologie des complexes $\rips(\X)$ et $\cech(\X)$ sont $q$-dociles.
\end{thm}
\begin{proof}
	On veut montrer que $I_{a}^{b}: H\left(\rips\left(\X, a\right)\right) \to H\left(\rips\left(\X, b\right)\right)$ sont de rang finis quand $a < b$.
	On pose $\epsilon = (b - a) / 2$ et $F \subseteq \X$ un ensemble fini tel que $d_{H}(\X, F) \leq \epsilon / 2$.
	Alors:
	\begin{equation*}
		C = \left\{\left(x, f\right) \in X \times F \suchthat d\left(x, f\right) \leq \frac{\epsilon}{2}\right\}
	\end{equation*}
	définit une $\epsilon$-correspondance.
	Utilisant l'application d'intercalage de $X$ et $F$, $I_{a}^{b}$ se factorise en:
	\begin{equation*}
		H(\rips(X, a))\longrightarrow \underset{\textcolor{vulm}{\text{de dimension finie}}}{H(\rips(F, a + \epsilon))}\rightarrow H(\rips(X, a + 2\epsilon)) = H(\rips(X, b))
	\end{equation*}
\end{proof}

\begin{thm}
	Si $\X$, $\Y$ sont des espaces métriques compacts, alors:
	\begin{align*}
		\dinf\left(\diag\left(H\left(\cech\left(\X\right)\right)\right), \diag\left(H\left(\cech\left(Y\right)\right)\right)\right) \leq & 2d_{GH}\left(\X, \Y\right) \\
		\dinf\left(\diag\left(H\left(\rips\left(\X\right)\right)\right), \diag\left(H\left(\rips\left(Y\right)\right)\right)\right) \leq & 2d_{GH}\left(\X, \Y\right) \\
	\end{align*}
\end{thm}
La preuve des deux derniers théorèmes n'utilise pas l'inégalité triangulaire on pourrait donc étendre les résultats précédents à des espaces munies d'une similarité.

\medskip

Cependant, on a des problèmes avec la dimension de nos espaces: pour tout $0 < \alpha \leq \beta \in \R$, il existe un
espace métrique compact $X$ (immersible dans $\R^{4}$) tel que pour tout $a \in [\alpha, \beta]$, $H_{k}(\rips(X, a))$ est de dimension indénombrable.
Toutefois:
\begin{itemize}
	\item Si $X$ est compact, $\dim H_{1}(\cech(X, a)) < + \infty$
	\item Si $X$ est géodésique, $\dim H_{1}(\rips(X, a)) < + \infty$ pour $a > 0$ et $\diag(H_{1}(\rips(X)))$ est contenu dans la ligne $x = 0$
	\item Si $X$ est un espace géodésique $\delta$-hyperbolique, $\diag(H_{2}(\rips(X)))$ est contenu dans une bande verticale de largeur $\O(\delta)$.
\end{itemize}

\subsection{Calculabilité et bruit}
Le complexe de Vietoris-Rips et ses filtrations se calculent en $\O\left(\abs{\X}^{d}\right)$, ce qui rend le calcul de persistence quasi impossible en pratique.
Par ailleurs, les filtrations et la distance de Gromov-Hausdorff sont très sensibles au bruit et aux anomalies.

\subsubsection{Calcul statistique}
On va s'intéresser à un espace métrique $(\mathbb{M}, \rho)$ et à une mesure de probabilité $\mu$ à support compact $X_{\mu}$ dans $\mathbb{M}$.
On échantillonne $m$ points selon $\mu$, ce qui nous donne un nuage de point $\hatx{m}$, et une filtration
$\filt{\hatx{m}}$.
On a alors:
\begin{proposition}
	Si $\epsilon > 0$:
	\begin{equation*}
		\P\left(\dinf\left(\diag\left(\filt\left(\X_{\mu}\right)\right), \diag\left(\filt\left(\hatx{m}\right)\right)\right) > \epsilon\right) \leq \P\left(d_{GH}\left(\X_{\mu}, \hatx{m}\right) > \frac{\epsilon}{2}\right)
	\end{equation*}
\end{proposition}
\begin{proof}
	Conséquence directe du Théorème \ref{thm:docile-distance-module} de stabilité.
\end{proof}

On obtient quasi immédiatement des inégalités de déviation:
\begin{definition}
	Pour $a, b > 0$, on dit que $\mu$ vérifie \define{la supposition $(a, b)$-standard} si pour $x \in \X_{\mu}$ et $r > 0$, on a:
	\begin{equation*}
		\mu(B(x, r)) \geq \min(ar^{b}, 1)
	\end{equation*}
	On note $\mP(a, b, \mathbb{M})$ \define{l'ensemble des distributions de probabilité $(a, b)$-standard} sur $\M$.
\end{definition}
\begin{thm}
	Si $\mu$ vérifie la supposition $(a, b)$-standard, pour tout $\epsilon > 0$:
	\begin{equation*}
		\P\left(\dinf\left(\diag\left(\filt\left(\X_{\mu}\right)\right), \diag\left(\filt\left(\hatx{m}\right)\right)\right) > \epsilon\right) \leq \min \left(\frac{8^{b}}{a\epsilon^{b}}\exp{\left(-ma\epsilon^{b}\right)}, 1\right)
	\end{equation*}
	De plus:
	\begin{equation*}
		\P\left(\dinf\left(\diag\left(\filt\left(\X_{\mu}\right)\right), \diag\left(\filt\left(\hatx{m}\right)\right)\right) \geq C_{1} \left(\frac{\log{m}}{m}\right)^{1 / b}\right) \xrightarrow[m \to \infty]{} 1
	\end{equation*}
	où $C_{1}$ est une constante qui ne dépend que de $a$ et $b$.
\end{thm}
\begin{proof}
	On commence par majorer $\P\left(\d_{GH}\left(\X_{\mu}, \hatx{m}\right) > \frac{\epsilon}{2}\right)$ puis,
	on obtient par la supposition $(a, b)$-standard une borne supérieure explicite pour la couverture de $\X_{\mu}$ par des boules de rayon $\epsilon / 2$.
	On peut alors conclure en prenant l'union des bornes.
\end{proof}

\begin{thm}
	On a:
	\begin{equation*}
		\sup_{\mu \in \mP(a, b, \mathbb{M})} \E[\dinf(\diag(\filt(\X_{\mu})), \diag(\filt(\hatx{m})))] \leq C(\frac{\ln m}{m})^{1 / b}
	\end{equation*}
	où $C$ ne dépend que de $a$ et $b$.
	Si de plus il y a un point non isolé $x$ dans $\mathbb{M}$, et si $x_{m} \in \mathbb{M} \setminus \{x\}$,
	telle que $\rho(x, x_{m}) \leq (am)^{-1/b}$, pour tout estimateur $\hat{\diag}_{m}$ de $\diag(\filt(\X_{\mu}))$:
	\begin{equation*}
		\liminf_{m \to \infty}\rho(x, x_{m})^{-1}\sup_{\mu \in \mP(a, b, \mathbb{M})}\E[\dinf(\diag(\filt(\X_{\mu})), \hat{\diag}_{m})] \geq C'
	\end{equation*}
	où $C'$ est une constante absolue.
\end{thm}

\subsubsection{Paysages de persistence et rééchantillonnage}
\begin{definition}
	Si on a un diagramme de persistence $(b_{i}, d_{i})$, son \define{paysage de persistence} est obtenu en y ajoutant à chaque point les deux projections orthogonales sur la diagonale par rapport aux axes, puis en plaçant à l'horizontale sa diagonale.
	Formellement, c'est l'union pour $p = (\frac{b + d}{2}, \frac{d - b}{2})$ des graphes:
	\begin{equation*}
		\Lambda_{p}(t) = \begin{cases}
			t - b & t \in [b, \frac{b + d}{2}] \\
			d - t & t \in [\frac{b + d}{2}, d] \\
			0     & \text{sinon}
		\end{cases}
	\end{equation*}
\end{definition}

C'est un encodage de la persistence comme un élément d'un espace fonctionnel, comme fait ci-dessous:
\begin{definition}
	On définit le \define{$k$-ème paysage} d'un diagramme $D$ par:
	\begin{equation*}
		\lambda_{D}(k, t) = \underset{p \in D}{\mathrm{kmax}} \Lambda_{p}(t), t \in \R, k \in \N
	\end{equation*}
	où $\mathrm{kmax}$ désigne le $k$-ème plus grand élément d'un ensemble.
\end{definition}

\begin{proposition}\label{thm:landscape-stability}
	\begin{itemize}
		\item Pour $t \in \R$ et $k \in \N$, $0 \leq \lambda_{D}(k, t) \leq \lambda_{D}(k + 1, t)$
		\item Pour $t \in \R$, et $k \in \N$, $\abs{\lambda_{D}(k, t) - \lambda_{D'}(k, t)} \leq \dinf(D, D')$
	\end{itemize}
\end{proposition}

Dans la suite, on note $\mL_{T}$ les paysages dont le support est dans $[0, T]$, on prend $P$ une distribution de probabilité
sur $\mL_{T}$ et $\lambda_{1}, \ldots, \lambda_{n} \sim P$ i.i.d.
On note $\mu(t) = \E[\lambda_{i}(t)]$ le paysage moyen et on l'estime par la moyenne échantillonnée:
\begin{equation*}
	\bar{\lambda}_{n}(t) = \frac{1}{n}\sum \lambda_{i}(t)
\end{equation*}
$\bar{\lambda}_{n}$ est un estimateur point à point non biaisé de $\mu$, qui converge point à point.

\begin{definition}
	Soit $\mF$ la famille des applications d'évaluation $f_{t} : \mL_{T} \to \R$.
	Le \define{processus empirique} indexé par les $f_{t}$ est défini par:
	\begin{equation*}
		\mathbb{G}_{n}(t) = \sqrt{n}(\bar{\lambda}_{n}(t) - \mu_{t}) = \sqrt{n}(P_{n} - P)(f_{t})
	\end{equation*}
\end{definition}

\begin{thm}
	Soit $\mathbb{G}$ un pont Brownien avec fonction de covariance:
	\begin{equation*}
		\kappa(s, t) = \int f_{t}(\lambda)f_{s}(\lambda)\d P(\lambda) - \int f_{t}(\lambda)\d P(\lambda)\int f_{s}(\lambda)\d P(\lambda).
	\end{equation*}
	On a alors:
	\begin{equation*}
		\mathbb{G}_{n} \to \mathbb{G}
	\end{equation*}
	pour la convergence faible.
\end{thm}

Si de plus on note $\sigma(t)$ l'écart-type de $\sqrt{n}\bar{\lambda}_{n}(t)$:
\begin{thm}\label{thm:bootstrap-clt}
	Si $\sigma(t) > c > 0$ sur un intervalle $I = [t_{*}, t^{*}] \subseteq [0, T]$ pour une constance $c$, avec
	$W = \sup_{t \in I} \abs{\mathbb{G}(f_{t})}$ on a:
	\begin{equation*}
		\sup_{z \in \R}\abs{\P\left(\sup_{t \in [t_{*}, t^{*}]}\abs{\mathbb{G}_{n}\left(t\right)} \leq z\right) - \P\left(W \leq z\right)} = \O\left(\frac{\left(\log n\right)^{7 / 8}}{n^{1 / 8}}\right)
	\end{equation*}
\end{thm}
C'est une forme de théorème central limite uniforme. On a de plus le corollaire suivant:

\begin{thm}
	Sous les mêmes hypothèses, étant donné un niveau de confiance $1 - \alpha$, on peut construire des fonctions de confiance
	$l_{n}(t)$ et $u_{n}(t)$ telles que:
	\begin{equation*}
		\P\left(l_{n}(t) \leq \mu(t) \leq u_{n}(t), \forall t \in I\right) \geq 1 - \alpha - \O\left(\frac{\left(\log n\right)^{7/8}}{n^{1 / 8}}\right)
	\end{equation*}
	De plus, on a:
	\begin{equation*}
		\sup_{t} u_{n}(t) - l_{n}(t) = \O\left(\sqrt{\frac{1}{n}}\right)
	\end{equation*}
\end{thm}
Autrement dit, le rééchantillonnage (ou \textit{bootstrap}) permet d'obtenir des intervalles de confiance pour les paysages.

\subsubsection{Bruit et méthode d'échantillonnage}
On va ici s'intéresser à l'impact de la procédure d'échantillonnage.
On rappelle que la définition des distances de $p$-Wasserstein peuvent se poser pour n'importe quelles
deux mesures de probabilité sur un même espace métrique $(\mathbb{M}, \rho)$.
On a notamment le théorème suivant:
\begin{thm}\label{thm:sampling-stability}
	Si $\mu, \nu$ sont des mesures de probabilité sur un même espace métrique $(\mathbb{M}, \rho)$, on a:
	\begin{equation*}
		\ninf{\Lambda_{\mu, m} - \Lambda_{\nu, m}} \leq m^{\frac{1}{p}} \mathrm{W}_{p}\left(\mu, \nu\right)
	\end{equation*}
	où $\mathrm{W}_{p}$ dénote la distance de Wasserstein associée à la fonction de coût $\rho(\cdot, \cdot)^{p}$.
\end{thm}
Ceci nous assure de l'utilisabilité des méthodes d'échantillonnage, et notamment de la robustesse
aux échantillons peu probables et des méthodes sous-échantillonnant.

Avant de démontrer ceci, donnons trois courts lemmes sur les passages des espaces d'échantillonnage aux espaces de paysages:
\begin{lemme}\label{lem:samp-one}
	Pour toutes mesures de probabilité $\mu, \nu$ sur $(\mathbb{M}, \rho)$, si $\rho_{m}$ est une métrique sur $\mathbb{M}^{m}$ telle que:
	\begin{equation*}
		\rho_{m}(X, Y) \leq \left(\sum_{i = 1}^{m}\rho(x_{i}, y_{i})^{p}\right)^{\frac{1}{p}}
	\end{equation*}
	alors:
	\begin{equation*}
		\W_{p}(\mu^{\otimes m}, \nu^{\otimes m}) \leq m^{\frac{1}{p}}\W_{p}(\mu, \nu)
	\end{equation*}
\end{lemme}
\begin{proof}
	Si $\Pi$ est un plan de transport entre $\mu$ et $\nu$, alors $\Pi^{\otimes m}$ est un plan de transport entre $\mu^{\otimes m}$ et $\nu^{\otimes m}$ et donc:
	\begin{align*}
		\int_{\MM^{2m}}\rho_{m}(X, Y)^{p}\d\Pi^{\otimes m}(X, Y) \leq & \int_{\MM^{m}\times \MM^{m}}\sum_{i = 1}^{m}\rho(x_{i}, y_{i})^{p}\d\Pi(x_{1}, y_{1})\cdots\d\Pi(x_{m}, y_{m}) \\
		= m\int_{\MM \times \MM}\rho(x_{1}, y_{1})^{p}\d \Pi(x_{1}, y_{1}),
	\end{align*}
	ce qui conclut la preuve.
\end{proof}

\begin{lemme}\label{lem:samp-two}
	En notant $\phi^{m}: \MM^{m} \to \mD$ la fonction qui à $X$ associe $\diag\left(\filt X\right)$ dans l'espace des diagrammes de persistence et si $\Phi_{\mu}^{m}$ est le poussé en avant de $\mu$ par $\phi^{m}$:
	\begin{equation*}
		\W_{p}(\Phi_{\mu}^{m}, \Phi_{\nu}^{m}) \leq \W_{p}(\mu^{\otimes m}, \nu^{\otimes m})
	\end{equation*}
\end{lemme}
\begin{proof}
	En notant $\Delta_{m}(X, Y) = (\psi(\phi^{m}(X)), \psi(\phi^{m}(Y)))$, si $\Pi$ est un plan entre $\mu^{\otimes m}$ et $\nu^{\otimes m}$, alors le plan $\Delta_{m}\sharp \Pi$ poussé en avant de $\Pi$ est un plan entre $\Phi_{\mu}^{m}$ et $\Phi_{\nu}^{m}$ et on a:
	\begin{align*}
		\int_{\mD^{2}} \W_{\infty}(D_{X}, D_{Y})^{p}\d \Delta_{m}\sharp\Pi(D_{X}, D_{Y}) = & \int_{\MM^{2m}}\W_{\infty}(\phi^{m}(X), \phi^{m}(Y))^{p}\d\Pi(X, Y)                          \\
		\leq                                                                               & \int_{\MM^{2m}} d_{H}(X, Y)^{p}\d\Pi(X, Y) \quad \text{(\ref{thm:docile-distance-function})} \\
		\leq                                                                               & \int_{\MM^{2m}}\rho_{m}(X, Y)^{p}\d\Pi(X, Y)
	\end{align*}
\end{proof}


\begin{lemme}\label{lem:samp-three}
	En notant $\psi$ l'application de l'espace des diagrammes vers l'espace des paysages munis de la norme infinie et $\Psi_{\mu}^{m}$ le poussé en avant de $\phi_{\mu}^{m}$ par $\psi$:
	\begin{equation*}
		\ninf{\E_{\lambda_{X} \sim \Psi_{\mu}^{m}}[\lambda_{X}] - \E_{\lambda_{Y} \sim \Psi_{\nu}^{m}}\left[\lambda_{Y}\right]} \leq \W_{\infty, p}(\Phi_{\mu}^{m}, \Phi_{\nu}^{m})
	\end{equation*}
	où $\W_{\infty, p}$ fait appel à la $p$-ème puissance de la distance infinie de Wasserstein pour les diagrammes sous-jacents.
\end{lemme}
\begin{proof}
	Si $\Pi$ est un plan entre $\Phi_{\mu}^{m}$ et $\Phi_{\nu}^{m}$, pour tout $t \in \R$ on a:
	\begin{align*}
		\abs{\E_{\lambda_{X} \sim \Psi_{\mu}^{m}}\left[\lambda_{X}\right]\left(t\right) - \E_{\lambda_{Y} \sim \Psi_{\nu}^{m}}\left[\lambda_{Y}\right]\left(t\right)}^{p} = & \abs{\E\left[\lambda_{X}(t) - \lambda_{Y}(t)\right]}^{p}                                             \\
		\leq                                                                                                                                                                & \E\left[\abs{\lambda_{X}(t) - \lambda_{Y}(t)}^{p}\right] \quad \text{(Jensen)}                       \\
		\leq                                                                                                                                                                & \E\left[\W_{\infty}\left(D_{X}, D_{Y}\right)^{p}\right] \quad \text{(\ref{thm:landscape-stability})} \\
		=                                                                                                                                                                   & \int_{\mD \times \mD}\W_{\infty}(D_{X}, D_{Y})^{p}\d\Pi(D_{X}, D_{Y})
	\end{align*}
\end{proof}

\begin{proof}[Preuve du Théorème \ref{thm:sampling-stability}]
	Par le Lemme \ref{lem:samp-one} ci-dessus
	\begin{equation*}
		\W_{p}\left(\mu^{\otimes m}, \nu^{\otimes m}\right) \leq m^{\frac{1}{p}}\W_{p}(\mu, \nu)
	\end{equation*}
	Par le Lemme \ref{lem:samp-two}, en notant $P_{\pi}$ le diagramme de persistence associé à $\pi$:
	\begin{equation*}
		\W_{p}(P_{\mu}, P_{\nu}) \leq \W_{p}(\mu^{\otimes m}, \nu^{\otimes m})
	\end{equation*}
	Enfin, par le Lemme \ref{lem:samp-three}:
	\begin{equation*}
		\ninf{\Lambda_{\mu, m} - \Lambda_{\nu, m}} \leq \W_{p}(P_{\mu}, P_{\nu})
	\end{equation*}
\end{proof}
