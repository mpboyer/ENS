\subsection{Invariance Topologique et Filtrations}
\subsubsection{Homologie Singulière}
On a le résultat important suivant, qui va nous permettre de simplifier la manière de définir l'homologie,
notamment dans la représentation des simplexes:
\begin{thm}
	Si $K$ et $K'$ sont des complexes simpliciaux de supports homéomorphes, leurs groupes d'homologie sont
	isomorphes et leurs nombres de Betti sont égaux.
\end{thm}
\begin{proof}
	\og Débrouille-toi, normalien.\fg
\end{proof}

On va donc définir l'homologie singulière, sur tout espace topologique, en considérant à homotopie près. On note $\Delta_{k}$ le $k$-simplexe standard dans $\R^{k}$.
\begin{definition}
	Un \define{$k$-simplexe singulier} dans un espace topologique $X$ est une fonction continue $\sigma: \Delta_{k}\to X$.
\end{definition}

On reprends les constructions de l'homologie simpliciale pour les complexes singuliers: c'est la définition de l'homologie singulière.

\begin{proposition}
	L'homologie singulière est définie pour tout espace $X$. Si $X$ est homotopiquement équivalent au support
	d'un complexe simpliciel, les homologies singulières et simpliciales coïncident.
\end{proposition}

\begin{proposition}
	Si $f: X \to Y$ est continue, et $\sigma: \Delta_{k} \to X$ est un simplexe dans $X$, alors $f\circ \sigma$ est un simplexe dans $Y$, et $f$ définit donc une application linéaire entre les groupes d'homologie:
	\begin{equation*}
		f_{\sharp}: H_{k}(X) \to H_{k}(Y)
	\end{equation*}
	Si $f$ est un homéomorphisme ou une équivalence d'homotopie, alors $f_{\sharp}$ est un isomorphisme.
\end{proposition}

\begin{remarque}
	En particulier, si $X \subset Y$, l'application d'inclusion induit une application linéaire d'homologie.
\end{remarque}

\subsubsection{Filtrations}
\begin{definition}
	Un \define{complexe simplicial filtré (ou une filtration)} $\K$ construit sur un ensemble $X$ est une famille
	$\left\{K_{a} \suchthat a \in T\right\}$, où $T \subseteq \R$, de sous-complexes d'un
	certain complexe simplicial fixé $K$ avec ensemble de sommets $X$ de sorte que $K_{a} \subseteq K_{b}$
	pour tout $a \leq b$.
\end{definition}

L'homologie persistente d'un complexe simplical filtré encode l'évolution de l'homologie des sous-complexes.

\begin{definition}
	Une \define{filtration d'un complexe simplicial fini} $K$ est une séquence de sous-complexes telle que:
	\begin{enumerate}
		\item $\emptyset = K^{0} \subset K^{1} \subset \cdots \subset K^{m} = K$
		\item $K^{i + 1} = K^{i} \cup \sigma^{i + 1}$ où $\sigma^{i + 1}$ est un simplexe de $K$.
	\end{enumerate}
\end{definition}
La famille des ensembles de sous-niveau pour une fonction est un exemple de filtration.
On verra plus bas comment définir de telles filtrations dans plus de cas.
On a toutefois un algorithme pour calculer itérativement l'homologie simpliciale d'un complexe étant donné
une filtration de celui-ci:
\begin{algorithm}
	\caption{Calcul d'homologie (simpliciale)}
	\label{alg:betti_nop}
	\begin{algorithmic}
		\Input{Une filtration $\left(K^{i}\right)_{i \leq m}$ d'un complexe simplical $K$ en dimension $d$}
		\EndInput
		\State {$\beta_{i} \gets 0$}
		\For{$i \in \left\{1, \ldots, m\right\}$}
		\State {$k \gets \dim \sigma^{i} - 1$}
		\If{$\sigma^{i}$ est dans un $\left(k + 1\right)$-cycle de $K^{i}$}
		\State{$\beta_{k + 1} \gets \beta_{k + 1} + 1$}
		\Else
		\State{$\beta_{k} \gets \beta_{k} - 1$}
		\EndIf
		\EndFor
		\Return{$\left(\beta_{0}, \ldots, \beta_{d}\right)$}
	\end{algorithmic}
\end{algorithm}

\begin{definition}
	Un $(k + 1)$-simplexe $\sigma^{i}$ est dit \define{positif} s'il est contenu dans un $(k + 1)$-cycle de $K^{i}$.
	Il est dit \define{négatif} sinon.
\end{definition}
Un $(k + 1)$-simplexe positif crée un $(k + 1)$-cycle dans $K^{i}$.
Un $(k + 1)$-simplexe négatif détruit un $k$-cycle dans $K^{i}$.

On veut donc vérifier:
\begin{equation*}
	\beta_{k}(K) = \abs{\text{$k$-simplexes positifs}} - \abs{\text{$(k - 1)$-simplexes négatifs}}
\end{equation*}

On a:
\begin{lemme}
	\label{lem:filter_cycle}
	Si $\sigma^{i}$ est un $k$-simplexe positif, il existe un unique $k$-cycle $c_{\sigma}$ tel que:
	\begin{enumerate}
		\item $c_{\sigma}$ n'est pas une bordure dans $K^{i}$
		\item $c_{\sigma}$ contient $\sigma^{i}$ mais pas d'autre $k$-simplexe positif.
	\end{enumerate}
\end{lemme}
\begin{proof}
	Par induction sur l'ordre d'apparition des simplexes dans la filtration.
\end{proof}

\begin{proof}[Preuve de correction de l'algorithme \ref{alg:betti_nop}]
	Si $\sigma^{i}$ est contenu dans un $(k + 1)$-cycle $c$ de $K^{i}$, alors ce cycle n'est pas une bordure dans $K^{i}$.
	De plus, $c$ ne peut pas être homologue à un cycle dans $K^{i - 1}$, et donc $\beta_{k + 1}\left(K^{i}\right) \geq \beta_{k + 1}(K^{i - 1}) + 1$.
	Si $\sigma^{i}$ n'est contenu dans aucun $(k + 1)$-cycle $c$ de $K^{i}$, alors $\partial\sigma^{i}$ n'est pas une bordure dans $K^{i - 1}$ et donc $\beta_{k}(K^{i}) \leq \beta_{k}\left(K^{i - 1}\right) - 1$.
\end{proof}

Cela pose quelques questions, qui vont nous conduire à introduire l'homologie persistente.

\subsection{Homologie Persistente}
\subsubsection{Sur les fonctions}
Pour définir les diagrammes de persistence pour une fonction $f: X \to \R$, on étudie ses
ensembles de sous-niveau.
On représente, pour chaque dimension $d$ d'homologie, la "durée de vie" d'une propriété topologique de dimension $d$.
Ces propriétés topologiques sont, entre autres, observées par la variation du $d$-ème nombre de Betti.
On représente alors sur un graphe $2$-D, en abscisse, la valeur $x$ pour laquelle une propriété apparaît, et en ordonnée la valeur $x'$ pour laquelle la propriété disparaît.
On a nécessairement $x' > x$.
On notera $D_{f, d}$ le diagramme défini ci-dessus, comme son ensemble de points du plan $\R_{+}^{2}$.

On définit alors une distance sur deux tels diagrammes:
\begin{definition}
	La \define{distance infinie de Wasserstein, ou distance du goulot} entre deux diagrammes $D_{1}$ et $D_{2}$ est définie par:
	\begin{equation*}
		d_{B}\left(D_{1}, D_{2}\right) = \inf_{\gamma \in \Gamma}\sup_{p \in D_{1}} \ninf{p - \gamma(p)}
	\end{equation*}
	où $\Gamma$ est l'ensemble des bijections entre $D_{1}$ et $D_{2}$.
\end{definition}
On note que, les normes étant équivalentes sur $\R^{2}$, la distance de Wasserstein $2$ est, à un facteur près, la distance infinie.

\begin{remarque}
	Pour pouvoir obtenir des bijections, on doit souvent \emph{augmenter} les diagrammes, en projetant les points de l'un sur la diagonale de l'autre et réciproquement.
\end{remarque}

On aura alors un théorème important de stabilité:
\begin{thm}\label{thm:persistence-stability}
	Pour toutes fonctions \emph{dociles} $f, g: \mathbb{X}\to \R$, $d_{B}(D_{f}, D_{g}) \leq \ninf{f - g}$.
\end{thm}


Cette définition, avec les mains, va être précisée et étendue plus bas.

\subsubsection{Sur les filtrations}
On va maintenant étendre la notion de diagrammes de persistences aux filtrations de complexes simpliciaux.
On a une relation fondamentale:
\begin{proposition}
	Si $t \leq t'$, $f^{-1}\left(\left]-\infty, t\right]\right) \subseteq f^{-1}\left(\left]-\infty, t'\right]\right)$.
	Si $f$ est définie sur les sommets d'un complexe simplicial $K$, et étendue de sorte que
	\begin{equation*}
		f\left(\sigma = \left[v_{0}, \ldots, v_{k}\right]\right) = \max f\left(v_{i}\right),
	\end{equation*}
	alors les ensembles de sous-niveau de $f$ définissent une filtration du complexe simplcial $K$.
\end{proposition}

Il suffit maintenant d'adapter l'algorithme \ref{alg:betti_nop} ci-dessus pour maintenir une base d'homologie et les paires naissance-mort d'une propriété.
On notera $H_{k}^{i} = H_{k}\left(K^{i}\right)$, et on va construire des bases par récurrence (les $H_{k}$ sont des $\mathbb{F}_{2}$-espaces vectoriels).

La base de $H_{k}^{0}$ est vide, puisque l'ensemble est vide.
Si on a construit une base de $H_{k}^{i - 1}$, on a deux cas:
\begin{enumerate}
	\item Si $\sigma^{i}$ est un $k$-simplexe positif, alors on ajoute la classe d'homologie du cycle
	      $c^{i}$ associé à $\sigma^{i}$ par le Lemme \ref{lem:filter_cycle} à la base de
	      $H_{k}^{i - 1}$ pour obtenir une base de $H_{k}^{i}$.
	\item Si $\sigma^{i}$ est un $(k + 1)$-simplexe négatif:
	      \begin{itemize}
		      \item On dénote $c^{j_{1}}, \ldots, c^{j_{p}}$ les cycles associés aux simplexes
		            positifs $\sigma^{j_{1}}, \ldots, \sigma^{j_{p}}$ de la base de $H_{k}^{i - 1}$.
		      \item On pose $d = \partial \sigma^{j} = \sum_{k = 1}^{p}\epsilon_{k}c^{j_{k}} + b$
		      \item On pose $l(i) = \max \left\{j_{k} \suchthat \epsilon_{k} = 1\right\}$
		      \item On enlève la classe d'homologie de $c^{l(i)}$ pour obtenir une base de $H_{k}^{i}$.
	      \end{itemize}
\end{enumerate}

Ceci explique comment modifier l'algorithme \ref{alg:betti_nop} pour calculer les diagrammes de persistence.
Cependant, avant de réécrire l'algorithme, on va s'intéresser à un test algorithmique pour vérifier que $\sigma^{j}$ est positif ou négatif.
Pour ce faire, on introduit la matrice de l'opérateur de bordure. On rappelle qu'on se donne une filtration:
d'un complexe simplicial fini $d$-dimensionnel
$\emptyset = K^{0} \subset K^{1} \subset \cdots \subset K^{m} = K$ telle que
$K^{i + 1} = K^{i} \cup \sigma^{i + 1}$ où $\sigma^{i + 1}$ est un simplexe de $K$.

\begin{definition}
	On pose $M = \left(m_{i, j}\right)_{1\leq i, j \leq m}$ telle que $m_{i, j} = 1 \text{ si, et seulement si } \sigma^{i} \text{ est une face de } \sigma^{j}$ et vaut $0$ sinon.
	C'est la \define{matrice de l'opérateur d'inclusion.}
\end{definition}
Pour toute colonne $C_{j}$, on définit donc $l(j)$ par:
\begin{equation*}
	(i = l(j)) \Leftrightarrow (m_{i, j} = 1 \land m_{i', j} = 0, \forall i' > i)
\end{equation*}

On obtient une version matricielle de l'algorithme de persistence:
\begin{algorithm}
	\caption{Algorithme de Persistence, version Matricielle}
	\label{alg:persistence_mat}
	\begin{algorithmic}
		\Input{Une filtration $\emptyset = K^{0} \subseteq \cdots \subseteq K^{m} = K$ d'un complexe simplical $d$-dimensionnel de sorte que $K^{i + 1} = K^{i} \cup \sigma^{i + 1}$ où $\sigma^{i + 1}$ est un simplexe de $K$}
		\EndInput
		\State{Calculer la matrice $M$ de l'opérateur de bordure.}
		\For{$j \in \left\{0, \ldots, m\right\}$}
		\While{$\exists j' < j, l(j') = l(j)$}
		\State{$C_{j} \gets C_{j} + C_{j'} \mod 2$}
		\EndWhile
		\EndFor
		\Return{Paires $\left(l(j), j\right)$}
	\end{algorithmic}
\end{algorithm}

Dans le pire des cas, on a un algorithme en $\O(m^{3})$.

\begin{proof}
	À chaque étape de l'algorithme, la colonne $C_{j}$ représente une chaîne de la forme:
	\begin{equation*}
		\partial \left(\sigma^{j} + \sum_{i < j} \epsilon_{i}\sigma^{i}\right), \epsilon_{i}\in \{0, 1\}
	\end{equation*}
	À la fin de l'algorithme, si $j$ est tel que $l(j)$ est défini, alors $\sigma^{l(j)}$ est un simplexe positif.
	Donc si à la fin de l'algorithme, $C_{j}$ est nulle alors $\sigma^{j}$ est positif.
	Donc, si $C_{j}$ n'est pas nulle, alors $(\sigma^{l(j)}, \sigma^{j})$ est une paire de persistence.
\end{proof}

\begin{definition}
	On représente sur un \define{diagramme de persistence} les \define{paires de persistence} $\left(\sigma^{l(j)}, \sigma^{j}\right)$ par $\left(l\left(j\right), j\right)$ ou $\left(f\left(\sigma^{l(j)}\right), f\left(\sigma^{j}\right)\right)$.
	On ajoute au diagramme la diagonale $\{y = x\}$ et, pour chaque simplexe positif qui n'est pas dans une paire $\sigma^{i}$, le point $(i, + \infty)$.
\end{definition}

\begin{definition}
	Si $D_{1}, D_{2}$ sont deux diagrammes (potentiellement augmentés pour avoir le même cardinal):
	\begin{description}
		\item[La Distance du Goulot] est définie par:
		      \begin{equation*}
			      d_{B}^{\infty}(D_{1}, D_{2}) = \inf_{\gamma \in \Gamma} \sup_{p \in D_{1}} \ninf{p - \gamma(p)}
		      \end{equation*}
		\item[La Distance $p$-Wasserstein] est définie, pour $p \geq 1$ par:
		      \begin{equation*}
			      W_{p}(D_{1}, D_{2}) = \inf_{\gamma \in \Gamma} (\sum_{\rho \in D_{1}} \norm{\rho - \gamma(\rho)}_{p}^{p})^{\frac{1}{p}}
		      \end{equation*}
	\end{description}
	Dans les deux cas, $\Gamma$ est l'ensemble des bijections entre $D_{1}$ et $D_{2}$.
\end{definition}

\begin{remarque}
	Ces deux définitions peuvent être vues comme le coût du transport optimal pour la norme infinie et la norme $p$.
	Ces deux définitions sont par ailleurs équivalentes à un facteur près, ce qui est important pour les théorèmes de stabilité.
\end{remarque}

\begin{thm}
	Si $f, g : X \to \R$ sont \emph{dociles}, on a:
	\begin{equation*}
		d_{B}^{\infty}(D_{f}, D_{g}) \leq \ninf{f - g}
	\end{equation*}
	où $D_{\phi}$ est le diagramme de persistence de la filtration associée aux ensembles de sous-niveau de $\phi$ sur $X$.
\end{thm}
On reviendra plus tard sur la notion de docilité.


