\subsection{Persistence et apprentissage automatique}
\subsubsection{Représentation de persistence}
Puisque l'espace des diagrammes de persistence n'est pas linéaire, les algorithmes de ML classique ne
fonctionnent pas bien.
La bibliothèque Python et C++ \emph{Gudhi} propose une large zoologie de représentations pour la persistence,
comme mesures discrètes, espaces métriques finis, racines de polynômes ou collections de fonctions 1D.

Par exemple, on peut représenter un diagramme en le plongeant dans $\R^{2}$ et l'espace des mesures par $D = \sum \delta_{p_{i}}$.
\begin{definition}\label{def:surface_persistence}
	Si on se donne un noyau $K: \R^{2} \to \R$ et $H$ une matrice de bande-passante (forme quadratique), en définissant $K_{H}(u) = \abs{H}^{-1/2}K(H^{-1/2}\cdot u)$, on obtient alors, étant donné une fonction de poids $w$, \define{la surface de persistence} de $D$ par:
	\begin{equation*}
		\forall u \in \R^{2}, \rho(D)(u) = D(wK_{H}(u - \cdot))
	\end{equation*}
\end{definition}

La question se pose alors de savoir comment choisir une représentation adaptée à un réseau de neurones.
Une réponse partielle peut être trouvée en regardant l'architecture à ensembles profonds:
on se donne $n$ points dans $\R^{d}$ et on construit un réseau dont les niveaux sont invariants
par permutation ($f \circ \sigma = f$)
\begin{thm}[Universalité]
	Une fonction $f$ est invariante par permutation si et seulement si, pour tout $X$ inclus dans un ensemble dénombrable $f(X) = \rho(\sum_{i} \phi(x_{i}))$ pour certaines fonctions $\rho$ et $\phi$.
\end{thm}

Les réseaux à niveaux invariants par permutation permettent de généraliser plusieurs approches générales en TDA, sous la forme de "niveaux de persistence"
\begin{equation*}
	\mathrm{PersLay}(\diag) = \rho(\mathrm{op}\{w(p), \phi(p)\}_{p\in \diag}),
\end{equation*}
où $\mathrm{op}$ est invariante par permutation, $w$ est une fonction de poids, et $\phi$ est une transformation permettant de se ramener à un ensemble dénombrable.

On peut par exemple retrouver la surface de persistence en se donnant $t_{1}, \ldots, t_{q}\in \R^{2}$ puis en posant:
\begin{itemize}
	\item $w(p) = w_{t}((x, y))$;
	\item $\phi_{\Gamma}: p \mapsto (\Gamma_{p}(t_{i}))_{i}$ avec $\Gamma_{p}$ la gaussienne centrée en $p$ d'écart-type fixé $\sigma$;
	\item $\mathrm{op} = \sum$.
\end{itemize}
Pour les paysages, on prend $w(p) = 1$, $\mathrm{op} = \mathrm{top-}k$ et $\phi_{\Lambda}$ l'évaluation de $\Lambda_{p}$ en $q$ paramètres $t_{1}, \ldots, t_{q}$.

\subsubsection{Différentiabilité de la persistence}
Nombre de méthodes permettent de minimiser une fonction sur l'ensemble des diagrammes, mais la plupart sont
restreintes à un type spécifique de filtration ou de fonction à minimiser.

\begin{definition}\label{def:filt_vec}
	Étant donné un ensemble $V$, un complexe simplicial $K$ sur $V$ et une filtration $K_{r}$ indexée
	par un ensemble $R \subseteq \R$, pour $\sigma \in K$ on pose
	\begin{equation*}
		\Phi_{\sigma} = \inf \left\{r\in R\suchthat \sigma \in K_{r}\right\}.
	\end{equation*}
\end{definition}

Par conséquent, une filtration de $K$ est un vecteur $\abs{K}$-dimensionnel
$\Phi = (\Phi_{\sigma})_{\sigma} \in \R^{\abs{K}}$ tel que
$\tau \subseteq \sigma \Rightarrow \Phi_{\tau}\leq \Phi_{\sigma}$.


\begin{proposition}
	L'ensemble $\filt_{K} \subseteq \R^{\abs{K}}$ des vecteurs sur $\R^{\abs{K}}$ définissant une filtration
	sur $K$ est semi-algébrique sur $\R$.
\end{proposition}

\begin{definition}
	Si $K$ est un complexe simplicial, une application $\Phi: A \to \R^{\abs{K}}$ est une \define{famille
		paramétrée de filtrations} si pour tout $\tau \subseteq \sigma$, on a $\Phi_{\tau} \leq \Phi_{\sigma}$.
\end{definition}

Le calcul de l'homologie persistente dans ce cas se fait avec l'Algorithme \ref{alg:betti_nop}, et peut donc
se voir comme suit:
\begin{category}
	\rm Filtration & \text{Appairage de Simplexes} & \text{Diagramme de Persistence}\\
	\Phi = (\Phi_{\sigma})_{\sigma \in K} \in \R^{\abs{K}}\ar[r]\ar[drr, bend right, "\pers" description]
	& \text{$p$ paires $(\sigma_{l(i)}, \sigma_{i})$, $q$ $\sigma_{l}$ esseulés avec $\abs{K} = 2p +q$}\ar[r]
	& (\Phi_{\sigma_{l(i)}}, \Phi_{\sigma_{i}}), (\Phi_{\sigma_{l}}, \infty)\ar[d, "\text{ordre lexicographique}" description]\\
	& & D(\Phi) \in \R^{\abs{K}}
\end{category}

L'application de persistence $\pers$ correspond à une permutation des coordonnées localement constante.
On va chercher à utiliser cette application pour restreindre l'ensemble des filtrations afin de pouvoir
différentier l'application $\pers$.
Pour cela, on rappelle la notion de structure o-minimale:
\begin{definition}
	Une \define{structure o-minimale} sur le corps des réels $\R$ est une collection $(S_{n})_{n\in \N}$
	où chaque $S_{n} \in \mP\left(\mP\left(\R\right)\right)$ est un ensemble de parties	de $\R^{n}$ tel que
	\begin{enumerate}
		\item $S_{1}$ est exactement la collection des unions finies de points et d'intervalles;
		\item Les parties algébriques de $\R^{n}$ sont dans $S_{n}$;
		\item $S_{n}$ est une sous-algèbre booléenne de $\R^{n}$ pour tout $n \in \N$;
		\item Si $A \in S_{n}$ et $B \in S_{m}$ alors $A\times B \in S_{n + m}$;
		\item Si $\pi_{n + 1}^{n}: \R^{n + 1} \to \R^{n}$  est la projection sur les $n$ premières coordonnées,
		      alors $\pi(S_{n + 1}) \subseteq S_{n}$;
	\end{enumerate}
	$A \in S_{n}$ est dit \define{définissable} dans la structure o-minimale.
	Pour $A \subseteq \R^{n}, f: A \to \R^{m}$ est une application définissable si son graphe est définissable
	dans $\R^{n + m}$.
\end{definition}

\begin{definition}
	Une \define{stratification} d'un espace topologique est une filtration finie par ensembles fermés $F_{i}$
	telles que la différence entre deux membres successifs de la filtration $F_{i}$ et $F_{i - 1}$ est soit
	vide soit une sous-variété lisse de dimension $i$.
	Les composantes connexes de la différence $F_{i} \setminus F_{i - 1}$ sont les \define{strates} de
	dimension $i$.
	Une stratification (dans $\R^{n}$) vérifie les \define{propriétés de Whitney} si toute paire de
	strates vérifient les deux conditions suivantes:
	\begin{enumerate}
		\item $X$ et $Y$ vérifient la condition $A$ de Whitney si, lorsqu'une suite $x_{m}$ de $X$ converge vers
		      $y \in Y$ et lorsque la suite des $i$-plans tangents $T_{m}$ à $X$ en $x_{m}$ converge vers un
		      $i$-plan $T$, alors $T$ contient le $j$-plan tangent à $Y$ en $y$;
		\item $X$ et $Y$ vérifient la condition $B$ de Whitney si, pour toute suite $x_{m}$ de $X$ et
		      toute suite $y_{m}$ de $Y$ qui convergent vers le même point $y$ de $Y$ de sorte que la suite
		      de ligne sécantes $L_{m}$ entre $x_{m}$ et $y_{m}$ converge vers une ligne $L$ et que la suite
		      de $i$-plans tangents $T_{m}$ à $X$ en $x_{m}$ converge vers un $i$-plan $T$ alors $L$ est
		      contenue dans $T$.
	\end{enumerate}
\end{definition}

\begin{proposition}
	Tous les ensembles définissables admettent des stratifications finies vérifiant les propriétés de Whitney.
\end{proposition}

\begin{proposition}
	Pour un complexe simplicial $K$, l'application
	\begin{equation*}
		\pers: \filt_K \subseteq \R^{\abs{K}} \to \R^{\abs{K}}
	\end{equation*}
	est semie-algébrique (et donc définissable dans toute structure o-minimale).
	De plus, il existe une partition semie-algébrique de $\filt_{K}$ telle que la restriction de $\pers$ à
	chaque élément de la partition est Lipschitz.
\end{proposition}

\begin{corollaire}
	Si $K$ est un complexe simplicial et $\Phi: A \to \R^{\abs{K}}$ est une famille paramétrée de
	filtrations définissable dans une structure o-minimale donnée, alors $\pers \circ \Phi$ est définissable.
\end{corollaire}

Il faut entendre définissable au sens de définissable dans une structure o-minimale donnée.
Puisque les ensembles semis algébriques définissent une structure o-minimale on peut toujours remplacer
définissable par semi-algébrique.

\begin{proposition}
	Si $K$ est un complexe simplicial et $\Phi$ est une famille paramétrée définissable sur $A$ de
	dimension finie $m$, il existe une partition finie définissable de $A$ notée $S\sqcup O_{1} \sqcup \cdots
		\sqcup O_{k}$ telle que $\dim S < \dim A = m$ et pour tout $i \leq k$, $O_{i}$ est une variété
	définissable de dimension $m$ sur laquelle $(\pers \circ \Phi)_{\mid O_{i}}: O_{i} \to \R^{\abs{K}}$
	est différentiable.
\end{proposition}

Dans le cas de la filtration de Vietoris-Rips, on prend $\Phi: A = (\R^{d})^{n} \to \R^{\abs{\Delta_{n}}}$
pour $\Delta_{n}$ le complexe simplicial des faces sur simplexe $(n- 1)$-dimensionnel et pour
$x = (x_{1}, \ldots, x_{n})\in A$ et tout simplexe $\sigma \subseteq \onen{n}$, on a
\begin{equation*}
	\Phi_{\sigma}(x) = \max_{i, j \in \sigma}\norm{x_{i} - x_{j}}
\end{equation*}

\smallskip

Pour $K$ un complexe simplicial avec $n$ sommets $v_{1}, \ldots, v_{n}$, on pose $\Phi: \R^{n} \to \R^{K}$
et pour toute fonction $f$ et tout simplexe $\sigma$
\begin{equation*}
	\Phi_{\sigma}(f) = \max_{i\in \sigma} f(v_{i})
\end{equation*}

\subsubsection{Fonctions sur la persistence}
\begin{definition}
	Une fonction
	\begin{equation*}
		E: \R^{\abs{K}} = \left(\R^{2}\right)^{p} \times \R^{q} \to \R
	\end{equation*}
	est une \define{fonction de persistence} si elle est invariante aux permutations des points du diagramme
	de persistence: pour toutes permutations $\sigma, \sigma' \in \mathfrak{S}_{p} \times \mathfrak{S}_{q}$,
	$E \circ (\sigma \otimes \sigma') = E$ (pour $\otimes$ le produit tensoriel dans Set).
\end{definition}

\begin{proposition}
	Soit $E$ une fonction de persistence.
	\begin{itemize}
		\item Si $E$ est localement Lipschitz, alors $E\circ \pers$ est localement Lipschitz.
		\item Si $E$ et $\Phi: A \subseteq \R^{d} \to \R^{\abs{K}}$ sont définissables, alors
		      $\mL = E \circ \pers \circ \Phi: A \to \R$ a une sous-différentielle de Clarke
		      \begin{equation*}
			      \partial \mL(z) = \mathrm{Conv}\left\{\lim_{z_{i} \to z}\nabla \mL(z_{i})
			      \suchthat \mL \text{ est différentiable en } z_{i}\right\}
		      \end{equation*}
		      bien définie.
	\end{itemize}
\end{proposition}

Par exemple, l'application de persistence totale $E(D) = \sum_{i = 1}^{p}\abs{d_{i} - b_{i}}$ est Lipschitz
et semie-algébrique.
De même, la distance du goulot (ou $\infty$-Wasserstein)
$E(D) = d_{B}(D, D^{*}) = \min_{m} \max_{(p, p^{*})\in m} \ninf{p - p^{*}}$, le minimum étant pris sur les
couplages partiels entre $D$ et $D^{*}$, est semie-algébrique et Lipschitz.

L'existence d'une sous-différentielle/d'un sous-gradient nous permet d'utiliser l'algorithme itératif
classique de descente stochastique de sous-gradient
\begin{equation}
	x_{k + 1} = x_{k} - \alpha_{k}\left(y_{k} + \zeta_{k}\right), y_{k} \in \partial\mL(x_{k}),\tag{PersSGD}
	\label{eq:persistence-sgd}
\end{equation}
où la suite $(\alpha_{k})$ est le taux d'apprentissage et $(\zeta_{k})$ est une suite de variables
aléatoires de bruit.

On va supposer les hypothèses classiques suivantes:
\begin{enumerate}
	\item $\alpha_{k} \geq 0, \sum \alpha_{k} = +\infty, \sum \alpha_{k}^{2} < +\infty$;
	\item $\sum \norm{x_{k}} < \infty$ presque sûrement;
	\item Si $\mathcal{F}_{k}$ est la suite croissante de $\sigma$-algèbres engendrées par les $x_{j}, y_{j},
		      \zeta_{j}$ pour $j < k$, il existe une fonction $p: \R^{d} \to \R$ bornée sur les ensembles bornés
	      telle que presque sûrement
	      \begin{equation*}
		      \E\left[\zeta_{k}\suchthat \mathcal{F}_{k}\right] = 0 \quad \E\left[\norm{\zeta_k}^{2}\suchthat \mathcal{F}_{k}\right] < p(x_{k}).
	      \end{equation*}
\end{enumerate}

On obtient alors le théorème de convergence suivant:
\begin{thm}
	Soit $K$ un complexe simplicial, $A \subseteq \R^{d}$ et $\Phi: A \to \R^{\abs{K}}$ une famille
	paramétrée de filtrations de $K$ définissable dans une structure o-minimale.
	Si $E$ est une fonction de persistence définissable telle que $\mL = E\circ \pers \circ \Phi$ est
	localement Lipschitz, alors, sous les hypothèses ci-dessus, presque sûrement, les points limites de la
	suite $(x_{k})$ obtenus par les itérations de l'Équation \eqref{eq:persistence-sgd} sont des points
	critiques de $\mL$ et la suite $(\mL(x_{k}))$ converge.
\end{thm}

\subsubsection{Densité de diagrammes de persistence espérés}
On s'intéresse ici à une nuage de points $V$ pondéré par $w: V \to \R$.
\begin{definition}
	Le \define{complexe pondéré de Vietoris-Rips} noté $\rips_{w}(V)$ est le complexe simplicial filtré indicé par $\R$ dont l'ensemble de sommet est $V$ et défini par
	\begin{equation*}
		\sigma \in [p_{0}\cdots p_{k}] \in \rips_{w}(V, \alpha) \Leftrightarrow d(p_{i}, p_{j}) \leq \alpha \land w(p_{i})\leq \alpha
	\end{equation*}
\end{definition}

On peut donc s'intéresser au calcul de diagrammes espérés, et à la densité (en tant que mesure de
probabilité) que l'espérance de diagrammes par rapport à des familles pondérées: si $\X$ est une
variable aléatoire sur $M^{n}$, et si $\mK$ est une fonction de filtration, que dire de
$\E\left[\mD\left[\mK\left(\X\right)\right]\right]$ dans le cas non-asymptotique
($\abs{\X} = n$ fixé/borné) ?

\smallskip

Pour ceci, on va simplement modifier la définition d'une filtration dans ce cas, comme dans la définition
\ref{def:filt_vec}:
\begin{definition}
	Soit $n > 0$ entier, $\mF_{n}$ l'ensemble des parties non-vides de $\onen{n}$ et $M$ une $d$-variété
	connexe lisse possiblement avec bord.
	Une fonction de filtration
	\begin{equation*}
		\phi = \left(\phi[J]\right)_{J \in \mF_{n}}: M^{n} \to \R^{\abs{\mF_{n}}}
	\end{equation*}
	est une fonction qui est
	\begin{itemize}
		\item invariante par permutation: si $\tau \in \mathfrak{S}_{n}$ a support dans $J$,
		      \begin{equation*}
			      \phi[J](x_{\tau(i)}) =\phi[J](x_{i});
		      \end{equation*}
		\item monotone: $J \subseteq J' \Rightarrow \phi[J] \leq \phi[J']$.
	\end{itemize}
	Sur $x = (x_{1}, x_{n})$, $\phi(x)$ induit un ordre sur les faces du $(n-1)$-simplexe, c'est à dire une
	filtration $\mK(x)$:
	\begin{equation*}
		\forall J \in \mF_{n}, J \in \mK(x, r) \Leftrightarrow \phi[J](x) \leq r
	\end{equation*}
\end{definition}

Dans la suite on note, pour $J \in \mF_{n}$ un simplexe et $x$ un vecteur de $M^{n}$,
$x(J) = (x_{j})_{j \in J}$.

En général, on a 5 hypothèses de base:
\begin{enumerate}[label=(K\arabic*)]
	\item Absence d'interaction: si $J \subseteq \mF_{n}$, $\phi[J](x)$ ne dépend que de $x(J)$;
	\item Invariance par permutation;
	\item Monotonie;
	\item Compatibilité: si $j \in J$ un simplexe de $\mF_{n}$, si $\phi[J](x_{1}, \ldots, x_{n})$ n'est pas
	      fonction de $x_{j}$ sur un ouvert de $U$, alors $\phi[J] \equiv \phi[J\setminus \{j\}]$ sur $U$;
	\item Régularité: $\phi$ est sous-analytique et le gradient de toutes ses entrées est non-nul presque
	      sûrement.
\end{enumerate}
On a également une hypothèse (K5') de régularité, quand le gradient de $\phi$ sur chaque entrée $J$ de
taille strictement supérieure à $1$ est non-nul presque sûrement et que $\phi[\{j\}] = 0$ pour tout $j$.

On vérifie facilement que la filtration de Vietoris-Rips vérifie les hypothèses (avec la régularité
affaiblie), par exemple.

\begin{definition}
	Soit $k \geq 0$. Pour $A \subseteq \R^{d}$ et $\delta > 0$, on considère
	\begin{equation*}
		\mH_{k}^{\delta}(A) = \inf \left\{\sum_{i} \mathrm{diam}\left(U_{i}\right)^{k}\suchthat A \subseteq \bigcup_{i} U_{i} \land \mathrm{diam}\left(U_{i}\right) < \delta\right\}.
	\end{equation*}
	La \define{mesure de Hausdorff $k$-dimensionnelle} sur $\R^{D}$ de $A$ est définie par
	\begin{equation*}
		\mH_{k}\left(A\right) = \lim_{\delta \to 0} \mH_{k}^{\delta}\left(A\right).
	\end{equation*}
\end{definition}

\begin{thm}
	Soit $n \geq 1$. On suppose que $M$ est une $d$-variété connexe réelle lisse compacte potentiellement avec
	bord, que $\X$ est une variable sur $M^{n}$ avec densité par rapport à la mesure de Hausdorff $\mH_{dn}$,
	et que $\mK$ vérifie les hypothèses (K1) à (K5) ci-dessus.
	Alors, pour tout $s \geq 0$, $\E\left[\mD_{s}\left[\mK\left(\X\right)\right]\right]$ a une densité par
	rapport à la mesure de Lebesgue sur le demi-plan $x \leq y$.

	\smallskip

	Si $\mK$ vérifie les hypothèses (K1)-(K4) et (K5'), alors pour tout $s \geq A$,
	$\E\left[\mD_{s}\left[\mK\left(\X\right)\right]\right]$ a une densité par
	rapport à la mesure de Lebesgue sur le demi-plan $x \leq y$,
	et de plus, $\E\left[\mD_{0}\left[\mK\left(\X\right)\right]\right]$ a densité par rapport à la mesure de Lebesgue sur la ligne verticale
	$\{0\} \times \left[0, \infty\right)$.
\end{thm}

\begin{thm}
	Sous les hypothèses du théorème précédent, si $\X$ a densité de classe $\cont^{k}$ par rapport à
	$\mH_{nd}$, alors de plus pour $s \geq 0$, la densité de
	$\E\left[\mD_{s}\left[\mK\left(\X\right)\right]\right]$ est $\cont^{k}$.
\end{thm}



Avant de prouver ces théorèmes (ou du moins de donner une esquisse de la preuve), rappelons quelques notion
de sous-analyticité.
\begin{definition}
	Soit $M \subseteq \R^{d}$ une $d$-sous-variété connexe réelle lisse possiblement avec bord.
	\begin{itemize}
		\item $X \subseteq M$ est \define{semi-analytique} si tout $p \in M$ a un voisinage $U_{p}$ tel que
		      \begin{equation*}
			      X \cap U_{p} = \bigcup_{i = 1}^{p}\bigcap_{j = 1}^{q}X_{i, j},
		      \end{equation*}
		      où les $X_{i, j}$ s'écrivent $f_{i,j}^{-1}\left(\left\{0\right\}\right)$ ou $f_{i, j}^{-1}\left(\left(0, \infty\right)\right)$ où les $f_{i, j}$ sont analytiques.
		\item $X \subseteq M$ est \define{sous-analytique} si tout point de $M$ admet un voisinage $U$, une
		      variété lisse $N$ et un ensemble semi-analytique $A$ relativement compact de $N \times M$ tel que
		      $X \cap U$ est la projection de $A$ sur $M$.
		\item $f: X \to \R$ est \define{sous-analytique} si son graphe est sous-analytique sur $M \times \R$.
		      On note $\mS(X)$ l'ensemble des fonctions réelles sous-analytiques sur $X$.
		\item $x \in X \subseteq M$ est \define{lisse de dimension $k$} si, dans un certain voisinage de $x$
		      dans $M$,$X$ est une $k$-sous-variété analytique.
		\item La \define{dimension} de $X$ est la dimension maximale d'un point lisse de $X$.
		\item On note $\mathrm{Reg}(X)$ les points réguliers de $X$, i.e. les points lisses de $X$ de
		      dimension $d$. C'est une partie ouverte de $M$, potentiellement vide.
		      On note $\mathrm{Sing}(X)$ les points singuliers de $X$.
	\end{itemize}
\end{definition}

Ceci nous permet de dériver quelques lemmes (sans démonstration):
\begin{lemme}
	Pour $f \in \mS(M)$, l'ensemble $A(f)$ sur lequel $f$ est analytique est une partie ouverte
	sous-analytique de $M$.
	Son complémentaire est sous-analytique de dimension $< d$.
\end{lemme}

\begin{lemme}
	Si $f, g: X \to \R$ sont sous-analytiques sur une partie sous-analytique de $M$ telles que l'image
	d'un borné est bornée, alors $fg$ et $f + g$ sont sous-analytiques, et les parties $f^{-1}(\{0\})$ et
	$f^{-1}((0, \infty))$ sont sous-analytiques dans $M$.
\end{lemme}

\begin{lemme}
	Soit $X$ sous-analytique dans $M$.
	Si la dimension de $X$ est strictement inférieure à $d$, alors $\mH_{d} = 0$
\end{lemme}

\begin{corollaire}
	On a:
	\begin{itemize}
		\item $\mH_{d}(X) = \mH_{d}(\mathrm{Reg}(X))$;
		\item Pour toute $f \in \mS(M)$, son gradient est défini partout sauf sur un ensemble sous-analytique
		      de dimension strictement inférieure à $d$ (et donc de mesure de Hausdorff nulle).
	\end{itemize}
\end{corollaire}

On va de plus avoir besoin de la formule suivante pour définir la densité de nos diagrammes:
\begin{thm}[Formule de la Co-Aire]
	Soit $M$ (respectivement $N$) une $m$-variété riemanienne de dimension $m$ (respectivement $n$).
	Supposons que $m \geq n$ et soit $\Phi: M \to N$ une application différentiable.
	On note $J \phi$ le jacobien $\sqrt{\det(\transpose{(D\Phi)}(D\Phi))}$ de $\Phi$.
	Pour $f: M \to \R^{+}$ une fonction mesurable positive, on a
	\begin{equation*}
		\int_{M} f(x)J\Phi(x)\d\mH_{m}(x) = \int_{N}\left(\int_{x\in \Phi^{-1}(\{y\})}f(x)\d\mH_{m - n}(x)\right) \d \mH_{n}(y)
	\end{equation*}
\end{thm}



\begin{proof}
	Il existe une partition du complémentaire d'un ensemble (sous-analytique) de mesure $0$ de $M^{n}$
	par des ouverts $V_{1}, \ldots, V_{R}$ telle que:
	\begin{itemize}
		\item L'ordre des simplexes de $\mK(x)$ est constant sur chaque $V_{r}$;
		\item Pour tout $r$ et tout $x \in V_{r}$,
		      \begin{equation*}
			      \mD_{s}[\mK(x)] = \sum_{i = 1}^{N_{r}}\delta_{r_{i}},
		      \end{equation*}
		      où $r_{i} = \left(\phi\left[J_{i_{1}}\right]\left(x\right),
			      \phi\left[J_{i_{2}}\right]\left(x\right)\right)$ et $N_{r}, J_{i_{1}}, J_{i_{2}}$ ne dépendent que
		      de $V_{r}$;
		\item $J_{i_{1}}$ et $J_{i_{2}}$ peuvent être choisis de sorte que
		      \begin{equation*}
			      \Phi_{ir}: x \in V_{r}\to r_{i} = \left(\phi\left[J_{i_{1}}\right]\left(x\right),
			      \phi\left[J_{i_{2}}\right]\left(x\right)\right)
		      \end{equation*}
		      a rang maximal $(2)$.
	\end{itemize}
	Le diagramme espéré peut donc s'écrire
	\begin{align*}
		\E\left[\mD_{s}\left[\mK\left(\X\right)\right]\right] & = \sum_{r = 1}^{R}\E\left[\indic\left\{\X \in V_{r}\right\}\mD_{s}[\mK(\X)]\right]                                                          \\
		                                                      & = \sum_{r = 1}^{R}\E\left[\indic \left\{\X \in V_{r}\right\}\sum_{i = 1}^{N_{r}}\delta_{r_{i}}\right]                                       \\
		                                                      & = \sum_{r = 1}^{R}\sum_{i = 1}^{N_{r}}\underbrace{\E\left[\indic\left\{\X \in V_{r}\right\}\delta_{r_{i}}\right]}_{\define{$= \mu_{i, r}$}}
	\end{align*}
	Par la formule de co-aire, en notant $\kappa$ la densité de $\X$
	\begin{align*}
		\mu_{i, r}\left(B\right) & = P\left(\Phi_{i, r}\left(\X\right) \in B, \X\in V_{r}\right)                             \\
		                         & = \int_{V_{r}}\indic\left\{\Phi_{i, r}\left(x\right) \in B\right\}\kappa(x)\d \mH_{nd}(x) \\
		                         & = \int_{u \in B}
		\underbrace{
			\int_{x \in \Phi_{i, r}^{-1}\left(u\right)}
			\left(J\Phi_{i, r}\left(x\right)\right)^{-1}\kappa\left(x\right)\d \mH_{nd - 2}(x)
		}_{\text{Densité de $\mu_{i, r}$}}\d u.
	\end{align*}
	On conclut puisque la somme de mesure à densité a densité.
\end{proof}

Si on revient à la Définition \ref{def:surface_persistence} de la surface de persistence de
$D = \sum \delta_{r_{i}}$ avec noyau $K$, matrice de bande-passante $H$ et poids $w$
\begin{equation*}
	\rho(D)(u) = \sum_{i}w(r_{i})K_{H}(u - r_{i}) = D(wK_{H}(u - \cdot)),
\end{equation*}
on peut voir la surface de persistence comme un estimateur basé sur un noyau de
$\E\left[\mD_{s}\left[\mK\left(\X\right)\right]\right]$.
