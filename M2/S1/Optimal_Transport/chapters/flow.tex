\section{Flots de Gradients de Wasserstein}
On va ici étudier les dynamiques de $\alpha_{t} \in \mP(\R^{d})$ pour $t$ une variable de temps dans $\R^{+}$.

\subsection{Formulation en vitesse}
Dans notre cas, on va utiliser une approche \emph{à la Lagrange}, on va prendre $v_{t}$ tel que:
\begin{equation*}
	\partial_{t}\mu_{t} + \mathrm{div}(\mu_{t}v_{t}) = 0
\end{equation*}
Si on suppose que $v_{t}$ est Lipschitz, il existe une unique solution en $\mu_{t}$.
Toutefois, il en existe beaucoup en $v_{t}$, puisqu'il existe beaucoup de champs vectoriels de
divergence nulle.
Dans le cas où $\mu_{t}$ a densité $\rho_{t}$, on écrira plutôt
\begin{equation}
	\partial_{t}\rho_{t} + \mathrm{div}(\rho_{t}v_{t}) = 0 \tag{Speed}\label{eq:speed}
\end{equation}
Pour se simplifier, on utilisera plutôt, dans la vraie vie,
la formulation faible de l'équation différentielle ci-dessus.
On intègre par rapport à $\phi_{t}(x)$ à support compact:
\begin{equation*}
	\int \partial_{t}\rho_{t}\phi_{t}(x)\d x + \int \mathrm{div}(\rho_{t}v_{t})\phi_{t}(x)\d x\d t
\end{equation*}
et on vérifiera à chaque fois qu'on peut réécrire nos propriétés via
\begin{equation}
	\forall \phi_{t} \quad -\int \partial_{t}\phi_{t}\d\alpha_{t}\d t - \int\scalar{\nabla\phi_{t}(x), v_{t}}\d \alpha\d t = 0 \tag{WeakSpeed}
\end{equation}

On retrouve le transport optimal dynamique du théorème \ref{thm:flot}. On remarque notamment que la vitesse
$v_{t}$ optimale est un gradient.

\subsection{Optimisation sur particules}
On cherche à optimiser $f(\alpha)$ sur $\alpha \in \mP(\R^{d})$.
Ici $\alpha$ représente une densité de particule, par exemple.
Dans le cas où $\alpha_{t} = \delta_{x_{t}}$ est une seule particule, on retrouve la descente de gradient,
si $g(x) = f(\delta_{x})$ on résout en prenant $\dot{x} = -\nabla g(x)$.

Dans le cas de $n$ particules, i.e. $\alpha_{t} = \frac{1}{n}\sum_{i = 1}^{n}\delta_{x_{i}(t)}$,
\begin{equation*}
	X(t) = \left(x_{1}(t), \ldots, x_{n}(t)\right) \in \R^{d\times n},
\end{equation*}
donc en écrivant $g(X) = f\left(\frac{1}{n}\sum \delta_{x_{i}}\right)$ on obtient la descente de particules
$\dot{X} = -\nabla g(X)$, avec
\begin{equation*}
	g(X) = \frac{1}{n^{2}}\sum_{i, j = 1}^{n}k(x_{i}, x_{j}) \quad \text{avec}\quad k(x, x') = \frac{1}{\norm{x - x'} + \epsilon}
\end{equation*}
On retrouve ainsi $f(\alpha) = \iint k(x, x')\d\alpha(x)\d\alpha(x')$.

On va donc s'intéresser à ce qui se passe quand $n \to \infty$ et essayer de faire sens de l'entropie dans ce
cadre.

Pour cela on utilise la méthode JKO (Jordan-Kinderlheren-Otto), une généralisation de la méthode
implicite d'Euler pour Wasserstein.
On se donne $\tau > 0$ un pas.
On prend $\alpha_{t + \tau} = \argmin \frac{1}{2\tau}\Wass_{2}^{2}(\alpha_t, \alpha) + f(\alpha)$.

Quand $\tau \to 0$, si on suppose que $\left(\alpha_{t}\right)_{t}$ converge vers une courbe continue
$t\in \R^{+} \mapsto \alpha_{t}$, alors elle converge vers une équation aux dérivées partielles
\eqref{eq:speed}, avec $v_{t}$ à définir.
On va avoir besoin de calculer deux dérivées différentes: la dérivée classique (aussi appelée verticale,
première variation ou Fréchet) et la dérivée de Wasserstein (aussi appelée dérivée d'Otto ou dérivée
horizontale).
La dérivée verticale est une fonction réelle $\delta f(\alpha)\in \cont(\R^{d}, \R)$ tandis que la dérivée
horizontale $\nabla_{\Wass}f(\alpha)\in \cont(\R^{d}, \R^{d})$ est un champ de vecteur.

\begin{definition}
	La dérivée verticale de $f$ en $\alpha$ est, étant donné $\gamma = \beta - \alpha$ une mesure de masse
	nulle définie par l'expansion de Taylor
	\begin{equation*}
		f(\alpha + t\gamma) = f(\alpha) t\scalar{\gamma, \delta f(\alpha)} + o(t),
	\end{equation*}
	le produit scalaire étant entendu au sens du produit fonction mesure.
\end{definition}

Dans le cas où $f(\alpha) = \int g(x)\d\alpha(x)$, on a $\nabla_{\Wass}f(\alpha) = t\int g\d \gamma = g$.
Dans le cas où $f(\alpha) = \int U(p(x))\d x$ avec $p(x) = \frac{\d\alpha}{\d x}$, $f(\alpha) = +\infty$ et
$U$ une fonction d'énergie interne, si $\alpha$ n'a pas de densité, on calcule
(avec $\eta = \frac{\d\gamma}{\d x}$)
\begin{align*}
	f(\alpha + t\gamma) & = \int U(p(x) + t\eta(x))\d x                            \\
	                    & = \int \left[U(p(x)) + tU(p(x))\eta(x) + o(t)\right]\d x \\
	                    & = \int f(\alpha) + t\int U'(p(x))\d \gamma(x) + o(t)
\end{align*}
Donc $\delta f(\alpha)(x) =U'(p(x))$.

\begin{definition}
	Le gradient de Wasserstein s'écrit donc $\nabla_{\Wass}f(\alpha) = \nabla_{\R^{d}}[\delta f(\alpha)]$.
\end{definition}

\begin{thm}
	Si $\alpha_{t}$ existe, les équations aux dérivées partielles
	\begin{equation*}
		\partial_{t}\alpha + \div(\alpha v) = 0 \quad \text{et} \quad v = -\nabla_{\Wass}f(\alpha)
	\end{equation*}
\end{thm}

Si on reprend l'exemple avec l'énergie interne, on obtient:
\begin{equation*}
	\partial_{t}\rho + \div(\rho v) + 0
\end{equation*}
avec $v = \nabla_{\Wass}f(\alpha) = \nabla\delta f(\alpha) = \nabla (\log \rho + 1) = \frac{\nabla \rho}{\rho}$, c'est à dire donc $\partial_{t}\rho = \Delta\rho$.
On retrouve l'équation de la chaleur.
