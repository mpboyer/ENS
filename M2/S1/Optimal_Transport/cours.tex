\documentclass[info, math, french]{mpb-cours}

\title{Transport Optimal Computationnel}
\author{D'après Gabriel Peyré}

\DeclareMathOperator{\Wass}{W}
\DeclareMathOperator{\diag}{diag}
\def\WW{\mathbb{W}}
\def\X{\mathcal{X}}
\def\Y{\mathcal{Y}}

\begin{document}
\bettertitle
\begin{abstract}
	\url{mailto:gabriel.peyre@ens.fr}
	Notes de cours sur \url{https://arxiv.org/abs/2505.06589}
	Syllabus: \url{https://docs.google.com/document/u/0/d/1JlDpcS0tkzX8CSgHlUf13ZHQRWu650EtNLrycT39dxk/mobilebasic}
\end{abstract}

\section*{Introduction}
L'une des motivations principales du cours est de comparer, en apprentissage statistique,
des données sous formes de distributions de probabilités, souvent discrètes (nuages de points).
On peut penser au transport optimal comme de l'apprentissage non-supervisé: comment associer une
distribution de probabilité paramétrique $\alpha_{\theta}$ à une probabilité observée $\beta$ sur un groupe de points.
Dans ce cours, les lettres grecques sont réservées aux distributions de probabilités et les lettres latines
aux points.
Pour ce faire, on va faire une association de densité $\min_{\theta} D(\alpha_{\theta}, \beta)$ qui
prend en compte une métrique $d$.
L'idée étant d'utiliser la métrique $d$ pour définir la métrique $D$, en utilisant la structure
de l'espace sous-jacent pour les données.
Il faut voir le transport optimal comme un mécanisme d'élévation de l'espace des données vers un espace
de probabilité, de sorte que $D = d$ lorsqu'on considère des diracs.


\input{chapters/monge}

\section{Formulation de Kantorovitch}
\subsection{Définition Discrète}
La formulation de Kantorovitch est une relaxation convexe de la formulation de Monge.
Il a obtenu un prix Nobel d'économie pour ceci.
Ici, on se limite au cas discret $\alpha = \sum_{n} \alpha_{i}\delta_{x_{i}}$ et $\beta = \sum_{m} b_{j}\delta_{y_{j}}$.

\begin{definition}
	Un \emph{couplage} ou un \emph{plan} est une matrice $M \in \R_{+}^{n \times m}$ qui représente le coût de transport de $x_{i}$ à $y_{j}$ et telle que:
	\begin{equation*}
		\sum_{j} P_{i, j} = a_{i} \land \sum_{i}P_{i, j} = b_{j}
	\end{equation*}
\end{definition}
\begin{remarque}
	C'est la même notion que celle de couplage en probabilité: un vecteur aléatoire sur l'espace produit.
\end{remarque}

On remarquera que les équations définissant un plan peuvent se mettre sous la forme:
\begin{equation*}
	P\mathds{1}_{m} = a \text{ et } \transpose{P}\mathds{1}_{n} = b
\end{equation*}

On autorise ainsi la division de masse, les problèmes de Kantorovitch devenant des problèmes sur des graphes bipartis.

\begin{definition}
	Le polytope des couplages entre $\alpha$ et $\beta$ est l'ensemble des plans entre $\alpha$ et $\beta$:
	\begin{equation*}
		\mathrm{Couplages}(\alpha, \beta) = \left\{P \in \R_{+}^{n \times m} \suchthat P_{i, j} \geq 0, P\mathds{1}_{m} = a \text{ et } \transpose{P}\mathds{1}_{n} = b\right\}
	\end{equation*}
\end{definition}

Kantorovitch avait fait l'hypothèse très forte que l'économie est linéaire.

\begin{definition}
	Le \emph{problème de Kantorovitch} est le problème d'optimisation suivant:
	\begin{equation}
		P = \argmin_{P} \left\{\scalar{C, P} \suchthat P \in \mathrm{Couplages}(\alpha, \beta)\right\} \tag{Kantorovitch}\label{eq:Kd}
	\end{equation}
	où $C \in \R^{n \times m}$ est une matrice de coût. On dit que $P$ est le plan optimal.
\end{definition}

C'est un problème de programmation linéaire.
En général la méthode du simplexe n'est pas polynomial, mais il existe un type de simplexes pour lequel elle l'est, et est en $\O\left(\left(n^{3}m + m^{3}n\right)\log\left(mn\right)\right)$

\begin{proposition}
	Il existe toujours une solution, et il existe toujours une solution dite \emph{éparse}, telle que:
	\begin{equation*}
		\abs{\left\{(i, j) \suchthat P_{i, j}\neq 0\right\}} \leq n + m - 1
	\end{equation*}
\end{proposition}
\begin{proof}
	La preuve d'existence vient du fait que l'ensemble des couplages est un compact non vide (car $P = a\transpose{b}$ est un couplage dit \emph{indépendant}).
\end{proof}
Le cas générique est de plus unique, c'est-à-dire que si $C, \alpha, \beta$ n'a pas une unique solution, en ajoutant du bruit on retrouve une solution optimale.

\subsection{Équivalence à Monge}

On s'intéresse ensuite aux matrices de permutation $P_{n}$, dans le cas $n = m$.
On cherche à résoudre le problème non-convexe $\displaystyle\min_{P\in P_{n}} \scalar{C, P}$.
Clairement, si $\B_{n}$ est l'ensemble convexe des matrices bistochastiques (l'ensemble des couplages~!):
\begin{equation*}
	\min_{P \in \B_{n}} \scalar{C, P} \leq \min_{P \in P_{n}}\scalar{C, P}
\end{equation*}

\begin{definition}
	L'ensemble des points extrême d'un convexe $C$ est:
	\begin{equation*}
		\mathrm{Extr}(C) = \left\{P \suchthat \forall \left(Q, R\right) \in C^{2}, P = \frac{Q + R}{2} \Rightarrow Q = R\right\}
	\end{equation*}
\end{definition}

\begin{thm}
	Si $C$ est compact, $\mathrm{Extr}(C) \neq \emptyset$.
\end{thm}

\begin{thm}[Krain-Millman]
	Si $C$ est un compact convexe, alors $C = \mathrm{Hull}\left(\mathrm{Extr}\left(C\right)\right)$.
\end{thm}

\begin{proposition}
	Si $C$ est compact:
	\begin{equation*}
		\mathrm{Extr}(C) \cap \left(\argmin_{p\in C} \scalar{C, P}\right) \neq \emptyset
	\end{equation*}
\end{proposition}
\begin{proof}
	En notant que l'ensemble de droite est convexe et compact, on trouve $\mathrm{Extr}(S) \neq \emptyset$ et que
	$\mathrm{Extr}(S) \subseteq \mathrm{Extr}(C)$.
\end{proof}

\begin{thm}[Birkhoff - von Neumann]
	On a $\mathrm{Extr}(\B_{n}) = P_{n}$
\end{thm}
\begin{proof}
	On montre d'abord $P_n \subset \mathrm{Extr}(\B_n)$.
	Cela découle de $\text{Extr}([0,1]) = \{0,1\}$.
	Si $P \in P_n$, est telle que $P=(Q+R)/2$ avec $Q_{i,j},R_{i,j} \in [0,1]$, puisque $P_{i,j} \in \{0,1\}$ alors nécessairement $Q_{i,j}=R_{i,j} \in \{0,1\}$.

	Montrons maintenant $\text{Extr}(\B_n) \subset P_n$ en montrant que
	$P_n^c \subset \text{Extr}(\B_n)^c$ avec le complémentaire considéré dans $\B_n$.
	Choisir $P \in \B_n \setminus P_n$ revient à choisir $\P = (Q+R)/2$ où $Q,R$ sont des matrices
	bistochastiques distinctes.
	$P$ définit un graphe biparti de taille $2n$.
	Le graphe est composé d'arêtes isolées quand $P_{i,j}=1$ et d'arêtes connectées quand $0 < P_{i,j} <1$.
	Si $i$ est un tel sommet à gauche ($j$ à droite), puisque $\sum_j P_{i,j}=1$, il y a deux arêtes $(i,j_1)$ et $(i,j_2)$ en sortant (de même, $(i_1,j)$ et $(i_2,j)$ entrant en $j$).
	On peut donc toujours extraire un cycle par récurrence de la forme:
	\begin{equation*}
		\left(i_1,j_1,i_2,j_2,\ldots,i_p,j_p\right),
		\quad \text{i.e.}\quad i_{p+1}=i_1.
	\end{equation*}
	On suppose que ce cycle est le plus court de l'ensemble fini de cycle. On a toujours:
	\begin{equation*}
		0 < P_{i_s,j_s}, P_{i_{s+1},j_s} < 1.
	\end{equation*}
	Les $(i_s)_s$ et $(j_s)_s$ sont distincts puisque le cycle est le plus court. On pose:
	\begin{equation*}
		\epsilon = \min_{0 \leq s \leq p} \left\{ P_{i_s,j_s}, P_{j_s,i_{s+1}}, 1-P_{i_s,j_s}, 1-P_{j_s,i_{s+1}} \right\}
	\end{equation*}
	c'est-à-dire $0 < \epsilon < 1$.
	En séparant le graphe en deux ensembles d'arêtes:
	\begin{equation*}
		\A = \left\{(i_s,j_s)\right\}_{s=1}^p
		\quad \text{ et } \quad
		\B = \left\{(j_s,i_{s+1})\right\}_{s=1}^p.
	\end{equation*}
	On pose $Q$ et $R$ telle que:
	\begin{equation*}
		Q_{i,j} =
		\begin{cases}
			P_{i,j}            & \text{si } (i,j) \notin \A \cup \B, \\
			P_{i,j}+\epsilon/2 & \text{si } (i,j) \in \A,            \\
			P_{i,j}-\epsilon/2 & \text{si } (i,j) \in \B,
		\end{cases}
		\quad \text{ et }\quad
		R_{i,j} =
		\begin{cases}
			P_{i,j}            & \text{si } (i,j) \notin \A \cup \B, \\
			P_{i,j}-\epsilon/2 & \text{si } (i,j) \in \A,            \\
			P_{i,j}+\epsilon/2 & \text{si } (i,j) \in \B,
		\end{cases}.
	\end{equation*}
	Par définition d'$\epsilon$, on a $0 \leq Q_{i,j}, R_{i,j} \leq 1$.
	Puisque chaque arête gauche de $\A$ a une arête droite dans $\B$, (et réciproquement) la
	contrainte de somme sur les lignes (et sur les colonnes) est maintenue, donc $Q,R \in \B_n$. Finalement, on trouve: $P=(Q+R)/2$.
\end{proof}

\begin{corollaire}
	Pour $m = n$ et $a = b = \mathds{1}_{n}$, il existe une solution optimale pour le problème \ref{eq:Kd}, qui est une matrice de permutation associée à une permutation optimale pour le problème \ref{eq:M}.
\end{corollaire}



\section{Métrique de Wasserstein}
\subsection{Formulation continue de Kantorovitch}
On se place ici dans le cadre continu, où $\alpha$ et $\beta$ sont des mesures de probabilité arbitraires sur $\X$, $\Y$.
On notera $P_{1}$ et $P_{2}$ les projections
\begin{definition}
	Un \emph{couplage} entre $\alpha$ et $\beta$ est une mesure de probabilité $\pi \in \M_{+}^{1}(\X \times \Y)$ telle que $P_{1} \sharp \pi = \alpha$ et $P_{2} \sharp \pi = \beta$.
	On note $\mU(\alpha, \beta)$ l'ensemble des couplages entre $\alpha$ et $\beta$.
\end{definition}
En prenant $\alpha$ et $\beta$ discrètes, on vérifie bien qu'on retrouve la forumlation de l'Equation \ref{eq:Kd}.
\begin{remarque}
	Dans le cas où $\mU(\alpha, \beta)$ est non vide, le produit tensoriel $\alpha \otimes \beta$ est un couplage, dit \emph{indépendant}.
\end{remarque}

\begin{definition}
	La formulation de Kantorovitch est le problème d'optimisation suivant:
	\begin{equation*}
		\mL_{c}(\alpha, \beta) = \inf_{\pi \in \mU(\alpha, \beta)} \int_{\X \times \Y}c(x, y)\d \pi(x, y) = \inf_{(X, y) \sim (\alpha, \beta)}\E[c(X, Y)] \tag{Kantorovitch}\label{eq:K}
	\end{equation*}
	où $c: (\X \times \Y) \to \R^{+}$ est la fonction de coût.
\end{definition}

\begin{proposition}
	On se place dans le cas $\X = \Y = \R^{d}$, $c(x, y) = \norm{x - y}^{2}$.
	Si $\alpha$ a densité par rapport à la mesure de Lebesgue, avec $T = \nabla \phi$ l'application
	optimale pour \ref{eq:Mc}, $\pi = (\Id, T) \sharp \alpha$ est l'application optimale pour le problème
	de \ref{eq:K}.
\end{proposition}

\subsection{Propriété Métriques du Transport}
Ici, on se place dans le cas $\X = \Y$ et on se donne une métrique $d$ sur $\X$.
On supposera $c = d^{p}$ pour un certain $p \geq 1$.

\begin{definition}
	Pour $\alpha, \beta \in \mP_{p}(\X)$, on définit la distance $p$-Wasserstein comme:
	\begin{equation*}
		\Wass_{p}(\alpha, \beta) = \left(\mL_{d^{p}}(\alpha, \beta)\right)^{1 / p} = \left(\inf_{\pi \in \mU(\alpha, \beta)}\int d^{p}(x, y)\d \pi(x, y)\right)^{\frac{1}{p}}
	\end{equation*}
\end{definition}

\begin{remarque}
	On a une version définie à partir du problème de Monge:
	\begin{equation*}
		\inf_{T\sharp \alpha = \beta}\int d^{2}(x, T(x))\d \alpha(x)
	\end{equation*}
	mais c'est seulement une pseudo-distance.
\end{remarque}

\begin{proof}
	Il reste à prouver que $\Wass_{p}$ définit bien une distance.
	Pour ça, on a simplement besoin de la séparation et de l'inégalité triangulaire.
	\begin{description}
		\item[Séparation] On a: $\Wass_{p}(\alpha, \beta) = 0 \Rightarrow \int d (x, y)\d \pi^{*}(x, y) = 0 \Rightarrow d(x, y) = 0, \pi^{*}$ presque surement.
		      Ceci implique que $\pi^{*}$ est supporté sur la diagonale de $\X^{2}$, c'est-à-dire $P_{1}\sharp \pi^{*} = \lambda = P_2 \sharp \pi^{*}$ et donc $\alpha = \beta$.
		\item[Inégalité Triangulaire] On ne donne la preuve que dans le cas discret, $\alpha = \sum_{i} a_{i}\delta_{x_{i}}$, $\beta=\sum_{j} b_{j} \delta_{y_{j}}$, $\gamma = \sum_{k} c_{k}\delta_{z_{k}}$.
		      En prenant $\pi_{\alpha\beta}$ et $\pi_{\beta\gamma}$ des plans optimaux, on pose:
		      \begin{equation*}
			      S_{ijk} = \frac{\pi_{\alpha\beta}(i, j)\pi_{\beta\gamma}(j, k)}{b_{j}}
		      \end{equation*}
		      si $b_{j} \neq 0$, $0$ sinon.
		      On a alors $\sum_{i} S_{ijk} = \pi_{\beta\gamma}(j, k)$ et $\sum_{k} S_{ijk} = \pi_{\alpha\beta}(i, j)$.
		      Les trois marginales de $S$ sont $(\alpha, \beta, \gamma)$.
		      On définit alors:
		      \begin{equation*}
			      \pi_{\alpha\gamma} = \sum_{i, k} \underbrace{(\sum_{j} S_{ijk})}_{\pi_{\alpha\gamma}(i, k)}\delta_{x_{i}, z_{k}}
		      \end{equation*}
		      On vérifie aisément que c'est bien un couplage entre $\alpha$ et $\gamma$.
		      On a alors:
		      \begin{align*}
			      W_{p}(\alpha, \gamma) \leq & \left(\sum_{i, k}\pi_{\alpha\gamma}(i, k)d\left(x_{i}, z_{k}\right)^{p}\right)^{1/p}                                                                                  \\
			      =                          & \left(\sum_{i, j, k}S_{ijk}d\left(x_{i}, z_{k}\right)^{p}\right)^{1/p}                                                                                                \\
			      \leq                       & \left(\sum_{i, j, k}S_{ijk}\left(d\left(x_{i}, y_{j}\right) + d\left(y_{j}, z_{k}\right)\right)^{p}\right)^{1/p}                                                      \\
			      \leq                       & \left(\sum_{i,j,k}S_{ijk}d\left(x_{i}, y_{j}\right)^{p}\right)^{1/p} + \left(\sum_{i,j,k}S_{ijk}d\left(y_{j},z_{k}\right)^{p}\right)^{1/p}                            \\
			      =                          & \left(\sum_{ij}\pi_{\alpha\beta}(i, j)d\left(x_{i}, y_{j}\right)^{p}\right)^{1/p} + \left(\sum_{jk}\pi_{\beta\gamma}(j, k)d\left(y_{j}, z_{k}\right)^{p}\right)^{1/p} \\
			      =                          & W_{p}(\alpha, \beta) + W_{p}(\beta, \gamma)
		      \end{align*}
		      Pour étendre la preuve à des mesures générales, on utilise le lemme de recollage.
	\end{description}
\end{proof}

Cette structure permet de munir l'espace $\mP_{p}(\X)$ d'une structure d'espace métrique.
Dans la suite, on va supposer que $\X$ est compact, mais les définitions peuvent être étendues à $\R^{d}$ par exemple.

\begin{proposition}
	Si $\X$ est borné, pour $1\leq p \leq q$, on a:
	\begin{equation*}
		\Wass_{p}(\alpha, \beta) \leq \Wass_{q}(\alpha, \beta)\leq (\mathrm{diam } \X)^{\frac{q -p}{q}}\Wass_{p}(\alpha, \beta)^{p/q}
	\end{equation*}
\end{proposition}
\begin{proof}
	En posant $\phi(s) = s^{q/p}$ (qui est convexe), par l'inégalité de Jensen:
	\begin{equation*}
		\Wass_{p}(\alpha, \beta)^{q} \leq (\int d(x, y)^{p}\d \pi(x, y))^{q / p} \leq \int d(x, y)^{q}\d \pi(x, y)
	\end{equation*}
	Donc:
	\begin{equation*}
		\Wass_{p}(\alpha, \beta) \leq \inf (\int d(x, y)^{q}\d\pi(x, y))^{1/q} = \Wass_{q}
	\end{equation*}
	La preuve est la même pour l'autre inégalité, en utilisant $d^{q} \leq (\mathrm{diam }\X)^{q - p}d^{p}$.
\end{proof}

\begin{remarque}
	La propriété précédente montre que sur un compact, toutes les distances $p$-Wasserstein définissent la même topologie.
\end{remarque}

\begin{definition}
	Une suite $(\alpha_{n})$ converge $\star$-faiblement vers $\alpha$, noté $\alpha_{n}\xrightarrow{\star}\alpha$ dans $\M_{1}^{+}(\X)$ si pour tout $f$:
	\begin{equation*}
		\int f\d\alpha_{n} \to \int f \d\alpha
	\end{equation*}
\end{definition}

\begin{remarque}
	\begin{itemize}
		\item Dans le cas où $\alpha_{n} = \delta_{x_{n}}$, on a convergence vers $\alpha = \delta_x$ si et seuleement $x_{n} \to x$.
		\item La convergence $\star$-faible est la convergence en loi pour les variables aléatoires.
	\end{itemize}
\end{remarque}

\begin{definition}
	La topologie forte sur $\M_{+}^{1}(\X)$ est celle définie par la distance de variation totale.
\end{definition}

\begin{proposition}
	Si on prend $d$ la distance $0$-$1$, alors $\Wass_{p}^{p}(\alpha, \beta) = \frac{1}{2}\norm{\alpha -\beta}_{TV}$.
\end{proposition}

\begin{proposition}
	Si $\X$ est compact, alors $\alpha_{n}$ converge $\star$-faiblement si et seulement si $W_{p}(\alpha_{n}, \alpha) \xrightarrow[n \to +\infty]{} 0$.
	$W_{p}$ métrise donc la convergence $\star$-faible.
	Si $\X$ n'est pas compact, la convergence pour $W_{p}$ équivaut à la convergence $\star$-faible et à la convergence des $p$-ème moments.
\end{proposition}

La convergence des sommes de Riemann est équivalente à la convergence $\star$-faible de $\frac{1}{n}\sum_{k} \delta{k/n} \to \mU_{[0, 1]}$.

\subsection{Interpolation de McCann}
On considère deux mesures de probabilité $\alpha, \beta$ sur $\X = \R^{d}$.

\begin{definition}
	Pour $\pi$ un plan optimal entre $\alpha$ et $\beta$, et pour $t \in [0, 1]$, on définit:
	\begin{equation*}
		\pi_{t} = P_{t}\sharp \pi \text{ où } P_{t}(x, y) = (1-t)x + ty
	\end{equation*}
	C'est une homotopie, qu'on appelle \emph{Interpolation de McKann} ou \emph{Interpolation de Déplacement}.
\end{definition}

\begin{proposition}
	L'interpolation de McCann vérifie:
	\begin{equation*}
		\alpha_{t}\in \argmin_{\rho} (1 - t)\Wass_{2}^{2}(\alpha, \rho) + t\Wass_{2}^{2}(\beta, \rho)
	\end{equation*}
\end{proposition}

\begin{proposition}
	L'espace $(\mP_{2}(\R^{d}), \Wass_{2})$ est un espace géodésique.
\end{proposition}
\begin{proof}
	Si $T$ est l'application optimale de Monge pour $\alpha, \beta$:
	\begin{align*}
		\Wass_{2}^{2}(\alpha_{t}, \alpha_{s}) = & \int \norm{(1-t)x + tT(x) - (1-s)x - sT(x)}^{2}\d \alpha(x) \\
		=                                       & \int\norm{(t - s)(T(x) - x)}^{2}\d\alpha(x)                 \\
		=                                       & \abs{t - s}^{2}W_{2}(\alpha, \beta)
	\end{align*}
\end{proof}


\section{Gaussiennes et Transport Optimal, Dualité}

\subsection{Dualité pour les mesures discrètes}

\begin{proposition}
	Si $\alpha = \sum a_{i}\delta_{x_{i}}$ et $\beta = \sum b_{j}\delta_{y_{j}}$, et $C$ est une matrice $n \times m$ de coût, en notant $U(\alpha, \beta)$ l'ensemble des couplages entre $\alpha$ et $\beta$:
	\begin{equation*}
		L_{c}(\alpha, \beta) = \max_{(u, v) \in R\left(\alpha, \beta\right)} \scalar{u, \alpha} + \scalar{v, \beta}
	\end{equation*}
	où $R(\alpha, \beta) = \left\{(u, v) \in \R^{n} \times \R^{m} \suchthat u \oplus v \leq C\right\}$.
\end{proposition}
\begin{proof}
	On a $L_{c}(\alpha, \beta) = \min_{\pi \in U(\alpha, \beta)} \scalar{c, \pi}$.
	On introduit donc:
	\begin{equation*}
		\max_{(u, v) \in \R^{n} \times \R^{m}} \scalar{u, a - \pi\mathds{1}_{m}} + \scalar{v + b-\transpose{\pi}\mathds{1}_{n}} =
		\begin{cases}
			0        & \text{si } \pi \in U(\alpha, \beta) \\
			+ \infty & \text{sinon}
		\end{cases}
	\end{equation*}
	On a donc:
	\begin{equation*}
		L_{c}(\alpha, \beta) = \min_{\pi \geq 0}\max_{(u, v) \in \R^{n} \times \R^{m}} L(\pi, u, v)
	\end{equation*}
	avec:
	\begin{equation*}
		\underset{\rm Lagrangien}{L(\pi, u, v)} ) \scalar{c, \pi} + \scalar{u, \alpha - \pi\mathds{1}_{m}} + \scalar{v + \beta-\transpose{\pi}\mathds{1}_{n}}
	\end{equation*}
	Puisque dans le cadre de la programmation linéaire, avoir des solutions suffit pour avoir la dualité forte:
	\begin{align*}
		L_{c}(\alpha, \beta) = & \max_{u, v} \min_{\pi\geq 0}L(\pi, u, v)                               \\
		=                      & \max_{u, v} \left(\scalar{u, \alpha} + \scalar{v, \beta} + \underset{=
			\begin{cases}
				0       & \text{si } u \oplus v \leq C \\
				-\infty & \text{sinon }
			\end{cases}%
		}{\min_{\pi \geq 0} \scalar{c, \pi} - \scalar{u, \pi\mathds{1}_{m}} - \scalar{v, \transpose{\pi}\mathds{1}_{n}}}\right)
	\end{align*}
	Finalement:
	\begin{equation*}
		\boxed{L_{c}(\alpha, \beta) = \max_{(u, v) \in R(\alpha, \beta)} \scalar{u, \alpha} + \scalar{v, \beta}}
	\end{equation*}
	ce qui conclut la preuve.
\end{proof}

\begin{remarque}
	\begin{itemize}
		\item Si $\pi$ est optimal, son support est inclus dans $\left\{(i, j)\suchthat u_{i} + v_{j} = C_{i, j}\right\}$
		\item $L_{c}(\alpha, \beta)$ est convexe en $\alpha, \beta$ (dual) mais est concave en $C$.
	\end{itemize}
\end{remarque}

\subsection{Dualité pour des mesures quelconques}
\begin{proposition}
	Si $\X, \Y$ sont compacts, alors:
	\begin{equation*}
		L_{C}(\alpha, \beta) = \sup_{(f, g) \in R(C)} \int_{\X}f\d \alpha + \int_{\Y}g\d\beta
	\end{equation*}
	où:
	\begin{equation*}
		R(C) = \left\{(f, g) \in \mC(\X) \times \mC(\Y)\suchthat \forall (x, y), f(x) + g(y) \leq C(x, y)\right\}
	\end{equation*}
	On dit que $f, g$ sont des \emph{potentiels de Kantorovitch}.
\end{proposition}
Dans le cas où $\X$ n'est pas compact, on remplace $\mC$ par $\mC_{b}$ et le résultat tient.
\begin{remarque}
	\begin{itemize}
		\item Si $\pi$ est optimal, de même, son support est tel que les potentiels sont égaux au coût.
		\item Dans le cas où $\alpha$ et $\beta$ on retrouve le résultat discret.
	\end{itemize}
\end{remarque}

\subsection{$c$-transformations}
On cherche à résoudre:
\begin{equation*}
	\max_{f \oplus g\leq c} \int g\d\beta, \quad f \text{ fixée}
\end{equation*}
On veut donc prendre $g$ aussi grande que possible en ayant:
\begin{align*}
	g(y) \leq & c(x, y) - f(x)                \\
	\leq      & \inf_{x\in \X} c(x, y) - f(x)
\end{align*}

\begin{definition}
	Étant donnée $f: \X \to \bar{\R}$, sa \emph{$c$-transformée} est:
	\begin{equation*}
		f^{c}: \applic{\Y}{\bar{\R}}{f^{c}(y)}{\inf_{x\in \X} c(x, y) - f(x)}
	\end{equation*}
	La \emph{$\bar{c}$-transformée} de $g: \Y\to \bar{R}$ est:
	\begin{equation*}
		g^{\bar{c}}(x) = \inf_{y\in \Y} c(x, y) - g(y)
	\end{equation*}
\end{definition}
Si $c$ est symmétrique, alors $f^{c} = f^{\bar{c}}$.
\begin{remarque}
	La transformation $(f, g) \to (f, f^{c})$ remplace les potentiels duaux par de meilleurs.
	De même pour $(f, g) \to (g^{\bar{c}}, g)$ et $(f, g) \to (g^{\bar{c}}, f^{c})$.
\end{remarque}

\begin{proposition}
	Si $c$ est $L$-lipschitzienne en $y$, alors $f^{c}$ est lipschitzienne.
\end{proposition}
\begin{proof}
	Exercice.
\end{proof}

À ce stade, on aurait envie d'itérer à partir de potentiels de base puis d'utiliser les $c$-transformées
pour obtenir un meilleur résultat.
La proposition ci-dessous montre que malheureusement ceci ne fonctionnera pas.

\begin{proposition}
	Si on note $f^{c\bar{c}} = \left(f^{c}\right)^{\bar{c}}$ alors:
	\begin{enumerate}
		\item $f \leq \phi \Rightarrow f^{c} \geq \phi^{c}$;
		\item $f^{c\bar{c}} \geq f$;
		\item $g^{\bar{c}c} \geq g$;
		\item $f^{c\bar{c}c} = f^{c}$.
	\end{enumerate}
\end{proposition}
\begin{proof}
	\begin{enumerate}
		\item Par définition.
		\item On a:
		      \begin{align*}
			      f^{c\bar{c}} = & \inf_{y \in \Y} \left(c(x, y) \underbrace{- \underbrace{\inf_{x' \in \X} \left(c(x', y) - f\left(x'\right)\right)}_{\leq c(x, y) - f(x)}}_{\geq -\left(c(x, y) - f(x)\right)}\right) \\
			      \geq           & \inf_{y \in \Y} \left(c(x, y) - c(x, y) + f(x)\right)
		      \end{align*}
		\item De même.
		\item On a $f^{c\bar{c}} \geq f \Rightarrow f^{c\bar{c}c} \leq f^{c}$. Avec $g = f^{c}$, on a $f^{c\bar{c}c}\geq f^{c}$.
	\end{enumerate}
\end{proof}


\subsection{Quelques Cas particuliers}
\subsubsection{Cas Euclidien Quadratique}
On veut ici calculer:
\begin{equation*}
	\min_{X \sim \alpha, Y \sim \beta} \E\left(\norm{X - Y}^{2}\right) = K - 2 \max_{X \sim \alpha, Y \sim \beta} \E\left(\scalar{X, Y}\right)
\end{equation*}
où $K$ est une constante ne dépendant que de $\alpha$ et $\beta$.
On va donc vérifier qu'écrire $\phi$ sous la forme $(\alpha, \nabla\phi\sharp\alpha)$ fonctionne, c'est-à-dire redémontrer le théorème de Brenier:

\begin{proof}[Preuve du théorème de Brenier]
	On prend ici $c(x, y) = -\scalar{x, y}$, qui est symmétrique, on obtient que:
	\begin{equation*}
		f^{c}(y) = -\sup_{x} \scalar{x, y} + f(x) = -(-f)^{*}(y)
	\end{equation*}
	pour $\cdot^{*}$ la transformation de Legendre-Fenchel.
	Puisqu'on sait que $(-f)^{*}$ est convexe, $f^{c}$ est concave.

	Si $\pi$ est optimal pour $L_{c}$ et $f$ est optimal pour le dual, alors:
	\begin{equation*}
		\mathrm{supp}(\pi) \subseteq \left\{(x, y) \suchthat f^{cc}(x) + f^{c}(y) = -\scalar{x, y}\right\}
	\end{equation*}
	En notant $\phi = -f^{cc}$, $\phi$ est convexe et donc:
	\begin{equation*}
		\phi^{*}(y) = \sup_{x} \scalar{x, y} - \phi(x) = -f^{c}(y)
	\end{equation*}
	On a donc:
	\begin{equation*}
		\mathrm{supp}(\pi) \subseteq \underbrace{\left\{(x, y) \suchthat \phi(x) + \phi^{*}(y) = \scalar{x, y}\right\}}_{\text{Sous-différentielle } \partial\phi(x)}
	\end{equation*}
	Si $\phi$ est différentiable:
	\begin{equation*}
		\mathrm{supp}(\pi) \subseteq \{(x, \nabla\phi(x))\}
	\end{equation*}
	Puisque $\phi$ est convexe, $\phi$ est différentiable presque partout pour la mesure de Lebesgue,
	donc si $\alpha$ est absolument continue par rapport à la mesure de Lebesgue, $\phi$ est aussi différentiable presque partout pour $\alpha$.
\end{proof}

\subsubsection{Cas Semi-discret}
On suppose ici $\alpha$ absolument continue et $\beta = \sum_{j = 1}^{m} b_{j} \delta_{y_{j}}$.
On a:
\begin{align*}
	L_{c}(\alpha, \beta) = & \max_{f, g \in \mC(\X) \times \mC(\Y), f\oplus g \leq c} \int f \d\alpha + \int g \d\beta \\
	=                      & \max_{f \oplus g \leq c} \int f\d \alpha + \sum_{j = 1}^{m}b_{j}g(y_{j})
\end{align*}
Avec $\phi_{v}(x) = \min_{j} c(x, y_{j}) - v_{j}$, on peut montrer que:
\begin{align*}
	L_{c}(\alpha, \beta) = & \max_{v \in (\R^{d})^{m}} \int \phi_{v} \d\alpha + \sum_{j= 1}^{m}b_{j}v_{j}                            \\
	=                      & \max_{v} \int \min_{j} \left(c\left(x, y_{j}\right)-v_{j}\right) \d \alpha + \sum_{j = 1}^{m}b_{j}v_{j}
\end{align*}
Si on considère les cellules de Laguerre:
\begin{equation*}
	L_{j}(v) = \left\{x \in \X \suchthat \forall j'\neq j, c(x, y_{j}) -v_{j} \leq c(x, y_{j'}) - v_{j'}\right\}
\end{equation*}
On obtient:
\begin{equation*}
	L_{c}(\alpha, \beta) = \max_{v} \sum_{j} \int_{L_{j}(v)} \left(c(x, y_{j}) - v_{j}\right) \d\alpha + \scalar{b, v}
\end{equation*}

\subsubsection{Distance $1$-Wasserstein}
On se place dans le cas $c(x, y) = d(x, y)$ sur $\X = \Y$.

\begin{proposition}
	\begin{enumerate}
		\item $f$ est $c$- concave si et seulement si $f$ est $\delta$-lipschitzienne pour $\delta \leq 1$.
		\item Is $\mathrm{Lip}(f) \leq 1$, alors $f^{c} = -f0$
	\end{enumerate}
\end{proposition}

\begin{proposition}
	Sous les hypothèses ci-dessus:
	\begin{equation*}
		\Wass_{1}(\alpha, \beta) = \max_{\phi, \mathrm{Lip}(\phi) \leq 1} \int \phi\d(\alpha - \beta)
	\end{equation*}
\end{proposition}

\begin{remarque}
	Dans le cas discret, $\alpha - \beta = \sum m_{k}\delta_{z_{k}}$ avec $\sum m_{k} = 0$.
	On a donc:
	\begin{equation*}
		\Wass_{1}(\alpha, \beta) = \max_{u_{k}} \left\{\sum_{k} u_{k}m_{k} \suchthat \forall k, l, \abs{u_{k} - u_{l}} \leq d(z_{k}, z_{l})\right\}
	\end{equation*}
\end{remarque}

\begin{remarque}
	Si on est aussi dans le cas euclidien, la condition lipschitzienne globale peut se remplacer par:
	\begin{equation*}
		\ninf{\nabla \phi} \leq 1
	\end{equation*}
\end{remarque}


\section{Transport Optimal Tranché}

\subsection{Définition et propriétés}

\begin{definition}
	Given $u \in \R^{d}$, l'opérateur de tranchage (ou \textit{slicing}) basé sur $u$ est défini par:
	\begin{equation*}
		p_{u}: x \in \R^{d} = \scalar{x, u}
	\end{equation*}
\end{definition}

On va généraliser cette opération sur les mesures grâce à l'opérateur de poussage $\sharp$:
Si $X \sim \mu$, alors $\scalar{u, X} \sim p_{u}\sharp \mu$.

Dans la suite on note $P_{u}$ l'application de projection orthogonale sur $u$.

\begin{definition}
	La distance de Wasserstein tranchée par $c$ $W_{c}$ entre $\mu$ et $\nu$ est la distance $1$-dimensionnelle de Wasserstein basée sur $c :\R \times \R \to \R_{+}$ continue est:
	\begin{equation*}
		S\Wass_{p}(\mu, \nu) = \E_{\theta \sim \mU_{\S^{d - 1}}}[\Wass(p_{\theta}\sharp \mu, p_{\theta}\sharp\nu)]
	\end{equation*}
\end{definition}
\begin{proof}
	C'est bien une distance, tkt.
\end{proof}

\begin{proposition}
	Pour toutes deux distributions $\mu, \nu$ sur $\R^{d}$:
	\begin{equation*}
		S\Wass_{p}^{p}(\mu, \nu) \leq C_{d, p}\Wass_{p}^{p}(\mu, \nu)
	\end{equation*}
	où:
	\begin{equation*}
		C_{d, p} = \frac{1}{d}\int_{\S^{d - 1}}\norm{\theta}_{p}^{p}\d\mU_{\S^{d - 1}}(\theta)
	\end{equation*}
\end{proposition}

\subsection{En pratique}
On a l'approximation de Monte-Carlo suivante (puisque l'espérance sur la sphère n'est pas calculable en général):
\begin{equation*}
	S\Wass_{p}^{p}(\mu, \nu) \simeq \frac{1}{K}\sum_{j = 1}^{K} W_{p}(p_{\theta_{j}}\sharp\mu, p_{\theta_{j}}\sharp\nu)^{p}
\end{equation*}



\section{Modèles de Flot et Diffusion}
Dans cette section, on s'intéresse à l'espace $\mathbb{W}_{p}$ des mesures de probabilités sur $\R^{d}$ muni de la distance de $p$-Wasserstein.

\subsection{Transport optimal dynamique}
Dans la suite, $c(x, y) = \norm{x - y}^{2}$.
On s'intéresse d'abord aux géodésiques sur cet espaces, et à ce qu'elles nous apprennent en apprentissage.

Si on se donne $\alpha, \beta$ deux mesures de probabilités avec $\alpha$ à densité, et $T$ une application optimale entre $\alpha$ et $\beta$, on définit $\mu_{t}$ la distribution de $X_{t} = (1 - t) X_{0} + tT(X_{0})$ avec $X_{0}\sim \alpha$.
L'application $t \mapsto \mu_{t}$ est donc une géodésique dans $\WW_{2}$.
\begin{definition}
	On définit $v_{t}(x) = (T - \Id)\circ T_{t}^{-1}(x)$, de sorte que $v_{t} \circ T_{t}(x) = T(x) - x$ est une constante.
	C'est le champ de vitesse constant sur les chemins de $x$ à $T(x)$.
\end{definition}

On a alors:
\begin{equation*}
	\int_{0}^{1}\int \norm{v_{t}(x)}^{2}\d\mu_{t}\d t = \int_{0}^{1}\int \norm{v_{t}\circ T_{t}(x)}^{2}\d\alpha\d t = \int_{0}^{1}\int \norm{T(x) - x}^{2}\d\alpha \d t = \int \norm{T(x) - x}^{2} \d\alpha = \Wass_{2}(\alpha, \beta)
\end{equation*}

\begin{definition}
	L'ensemble des chemins avec vitesses entre $\alpha$ et $\beta$ deux mesures de probabilités est:
	\begin{equation*}
		V(\alpha, \beta) = \left\{\left(\mu_{t}, v_{t}\right) \suchthat \mu_{0} = \alpha, \mu_{1} = \beta\right\}
	\end{equation*}
	où, pour tout $t$, $\mu_{t} \in \mP(\R^{d})$ et où $v: [0, 1] \times \R^{d} \to \R^{d}$ est un champ de vitesses et où la paire $(\mu_{t}, v_{t})$ vérifie l'équation de continuité \eqref{eq:CE}.
\end{definition}

\begin{proposition}
	Si $T = \nabla \phi$ avec $\phi$ convexe, alors $T$ est monotone et pour tout $t$, $T_{t}$ est injective.
\end{proposition}

\begin{definition}
	La paire $(\mu_{t}, v_{t})$ satisfie l'équation de continuité si:
	\begin{equation}
		\frac{\d\mu_{t}}{\d t} + \div (\mu_{t}v_{t}) = 0
		\tag{CE}\label{eq:CE}
	\end{equation}
	où l'égalité est entendue au sens des distributions.
\end{definition}

Il reste donc à vérifier que $\mu_{t} = T_{t}\sharp \alpha$ et $v_{t} = (T - \Id) \circ T_{t}^{-1}(x)$ vérifient l'équation de continuité.
On se donne $\psi$ a support compact sur $]0, 1[\times \R^{d}$ et on a:
\begin{align*}
	\int_{0}^{1}\int \frac{\d\psi}{\d t}\d\mu_{t}\d t =                       & \int_{0}^{1}\int \frac{\d \psi_{t}}{\d t}(T_{t}(x))\d\alpha\d t                       \\
	\int_{0}^{1}\int_{\R^{d}}\scalar{\nabla_{x}\psi, v_{t}(x)}\d\mu_{t}\d t = & \int_{0}^{1}\int_{\R^{d}}\scalar{\nabla_{x}\psi \circ T_{t}(x), T(x) - x}\d\alpha\d t
\end{align*}
Ainsi:
\begin{equation*}
	\int_{0}^{1}\int_{\R^{d}}\left(\frac{\d\psi_{t}}{\d t} \circ T_{t}(x) + \scalar{\nabla_{x}\psi, v_{t}(x) }\right)\d\mu_{t}\d t =
	\int_{0}^{1}\int_{\R^{d}} \underbrace{\left(\frac{\d\psi_{t}}{\d t} \circ T_{t}(x) + \scalar{\nabla_{x}\psi \circ T_{t}(x), T(x) - x}\right)}_{=\frac{\d}{\d t}(\psi(t, T(t, x)))}\d\alpha \d t
\end{equation*}
Finalement:
\begin{equation*}
	\int_{0}^{1}\int_{\R^{d}}\left(\frac{\d\psi_{t}}{\d t} \circ T_{t}(x) + \scalar{\nabla_{x}\psi, v_{t}(x) }\right)\d\mu_{t}\d t =
	\int_{\R^{d}} \underbrace{\left(\psi(1, T(1, x)) - \psi(0, T(0, x))\right)}_{= 0}\d\alpha
\end{equation*}

On vérifie donc bien que:
\begin{proposition}
	\label{prop:geodesicswass}
	\begin{equation*}
		W_{2}^{2}(\alpha, \beta) \geq \inf_{V(\alpha, \beta)} \int_{0}^{1}\int \norm{v_{t}(x)}^{2}\d\mu_{t}\d t
	\end{equation*}
\end{proposition}

On rappelle que le flot $\phi$ de $v$ est défini par $\frac{\d\phi_{t}(x)}{\d t} = v(t, \phi_{t}(x))$ et $\phi_{0}(x) = x$.

Avec la définition de $v_{t}$ ci-dessus, on voit que $T_{t}$ est le flot de $v_{t}$.

\begin{thm}
	\label{thm:flot}
	Si $v_{t}$ est uniformément borné et Lipschitz en $x$ (uniformément en $t$),
	si $\phi_{t}$ est son flot et si de plus $\alpha$ est absoluement continue par rapport à la
	mesure de Lebesgue, alors, $\rho_{t} = \phi_{t} \sharp \alpha$ est l'unique solution
	de \eqref{eq:CE} avec la confition initiale $\rho_{0} = \alpha$.
\end{thm}

On obtient donc le Théorème suivant:
\begin{thm}[Bénamou-Brenier]
	On a:
	\begin{equation*}
		W_{2}^{2}(\alpha, \beta) = \min_{V(\alpha, \beta)} \int_{0}^{1}\int \norm{v_{t}}^{2}\d\mu_{t}\d t
	\end{equation*}
	la solution étant unique et donnée par l'interpolation de McCann et sa vitesse correspondante $v_{t}$.
\end{thm}
\begin{proof}
	Utilisant l'équation de la Proposition \ref{prop:geodesicswass} on a déjà une inégalité.
	En utilisant le Théorème \ref{thm:flot} ci-dessus, on obtient:
	\begin{align*}
		\int_{0}^{1}\int_{\R^{d}}\norm{v_{t}(x)}^{2}\d\mu_{t}\d t
		=                                             & \int_{0}^{1}\int_{\R^{d}}\norm{v_{t}\circ \phi_{t}(x)}^{2} \d\alpha \d t                                        \\
		\overset{Fubini}{=}                           & \int_{\R^{d}}\int_{0}^{1}\norm{v_{t}\circ \phi_{t}(x)}^{2} \d t \d \alpha                                       \\
		\overset{Jensen}{\geq}                        & \int_{\R^{d}}\norm{\int_{0}^{1}\underbrace{v_{t}\circ\phi_{t}(x)}_{\frac{\d\phi_{t}}{\d t}(x)}\d t}^{2}\d\alpha \\
		=                                             & \int_{\R^{d}} \norm{\phi(1, x) - x}^{2}\d\alpha                                                                 \\
		\overset{\phi_{1}\sharp \alpha = \beta}{\geq} & W_{2}^{2}(\alpha, \beta)
	\end{align*}
	Pour l'unicité de la solution, on se donne deux solutions et on introduit leur moyenne, l'inégalité de Cauchy-Schwarz (et son cas d'égalité) nous permet de conclure la preuve.
\end{proof}

\subsection{Couplage par flot}
On va appliquer le Théorème \ref{thm:flot} et construire une paire $(\mu_{t}, v_{t}) \sim \eqref{eq:CE}$ telle que $\mu_{0} = \alpha$, $\mu_{1} = \beta$.
Le couplage par flot est une manière de construire un poussé en avant entre $\alpha$ et $\beta$, mais celui ci ne sera pas nécessairement optimal.

C'est une méthode applicable à la modélisation générative:
\begin{enumerate}
	\item On échantillonne $\alpha = \mN(0, \Id)$.
	\item On calcule un champ $v_{\theta}$ tel que $\mu_{t}, v_{\theta} \sim \eqref{eq:CE}$ où $\mu_{t}$ est un chemin de $\alpha$ à $\beta$.
	\item On intègre $v_{\theta}$ pour optenir $\phi^{\theta}$.
	\item On échantillone $\beta$ en échantillonnant $\alpha$ puis en considérant l'image par $\phi_{1}^{\theta}$.
\end{enumerate}

Pour définir $\mu_{t}, v_{t}$ de manière efficace on a l'algorithme suivant:
\begin{enumerate}
	\item On part d'un couplage $X_{0}, X_{1} \sim \Pi \in \mU(\alpha, \beta)$.
	\item On définit $X_{t} = tX_{1} + (1 - t)X_{0}$ et $\mu_{t}$ la loi de $X_{t}$.
	\item On calcule ensuite $v_{t}$ comme l'espérance de $X_{1} - X_{0}$ sachant $X_{t} = x$.
\end{enumerate}

\begin{proposition}
	La paire $(\mu_{t}, v_{t})$ définie ci-dessus vérifie \eqref{eq:CE}, et donc $(\mu_{t}, v_{t}) \in V(\alpha, \beta)$.
\end{proposition}

\section{Barycentres et Lois Multimarginales}

\section{Barycentre de Sinkhorn}
\subsection{Régularisation entropique}
Considérons deux mesures discrètes $\alpha = \sum_{i=1}^{n} a_i\delta_{x_{i}}$ et $\beta = \sum_{j=1}^{m} b_j\delta_{y_{j}}$.
\begin{definition}
	Le problème de Schrödinger statique s'écrit:
	\begin{equation}
		\tag{Schr}\label{eq:schrodinger}
		\min_{P\in\R_+^{n\times m}, P\mathbf{1}_m = a, P^\top \mathbf{1}_n = b} \scalar{C, P} + \epsilon H(P)
	\end{equation}
	où $H(P) = \sum_{i,j} P_{i,j}\log P_{i,j}$ est l'entropie de Shannon négative de la matrice de
	transport $P$, avec la convention $0\log 0 = 0$.
	Le paramètre $\epsilon$ est appelé \emph{paramètre de régularisation entropique},
	ou \emph{température} en physique statistique.
\end{definition}

\begin{remarque}
	$H$ est strictement convexe, puisqu'on se trouve dans le simplexe des matrices à
	coefficients compris entre $0$ et $1$.
\end{remarque}

\begin{proposition}
	Si $\epsilon > 0$, le problème de Schrödinger admet une unique solution $P_{\epsilon}$.
	Si $\epsilon = 0$, on retrouve la formulation de Kantorovich, qui peut admettre plusieurs solutions.
\end{proposition}

\begin{proposition}
	Lorsque $\epsilon \rightarrow 0$, la solution du problème de Schrödinger converge
	vers la solution de Kantorovich qui maximise l'entropie:
	\begin{equation*}
		P_\epsilon \xrightarrow[\epsilon \rightarrow 0]{} \argmin_P \left\{ H(P) : P \text{ est une solution de Kantorovich} \right\}
	\end{equation*}

	À l'inverse, lorsque $\epsilon \rightarrow +\infty$, on retrouve le couplage trivial:
	\begin{equation*}
		P_\epsilon \xrightarrow[\epsilon \rightarrow +\infty]{} a \otimes b
	\end{equation*}
\end{proposition}

L'intuition derrière cette régularisation entropique est l'ajout d'une fonction de barrière, de manière similaire aux méthodes de points intérieurs en optimisation convexe. Une différence importante étant que l'on utilise ici l'entropie de Shannon négative, et non pas une fonction logarithmique classique. Cela permet de garantir la positivité des coefficients de la matrice de transport $P$.

\begin{thm}
	Supposons sans perte de généralité que $a_i>0$ pour tout $i$ et $b_j>0$ pour tout $j$. On a alors:
	\begin{equation*}
		P \text{ est solution du problème de Schrödinger } \Longleftrightarrow\begin{cases}
			P\textbf{1}_m = a, P^\top \mathbf{1}_n = b \\
			\exists u \in \R^n, v \in \R^m, P_{i,j} = u_iK_{i,j}v_j \text{ avec } K_{i,j} = e^{-C_{i,j}/\epsilon}
		\end{cases}
	\end{equation*}
\end{thm}
De manière équivalente, un couplage $P$ est solution s'il existe des vecteurs $u \in \R^n$ et $v \in \R^m$ tels que:
\begin{equation*}
	P = \diag(u) K \diag(v)
\end{equation*}
où $K$ est la matrice de noyau exponentiel défini par $K_{i,j} = e^{-C_{i,j}/\epsilon}$.

\begin{proof}
	On veut résoudre le problème de Schrödinger:
	\begin{equation*}
		\min_{P\mathbf{1}_m = a, P^\top \mathbf{1}_n = b} f(P) := \scalar{C, P} + \epsilon H(P)
	\end{equation*}
	Commençons par montrer par l'absurde que si $P^\star$ est solution, alors $P^\star_{i,j} > 0$ pour tout $(i,j)$. Supposons qu'il existe $(i_0, j_0)$ tel que $P^\star_{i_0, j_0} = 0$. Posons $P^t(1-t)P^\star + t (a\otimes b)$ pour $t \in [0,1]$. Alors $P^t$ est admissible pour tout $t$. De plus, si l'on pose $g(t) = f(P^t)$, alors $g'(0) = -\infty$ car la dérivée de l'entropie en $0$ est infinie. Donc pour $t$ suffisamment petit, $f(P^t) < f(P^\star)$, ce qui contredit le fait que $P^\star$ est solution. Ainsi, $P^\star_{i,j} > 0$ pour tout $(i,j)$. Ceci justifie (peu rigoureusement) le fait que l'on peut omettre la contrainte de positivité dans la suite de la preuve.

	Dérivons le problème dual de Lagrange. On introduit les multiplicateurs de Lagrange $\{f_i\}_{i=1}^n$ et $\{g_j\}_{j=1}^m$ associés aux contraintes de marges. Le lagrangien s'écrit:
	\begin{equation*}
		L(P, f, g) = \scalar{C, P} + \epsilon H(P) + \scalar{a-P\mathbf{1}_m, f} + \scalar{b - P^\top \mathbf{1}_n, g}
	\end{equation*}
	(Ici, les conditions de Slater sont vérifiées car le problème est convexe et admet une solution intérieure, on peut donc échanger le $\min$ et le $\max$.)

	Remarquons que $\scalar{a-P\mathbf{1}_m, f} = \scalar{a, f} - \scalar{P\mathbf{1}_m, f} = \scalar{a, f} - \scalar{P, f \mathbf{1}_m^\top}$. On a alors:
	\begin{equation*}
		\nabla_P L = C + \epsilon (\log P + 1) - f \mathbf{1}_m^\top - \mathbf{1}_n g^\top.
	\end{equation*}
	En annulant ce gradient, on obtient:
	\begin{equation*}
		P_{i,j} = e^{(f_i + g_j - C_{i,j})/\epsilon - 1} = e^{-1} e^{f_i/\epsilon} e^{-C_{i,j}/\epsilon} e^{g_j/\epsilon}.
	\end{equation*}
	En posant $u_i = e^{f_i/\epsilon - 1/2}$ et $v_j = e^{g_j/\epsilon - 1/2}$, on retrouve bien la forme annoncée.
\end{proof}

\subsection{Algorithme de Sinkhorn}
On note le produit de Kronecker $\diag(u) z = u \odot z = (u_i z_i)_i$ le produit terme à terme entre les vecteurs $u$ et $z$. On a donc $P\mathbf{1}_m = u\odot (K v)$ qui doit valoir $a$, et de même $P^\top \mathbf{1}_n = v \odot (K^\top u) = b$. On a ainsi le système suivant:
\begin{equation}
	\tag{SchrSys}\label{eq:sinkhornsystem}
	\begin{cases}
		u \odot (K v) = a \\
		v \odot (K^\top u) = b
	\end{cases}
\end{equation}

\begin{thm}
	Le problème de Schrödinger \eqref{eq:schrodinger} est équivalent à la résolution du système \eqref{eq:sinkhornsystem}.
\end{thm}

\begin{proposition}
	L'algorithme de Sinkhorn pour la résolution du problème de Schrödinger est le suivant:
	\begin{itemize}
		\item $v \leftarrow \mathbf{1}_m$
		\item Répéter jusqu'à convergence:
		      \begin{itemize}
			      \item $u \leftarrow a / (K v)$
			      \item $v \leftarrow b / (K^\top u)$
		      \end{itemize}
	\end{itemize}
	où l'on note $z/w$ le quotient terme à terme entre les vecteurs $z$ et $w$.
\end{proposition}
Cet algorithme est très simple et facilement parallélisable. Chaque itération coûte $\O(n^2)$ opérations, donnant une complexité totale de $\O(Tn^2)$ pour atteindre une précision $\epsilon$ en $T$ étapes; ceci est à comparer à des algorithmes comme celui du simplexe, qui a une complexité cubique.

\begin{thm}
	Il suffit de $T=\frac{1}{\epsilon^2}$ itérations pour atteindre une précision $\epsilon$.
\end{thm}
\begin{remarque}
	On a donc une complexité totale de $\O\left(\frac{n^2}{\epsilon^2}\right)$ pour atteindre une précision $\epsilon$. Par rapport aux méthodes à points intérieurs de complexité $\O(n^3\log(\epsilon))$, la méthode de Sinkhorn est donc plus efficace en termes d'échantillons $n$, mais moins efficace en termes de précision $\epsilon$.
\end{remarque}

\begin{proof}
	S'intéresse à la preuve par contraction qui donne une convergence linéaire, contrairement à une preuve par projection itérative qui donnerait une convergence sous-linéaire, la constante étant meilleure dans le cas linéaire. \emph{Voir les notes de cours pour la preuve.}
\end{proof}

\subsection{Reformulation en divergence de Kullback-Leibler}
\begin{definition}
	La divergence de Kullback-Leibler entre deux matrices $P, Q \in \R_+^{n\times m}$ est définie par:
	\begin{equation*}
		\KL(P|Q) = \sum_{i,j} P_{i,j} \log\left(\frac{P_{i,j}}{Q_{i,j}}\right) - P_{i,j} + Q_{i,j}
	\end{equation*}
\end{definition}

\begin{proposition}
	$\KL(P|Q) \geq 0$ avec égalité si et seulement si $P = Q$.
\end{proposition}

L'idée va être d'utiliser $Q=a\otimes b$ comme mesure de référence.

\begin{proposition}
	Le problème de Schrödinger \eqref{eq:schrodinger} s'écrit de manière équivalente:
	\begin{equation*}
		\min_{P\in\R_+^{n\times m}, P\mathbf{1}_m = a, P^\top \mathbf{1}_n = b} \scalar{C, P} + \epsilon \KL(P|a\otimes b).
	\end{equation*}
\end{proposition}

\begin{proof}
	$\KL(P|a\otimes b)$ et $\KL(P|a'\otimes b')$ diffèrent d'une constante additive indépendante de $P$, donc le minimum est le même.
\end{proof}

\begin{definition}
	Soit $\frac{\pi}{\xi}\in\mP(X\times Y)$. Si $\frac{\d\pi}{\d\xi}$ existe, on définit la divergence de Kullback-Leibler entre $\pi$ et $\xi$ par:
	\begin{equation*}
		\KL(\pi|\xi) = \int_{X\times Y} \log\left(\frac{\d\pi}{\d\xi}\right) \d\pi - \pi(X\times Y) + \xi(X\times Y)
	\end{equation*}
	Si $\frac{\d\pi}{\d\xi}$ n'existe pas, on pose $\KL(\pi|\xi) = +\infty$.
\end{definition}

\begin{definition}
	Le \textbf{problème de Schrödinger général} s'écrit:
	\begin{equation}
		\tag{SchrGen}\label{eq:schrodinger-general}
		\inf_{\pi\in\mP(X\times Y)}\left\{ \int c\,\d\pi + \epsilon\,\KL(\pi|a\otimes b) : \pi_1 = \alpha,\, \pi_2 = \beta \right\}
	\end{equation}
\end{definition}
\begin{remarque}
	L'information mutuelle entre deux variables aléatoires $X$ et $Y$ de loi jointe $\pi$ est donnée par $I(X, Y) = \KL(\pi|\pi_1 \otimes \pi_2)$. Ainsi, le problème de Schrödinger cherche un couplage $\pi$ qui minimise le coût total plus une pénalisation de l'information mutuelle entre les deux variables aléatoires. Lorsque $I(X,Y) = 0$, les variables sont indépendantes, et le couplage est donc le produit tensoriel des marges.
\end{remarque}

Calculons le problème dual de Schrödinger dans ce cadre général. En réutilisant les notations des multiplicateurs de Lagrange, on a:
\begin{equation*}
	\min_P\max_{f,g} \scalar{C, P} + \epsilon \KL(P|a\otimes b) + \scalar{\alpha - P\mathbf{1}, f} + \scalar{\beta - P^\top\mathbf{1}, g}
\end{equation*}
Puisque les conditions de Slater sont vérifiées, on peut échanger le $\min$ et le $\max$:
\begin{equation*}
	\max_{f,g} \scalar{a, f} + \scalar{b, g} + \min_P \scalar{C-f\mathbf{1}^\top - \mathbf{1}g^\top, P} + \epsilon \KL(P|a\otimes b),
\end{equation*}
ce qui est équivalent à:
\begin{equation*}
	\max_{f,g} \scalar{a, f} + \scalar{b, g} - \max_P \scalar{C-f\mathbf{1}^\top - \mathbf{1}g^\top, P} - \epsilon \KL(P|a\otimes b).
\end{equation*}

On remarque alors que ceci correspond à la transformation de Legendre-Fenchel de $\KL(\cdot|a\otimes b)$ évaluée en $(f\mathbf{1}^\top + \mathbf{1}g^\top - C)/\epsilon$. On a donc:
\begin{equation*}
	\max_{f,g} \scalar{a, f} + \scalar{b, g} -\epsilon\KL^\star(z|ab^\top)
\end{equation*}
où $z = (f\mathbf{1}^\top + \mathbf{1}g^\top - C)/\epsilon$.

\begin{lemme}
	La transformation de Legendre-Fenchel de $\KL(\cdot|a\otimes b)$ est donnée par:
	\begin{equation*}
		\KL^\star(Z|Q) = \sum_{i,j} Q_{i,j}\exp(Z_{i,j}) - 1
	\end{equation*}
\end{lemme}

On en déduit le dual de Schrödinger:
\begin{equation*}
	\max_{f,g} \scalar{a, f} + \scalar{b, g} - \epsilon \sum_{i,j} a_i b_j \exp\left(\frac{f_i + g_j - C_{i,j}}{\epsilon}\right) + \epsilon
\end{equation*}

Dans le cas général, pour des mesures non-discrètes, on retrouve une formulation similaire:
\begin{equation*}
	\inf_{f\in\mC(X), g\in\mC(Y)} \int f(x)\d\alpha(x) + \int g(y)\d\beta(y) - \epsilon \int e^{(f(x) + g(y) - c(x,y))/\epsilon} \d\alpha(x)\d\beta(y) + \epsilon
\end{equation*}


\begin{proposition}
	L'\textbf{algorithme de Sinkhorn} dans ce cadre s'écrit alors:
	\begin{itemize}
		\item Initialiser $g$
		\item Répéter jusqu'à convergence:
		      \begin{itemize}
			      \item $f \leftarrow g^{c,\epsilon} := \argmin_f D(f, g)$
			      \item $g \leftarrow f^{c,\epsilon} := \argmin_g D(f, g)$
		      \end{itemize}
	\end{itemize}
	où l'on dénote $(\cdot)^{c,\epsilon}$ l'opérateur de c-transformée réguliarisée par l'entropie, et $D$ la fonctionnelle duale de Schrödinger.
\end{proposition}

\begin{proposition}
	L'\textbf{opérateur de c-transformée réguliarisée par l'entropie} s'écrit:
	\begin{equation*}
		f^{c,\epsilon}(y) = -\epsilon \log\left( \int e^{(f(x) - c(x,y))/\epsilon} \d\alpha(x) \right)
	\end{equation*}
	et de manière similaire:
	\begin{equation*}
		g^{c,\epsilon}(x) = -\epsilon \log\left( \int e^{(g(y) - c(x,y))/\epsilon} \d\beta(y) \right)
	\end{equation*}
\end{proposition}


\end{document}
