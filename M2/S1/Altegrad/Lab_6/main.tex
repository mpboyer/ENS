\documentclass[a4paper]{article} 
\input{style/head.tex}

%-------------------------------
%	TITLE VARIABLES (identify your work!)
%-------------------------------

\newcommand{\yourname}{Matthieu Boyer} % replace YOURNAME with your name
\newcommand{\youremail}{matthieu.boyer@ens.fr} % replace YOUREMAIL with your email
\newcommand{\assignmentnumber}{6} % replace X with the lab session number

\begin{document}

\fancyhead[C]{}
\hrule \medskip
\begin{minipage}{0.295\textwidth} 
\raggedright
\footnotesize
\yourname \hfill\\
\youremail
\end{minipage}
\begin{minipage}{0.4\textwidth} 
\centering 
\large 
Lab session \# \assignmentnumber\\ 
\normalsize 
ALTEGRAD 2023\\ 
\end{minipage}
\begin{minipage}{0.295\textwidth} 
\raggedleft
\today\hfill\\
\end{minipage}
\medskip\hrule 
\bigskip




\section{Task 4}
\begin{center}
	\includegraphics[width=\textwidth]{code/embeddings.pdf}
\end{center}

\section{Question 1}
As linguists have the traditional maxim \textit{you shall know a work by its company}, here, given a
random walk (which can be seen as a (near-)randomly generated sequence of words), we increase the context
considered to understand the meaning (or in this case the probability) that a word (vertex) is generated.
The window size maps to the size of the context considered to understand what the meaning of a vertex is:
what the probability of it filling in the random walk is.

\section{Task 5}

\begin{center}
	\includegraphics[width=\textwidth]{code/karate.pdf}
\end{center}

\section{Task 8}

\begin{itemize}
	\item DeepWalk Embeddings (using \texttt{gensim.Word2Vec}) yield $42.86\%$ accuracy.
	\item Sklearn's Spectral Embeddings (using \texttt{sklearn.manifold.SpectralEmbedding}) yield the same $42.86\%$ accuracy.
	\item Lrw Spectral Embeddings yield $85.71\%$ accuracy.
\end{itemize}

\section{Question 2}
The two embeddings provided show an important similarity: $X_{1}$ and $X_{2}$ define the exact same space as
the second column of $X_{1}$ is the opposite of the second column of $X_{2}$.
What this means is that $X_{2} = X_{1} R$ where $R$ is the reflection around the $y$-axis $\begin{pmatrix}
		1 & 0 \\ 0 & - 1
	\end{pmatrix}$


\section{Question 3}
Since $P$ is a permutation matrix, $P_{ij} \in \{0,1\}$.
Moreover, for a permutation, each row and column has exactly one 1.
The degree of node $i$ in the permuted graph equals the degree of the node that was mapped to position $i$.
Therefore:
\[
	\tilde{D}' = P\tilde{D}P^T
\]

Now we can compute the normalized adjacency matrix $\hat{A}'$:
\begin{align*}
	\hat{A}' & = (\tilde{D}')^{-\frac{1}{2}}\tilde{A}'(\tilde{D}')^{-\frac{1}{2}}                    \\
	         & = (P\tilde{D}P^T)^{-\frac{1}{2}}(P\tilde{A}P^T)(P\tilde{D}P^T)^{-\frac{1}{2}}         \\
	         & = P\tilde{D}^{-\frac{1}{2}}P^T \cdot P\tilde{A}P^T \cdot P\tilde{D}^{-\frac{1}{2}}P^T \\
	         & = P\tilde{D}^{-\frac{1}{2}}\tilde{A}\tilde{D}^{-\frac{1}{2}}P^T                       \\
	         & = P\hat{A}P^T
\end{align*}

Finally, we compute the GNN output for the permuted inputs:
\begin{align*}
	\text{GNN}(PAP^T, PX) & = f(\hat{A}'(PX)W)                                                             \\
	                      & = f(P\hat{A}P^T \cdot PX \cdot W)                                              \\
	                      & = f(P\hat{A}(P^TP)XW)                                                          \\
	                      & = f(P\hat{A}XW) \quad \text{(since } P^TP = I\text{)}                          \\
	                      & = P \cdot f(\hat{A}XW) \quad \text{(since } f \text{ is applied element-wise)} \\
	                      & = P \cdot \text{GNN}(A, X)
\end{align*}

Therefore, the GNN layer is permutation equivariant. \qed

\section{Task 11}
Test set results: loss= 0.0002 accuracy= 1.0000
\begin{center}
	\hfill
	\includegraphics[width=.45\textwidth]{code/gnn_training_accuracy.pdf}
	\hfill
	\includegraphics[width=.45\textwidth]{code/gnn_training_loss.pdf}
	\hfill
\end{center}

\section{Task 12}

Test set results: loss= 0.8166 accuracy= 0.2857
\begin{center}
	\hfill
	\includegraphics[width=.45\textwidth]{code/gnn_training_accuracy_equal_features.pdf}
	\hfill
	\includegraphics[width=.45\textwidth]{code/gnn_training_loss_equal_features.pdf}
	\hfill
\end{center}


\section{Question 4}

We analyze a linear GCN with $k$ layers defined as:
\[
	Z^{(k)} = \hat{A}^k XW
\]
where $\hat{A} = \tilde{D}^{-\frac{1}{2}}\tilde{A}\tilde{D}^{-\frac{1}{2}}$ is the normalized adjacency matrix with self-loops, and the graph is connected and non-bipartite.

\subsection*{Part 1: Proving $u$ is an eigenvector of $\hat{A}$ with eigenvalue $\lambda = 1$}

Let $u \in \mathbb{R}^n$ be a vector where $u_i = \sqrt{\tilde{d}_i}$ (the square root of the degree of node $i$ including the self-loop).

We compute $\hat{A}u$:
\begin{align*}
	(\hat{A}u)_i & = \sum_{j=1}^n \hat{A}_{ij} u_j                                                                            \\
	             & = \sum_{j=1}^n \left(\tilde{D}^{-\frac{1}{2}}\tilde{A}\tilde{D}^{-\frac{1}{2}}\right)_{ij} u_j             \\
	             & = \sum_{j=1}^n \frac{1}{\sqrt{\tilde{d}_i}} \tilde{A}_{ij} \frac{1}{\sqrt{\tilde{d}_j}} u_j                \\
	             & = \sum_{j=1}^n \frac{1}{\sqrt{\tilde{d}_i}} \tilde{A}_{ij} \frac{1}{\sqrt{\tilde{d}_j}} \sqrt{\tilde{d}_j} \\
	             & = \sum_{j=1}^n \frac{1}{\sqrt{\tilde{d}_i}} \tilde{A}_{ij}                                                 \\
	             & = \frac{1}{\sqrt{\tilde{d}_i}} \sum_{j=1}^n \tilde{A}_{ij}                                                 \\
	             & = \frac{1}{\sqrt{\tilde{d}_i}} \cdot \tilde{d}_i \quad \text{(by definition of degree)}                    \\
	             & = \sqrt{\tilde{d}_i}                                                                                       \\
	             & = u_i
\end{align*}

Therefore, $\hat{A}u = u$, which proves that $u$ is an eigenvector of $\hat{A}$ corresponding to eigenvalue $\lambda = 1$. \qed

\subsection*{Part 2: Deriving the limit of $Z^{(k)}$ as $k \to \infty$}

Since $\hat{A}$ is symmetric, it has a spectral decomposition:
\[
	\hat{A} = \sum_{i=1}^n \lambda_i v_i v_i^T
\]
where $\lambda_1 = 1 > |\lambda_2| \geq |\lambda_3| \geq \cdots \geq |\lambda_n|$ and $\{v_1, v_2, \ldots, v_n\}$ are orthonormal eigenvectors.

From Part 1, we know that $v_1 = \frac{u}{\|u\|} = \frac{u}{\sqrt{\sum_i \tilde{d}_i}}$ is the eigenvector corresponding to $\lambda_1 = 1$.

Taking powers of $\hat{A}$:
\begin{align*}
	\hat{A}^k & = \sum_{i=1}^n \lambda_i^k v_i v_i^T                         \\
	          & = \lambda_1^k v_1 v_1^T + \sum_{i=2}^n \lambda_i^k v_i v_i^T \\
	          & = v_1 v_1^T + \sum_{i=2}^n \lambda_i^k v_i v_i^T
\end{align*}

Since $|\lambda_i| < 1$ for $i \geq 2$, we have $\lambda_i^k \to 0$ as $k \to \infty$. Therefore:
\[
	\lim_{k \to \infty} \hat{A}^k = v_1 v_1^T
\]

The node representations become:
\begin{align*}
	\lim_{k \to \infty} Z^{(k)} & = \lim_{k \to \infty} \hat{A}^k XW \\
	                            & = v_1 v_1^T XW                     \\
	                            & = v_1 (v_1^T X) W
\end{align*}

Note that $v_1^T X$ is a row vector (representing a global aggregation of features weighted by normalized degrees), and the result $v_1 (v_1^T X) W$ assigns the same linear combination to every node, scaled by the normalized degree component $v_{1,i} = \frac{\sqrt{\tilde{d}_i}}{\sqrt{\sum_j \tilde{d}_j}}$.

More explicitly, the $i$-th row of $Z^{(k)}$ converges to:
\[
	Z^{(k)}_i \to \frac{\sqrt{\tilde{d}_i}}{\sqrt{\sum_j \tilde{d}_j}} \cdot (v_1^T X) W
\]

This means all node representations become proportional to $v_1$, differing only by their degree-dependent scaling factor.

\subsection*{Part 3: Explanation of failure to distinguish nodes}

Nodes with the same degree have identical entries in $v_1$, so their representations converge to exactly the same vector regardless of their initial features $X$, causing complete oversmoothing.
\section{Task 13}
Test set results: loss= 0.6081 accuracy= 0.8376

\begin{center}
	\includegraphics[width=\textwidth]{code/tsne-cora.pdf}
\end{center}

\end{document}
