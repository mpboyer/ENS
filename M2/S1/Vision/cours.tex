\documentclass[info, math]{mpb-cours}

\title{Vision Artificielle 3D}
\author{D'après Pascal Monasse, Loic Landrieu, Aslan Artykov}

\begin{document}
\bettertitle
\url{mailto:pascal.monasse@enpc.fr}, \url{mailto:loic.landrieu@enpc.fr}, \url{mailto:arslan.artykov@enpc.fr}

\section*{Introduction}
\noindent Le but principal de la classe est de démontrer et d'expliquer l'équation suivante:
\begin{equation*}
	\arg \min_{\{R_{i}\}, \{T_{i}\}, \{X_{j}\}} \sum_{i, j} \epsilon_{i, j}d\left(x_{i, j}, \Pi_{i}(R_{i}X_{j} + T_{i})\right)^{2}
\end{equation*}
où, pour le $j$-ème point 3D, observé dans la $i$-ème image:
\begin{itemize}
	\item $\epsilon_{i, j} \in \{0, 1\}$ est la visibilité;
	\item $x_{i, j}$ est le point 2D dans l'image;
	\item $X_{i, j}$ est le point 3D;
	\item $\Pi_{i}$ est la fonction de projection de l'image $i$;
	\item $R_{i}$ est la rotation de caméra de l'image $i$;
	\item $T_{i}$ est la translation de caméra de l'image $i$.
\end{itemize}

\section{Géométrie Projective}
\subsection{Géométrie Projective}
On s'adaptera au modèle de caméra dit \emph{sténopé}.
On suppose ainsi que la caméra est un objectif idéal modélisé par un point.
Ceci ne permet pas de prendre en compte le flou ni la distortion géométrique de la lentille.

Dans ce modèle, les rayons qui viennent du point $m$ sur l'image sont les mêmes que ceux qui viennent du point réel $M$: $\vec{Cm} = \lambda\vec{CM}$.
Dans le repère de coordonnées de la caméra CXYZ:
\begin{equation*}
	\begin{pmatrix} x\\ y \\f\end{pmatrix} = \lambda \begin{pmatrix} X \\ Y \\ Z \end{pmatrix}
\end{equation*}
où $f$ est la distance focale de la caméra.
On a $\lambda = f / Z$ et donc:
\begin{equation*}
	\begin{pmatrix} x \\ y\end{pmatrix} = f\begin{pmatrix} X / Z \\ Y / Z\end{pmatrix}
\end{equation*}
En coordonnées de pixels:
\begin{equation*}
	\begin{pmatrix}u \\ v\end{pmatrix} = \begin{pmatrix} \alpha x + c_{x}  \\ \alpha y + c_{y}\end{pmatrix}
\end{equation*}
$\alpha f$ est la distance focale en pixel et $c_{x}, c_{y}$ est la position du point principal en pixel.

\begin{definition}
	L'espace projectif $\mathbb{P}^{2}$ est défini comme le quotient de $\R^{3} \setminus 0$ par la relation de colinéarité.
\end{definition}
C'est donc l'ensemble des rayons passant par l'origine.

Il y a deux types de points dans $\P^{2}$:
\begin{itemize}
	\item Ceux de la forme $\begin{pmatrix} x & y & 1\end{pmatrix}$
	\item Ceux de la forme $\begin{pmatrix} x & y & 0\end{pmatrix}$ (qui sont les points à l'infini dans la direction de pente $y / x$).
\end{itemize}

Le plan de $\R^{3}$ contenant $0$ d'équation $aX + bY + cZ$ correspond à une ligne de $\P^{2}$ défini en coordonnées homogènes par $(a, b, c)$ d'équation:
\begin{equation*}
	\begin{pmatrix} a & b & c\end{pmatrix} \cdot \begin{pmatrix} X & Y & Z\end{pmatrix} = 0
\end{equation*}

On peut alors aussi trouver des équations pour:
\begin{itemize}
	\item la droite entre deux points $x_{1}$ et $x_{2}$: $l = x_{1} \times x_{2}$;
	\item l'intersection de deux droites $l_{1}, l_{2}$: $x = l_{1} \times l_{2}$;
	\item la droite à l'infini: $\ell_{\infty} = \begin{pmatrix} 0 \\ 0 \\ 1 \end{pmatrix}$;
	\item l'intersection de deux droites "parallèles":
	      \begin{equation*}
		      \begin{pmatrix}
			      a \\ b \\c_{1}
		      \end{pmatrix} \times
		      \begin{pmatrix}
			      a \\ b \\c_{2}
		      \end{pmatrix} = (c_{2} - c_{1})
		      \begin{pmatrix}
			      b \\ -a \\ 0
		      \end{pmatrix} \in \ell_{\infty}
	      \end{equation*}
\end{itemize}

L'équation de projection devient alors:
\begin{equation*}
	\begin{pmatrix}
		u \\ v
	\end{pmatrix} = \frac{1}{Z}
	\begin{pmatrix}
		fX + c_{x}Z \\ fY + c_{y}Z
	\end{pmatrix}
\end{equation*}
Et donc:
\begin{equation*}
	Z \begin{pmatrix}
		u \\ v \\ 1
	\end{pmatrix} = x =
	\begin{pmatrix}
		f & 0 & c_{x} \\ 0 & f & c_{y}\\ 0 & 0 & 1
	\end{pmatrix}
	\begin{pmatrix}
		X \\ Y \\ Z
	\end{pmatrix}
\end{equation*}
Le point 3D étant exprimé dans un autre système orthonormal de coordonnées:
\begin{equation*}
	x =	\begin{pmatrix}
		f & 0 & c_{x} \\ 0 & f & c_{y}\\ 0 & 0 & 1
	\end{pmatrix}
	\begin{pmatrix}
		R & T
	\end{pmatrix}
	\begin{pmatrix}
		X \\ Y \\ Z\\ 1
	\end{pmatrix}
\end{equation*}

On définit la matrice de calibration interne:
\begin{equation*}
	K = \begin{pmatrix}
		f & 0 & c_{x} \\ 0 & f & c_{y}\\ 0 & 0 & 1
	\end{pmatrix}
\end{equation*}
et la matrice de projection $P = K \left(R \ T\right)$.
Ici, la matrice de calibration interne utilise la distance focale $f$ exprimée en pixels.

Dans le cas où les pixels sont des trapézoïdes, on peut généraliser $K$ par:
\begin{equation*}
	\begin{pmatrix}
		f_{x} & s & c_{x} \\ 0 & f_{y} & c_{y}\\ 0 & 0 & 1
	\end{pmatrix}
\end{equation*}
où $s = -f_{x} \cotan \theta$ et $\theta$ est l'angle du parallélograme.

\begin{thm}
	Si $P \in \M_{3, 4}(\R)$ dont la sous-matrice $3\times 3$ gauche est inversible, elle se
	factorise de manière unique en $P = K (R \ T)$ où $K$ est inversible, $R$ est une matrice
	de rotation et $T$ une matrice $3 \times 4$ de translation.
\end{thm}
\begin{proof}
	On décompose la sous-matrice de gauche de $P$ via le procédé de Gram-Schmidt. L'unicité décolue de l'unicité de la décomposition QR.
\end{proof}

Des lignes parallèles dans l'espace se projètent dans un faisceau (ensemble de lignes parallèles ou concourrentes).
Si on se donne un vecteur $d$ de direction:
\begin{equation*}
	\begin{aligned}
		K(X + \lambda d)                 & = KX + \lambda Kd        \\
		l_{x}                            & = (KX) \times (Kd)       \\
		\forall X, \transpose{\ell_{X}}v & = 0, \text{ for } v = Kd
	\end{aligned}
\end{equation*}
On dit que $v$ est le point de fuite des droites de direction $d$.
Si $v_{1} = Kd_{1}$ et $v_{2} = Kd_{2}$ sont des points de fuites de droites "horizontales", un autre ensemble
de droites horizontales a direction $\alpha d_{1} + \beta d_{2}$ et donc son point de fuite est $\alpha v_{1} + \beta v_{2}$
qui appartient à la droite d'horizon $v_{1} \times v_{2}$.

\subsection{Homographies}
Si on étudie deux images différentes, la matrice pour passer de la première image à la seconde est une homographie $3 \times 3$, i.e. une matrice inversible.
\begin{proposition}
	Une homographie préserve l'alignement.
\end{proposition}
\begin{proof}
	En effet, on a:
	\begin{equation*}
		\abs{
			\begin{matrix}
				Hx_{1} & Hx_{2} & Hx_{3}
			\end{matrix}
		} = \abs{H} \abs{\begin{matrix} x_{1}x_{2}x_{3}\end{matrix}}
	\end{equation*}
\end{proof}

Dans le cadre de transformation simples, on a des formes plus simples pour les homographies:
\begin{center}
	\begin{tabular}{cccp{.2\textwidth}}
		Type         & Matrice                                      & Condition              & Invariants      \\
		Rigide       & $\begin{pmatrix}
				                c & -s & t_{x} \\ s & c & t_{y} \\ 0 & 0 & 1
			                \end{pmatrix}$ & $c^{2} + s^{2} = 1$    & Distances
		Similaritude & $ \begin{pmatrix}
				                 c & -s & t_{x} \\ s & c & t_{y} \\ 0 & 0 & 1
			                 \end{pmatrix}$ & $c^{2} + s^{2} \neq 0$ & $\abs{\text{Angles}}$, ratio des distances.
		$
	\end{tabular}
\end{center}


\subsection{Algorithmes et Calibration}

\section{Matrices Fondamentales et Essentielles}




\end{document}

