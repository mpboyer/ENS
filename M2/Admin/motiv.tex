\documentclass[11pt,a4paper,roman]{moderncv}
\usepackage[french]{babel}

\moderncvstyle{classic}
\moderncvcolor{black}

% character encoding
\usepackage[utf8]{inputenc}

% adjust the page margins
\usepackage[scale=0.80]{geometry}

% personal data
\name{Matthieu Pierre Boyer}{}
\email{matthieu.boyer@ens.fr}
\phone[mobile]{+33 6\,13\,68\,72\,59}
\address{4 Rue Commandant Marchand, 69003 Lyon}

\def\nl{\\ \vspace{1em}}

\begin{document}

\recipient{À}{Julien Tierny}
\date{\today}
\opening{\textbf{Objet: Candidature au cours d'Analyse Topologique des Données}}
\closing{Bien cordialement, \vspace{-2em}}

\makelettertitle

Monsieur,
\nl

Je suis un élève de 3ème année au Département d'Informatique de l'École Normale
Supérieure (ÉNS), et je souhaiterais intégrer le cours d'Analyse Topologique des
Données que vous proposez au M2 MVA.
\nl

J'avais déjà commencé à m'intéresser à l'analyse topologique de données lors de mon
stage d'informatique de L3, en les appliquant notamment à l'étdue des différences
entre diverses notions linguistiques portant le même nom.
Mes travaux sur ce sujet ont donné lieu à un article présenté à la conférence
NoDaLiDa/Baltic--HLT en mars 2025, en collaboration avec Mathieu
\textsc{Dehouck}, bien que la majorité de mes tentatives d'utiliser différentes
méthodes d'homologie persistente ne furent pas intégrées à l'article car trop
compliquées à présenter à des linguistes dans un si court article.
\nl

J'ai suivi le cours de Topologie Algébrique de Muriel Livernet au Département de Mathématiques
de l'ÉNS, et m'intéresse tout particulièrement à l'applicabilité de méthodes géométriques,
algébriques et visuelles à la représentation et l'objectivation de propriétés statistiques
de données en sciences expérimentales et particulièrement humaines.
C'est dans ce cadre qu'il me semble que l'analyse topologique des données devient tout
particulièrement utile et importante, et c'est pourquoi je souhaiterais suivre votre cours.

\vspace{1em}
\makeletterclosing
\end{document}
