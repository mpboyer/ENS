\documentclass[11pt,a4paper,roman]{moderncv}
\usepackage[french]{babel}

\moderncvstyle{classic}
\moderncvcolor{black}

% character encoding
\usepackage[utf8]{inputenc}

% adjust the page margins
\usepackage[scale=0.80]{geometry}

% personal data
\name{Matthieu Pierre Boyer}{}
\email{matthieu.boyer@ens.fr}
\phone[mobile]{+33 6\,13\,68\,72\,59}
\address{4 Rue Commandant Marchand, 69003 Lyon}

\def\nl{\\ \vspace{1em}}

\begin{document}

\recipient{À}{Bureau des Admissions\\
	Master Mathématiques, Vision, Apprentissage\\
	Université Paris-Saclay}
\date{\today}
\opening{\textbf{Objet: Candidature au Master M2 Mathématiques, Vision, Apprentissage}}
\closing{En vous priant d'agréer, Madame, Monsieur, à l'expression de mes respectueuses salutations, \vspace{-2em}}

\makelettertitle

Madame, Monsieur,
\nl

Je suis un élève de 2ème année au Département d'Informatique de l'École Normale
Supérieure (ÉNS), et je souhaiterais faire partie du Master Mathématiques,
Vision, Apprentissage (MVA) proposé par votre université.
\nl

J'ai intégré l'ÉNS en 2023 par le biais du concours Informatique MP, ayant pour
objectif de poursuivre une double licence mathématiques-informatique.
Durant ma première année, j'ai finalement décidé de suivre un cursus plus axé
sur l'informatique, tout en essayant d'élargir mes horizons, me permettant
de suivre 26 cours en 1 an et demi, tout en étant membre du Bureau
des Élèves de l'ENS, en donnant régulièrement des cours particuliers et en
étant examinateur oral plusieurs heures par semaine en CPGE MP2I et MPI.
J'ai alors suivi le cours d'introduction à la linguistique de Salvador
\textsc{Mascarenhas} et ai effectué un stage en juin-juillet 2024 sous la
direction de Pascal \textsc{Amsili} et Mathieu \textsc{Dehouck} au Laboratoire
LATTICE en linguistique computationnelle.
C'est à ce moment que j'ai décidé de me tourner plutôt vers les mathématiques
appliquées et l'analyse informatique et algorithmique de données.
\nl

J'avais déjà commencé à m'intéresser à ces sujets lors de mon cursus en CPGE,
mon projet de TIPE portant sur la construction d'un moteur de résolution de
mots-croisés à l'anglaise, en utilisant que des méthodes géométriques ou
quantitatives, mais c'est tout particulièrement durant ce stage que j'ai
orienté ma scolarité.
J'ai notamment découvert les travaux de Julien \textsc{Tierny}
en analyse topologique des données, dans le but d'étudier les différences
entre diverses notions linguistiques portant le même nom.
Mes travaux sur ce sujet ont donné lieu à un article présenté à la conférence
NoDaLiDa/Baltic--HLT en mars 2025 et ma collaboration avec Mathieu
\textsc{Dehouck} s'est poursuivi au long du second semestre 2024, pour le
développement d'un modèle intégrant une mesure géométrique de la proximité des
populations dans l'évolution dans les langages naturels.
Mon stage a par ailleurs obtenu la meilleure note de la promotion.
J'ai par ailleurs récemment entamé une collaboration avec Salvador
\textsc{Mascarenhas} portant sur la loi de Zipf et son utilisation souvent
mal justifiée dans l'étude du langage naturel.
\nl

Il est donc tout naturel pour moi de souhaiter intégrer le Master MVA, puisque
celui-ci est le mieux axé sur les notions de mathématiques appliquées qui
m'intéressent.
Je trouve en effet un intérêt tout particulier dans la recherche de motifs, en
particulier visuels et géométiques, observables dans de nombreux domaines de
recherche, et pas seulement en lien avec l'informatique ou les mathématiques.
J'ai notamment un projet en cours visant à étudier statistiquement les
peintures de Vincent van Gogh pour mieux les comprendre.
L'apprentissage statistique est, à mon avis, dans les plus importants domaines
de recherche des années à venir, et tout particulièrement dans le domaine de
l'étude du langage, et c'est une composante importante de mon intérêt sur ces
questions.
Toutefois, pour pouvoir affirmer que le passage par les statistiques,
l'apprentissage et l'analyse géométrique me paraît bien nécessaire,
j'ai décidé cette année de faire un stage à l'université de Yale sous la
direction de Simon \textsc{Charlow}, axé sur des questions plus théoriques,
à l'intersection de la sémantique des langages naturels et de la théorie des
langages de programmation, dont l'objectif (et le résultat !) est la
formalisation d'un système théorique permettant une extension des systèmes de
définition mathématique de mots d'un lexique courant.
Derrière ces termes complexes, j'ai surtout découvert la réalité de la théorie:
sans travail statistique, une théorie de la sémantique d'un langage comme
l'anglais ne peut devenir réalité.
Au final, mon travail plus théorique durant cette année académique n'a fait que
me conforter dans ma volonté de chercher - et de trouver - des motifs
intrinsèques aux choses qui nous entourent, que ce soit dans le langage
naturel, dans les arts ou dans la médecine.
Il n'y a pour moi, pas de meilleure option de poursuite d'études dans ces
domaines que votre Master, qui est le seul à mettre en avant des cours
d'analyse des données et d'apprentissage instruits qui permettent d'étudier
efficacement des problèmes à la pointe de la recherche et qui met aussi en
avant des méthodes, par exemple en apprentissage profond, qui manquent encore
à mon arsenal de méthodes.
Je pense notamment aux cours d'analyse des données topologique et géométrique,
de transport optimal, d'apprentissage géométrique, ceux liés à l'étude du
langage naturel et aux gros modèles de langue, ainsi qu'aux cours de vision
artificielle.

\vspace{1em}
\makeletterclosing
\vspace{-10cm}
\hfill \includegraphics[width=.3\linewidth]{~/Documents/PERSO/Important/Général/signature.png}
\end{document}
