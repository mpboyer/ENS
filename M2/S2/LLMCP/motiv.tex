\documentclass[11pt,a4paper,roman]{moderncv}
\usepackage[french]{babel}

\moderncvstyle{classic}
\moderncvcolor{black}

% character encoding
\usepackage[utf8]{inputenc}

% adjust the page margins
\usepackage[scale=0.80]{geometry}

% personal data
\name{Matthieu Pierre Boyer}{}
\email{matthieu.boyer@ens.fr}
\phone[mobile]{+33 6\,13\,68\,72\,59}
\address{4 Rue Commandant Marchand, 69003 Lyon}

\def\nl{\\ \vspace{1em}}

\begin{document}

\recipient{À}{Marc Lelarge, Nathanaël Fijalkow}
\date{\today}
\opening{\textbf{Objet: Candidature au cours de Gros Modèles de Langues pour le Code et les Preuves}}
\closing{Bien cordialement, \vspace{-2em}}

\makelettertitle

Messieurs,
\nl

Je suis un élève de 3ème année au Département d'Informatique de l'École Normale
Supérieure (ÉNS), et je souhaiterais intégrer le cours de Gros Modèles de Langues
pour le Code et les Preuves que vous proposez au M2 MVA.
\nl

J'ai un fort intérêt notamment dans l'utilisation de méthodes tirées de la vérification
formelle (et du typage) à l'étude des langages naturels et à la linguistique, et votre
cours me permettrait aussi d'étudier le point de vue contraire: utiliser des méthodes
tirées de l'études langages naturels pour la vérification.
Il me semble que la vérification formelle est d'ailleurs l'un des rares domaines
dans lequel les défauts des gros modèles de langues sont presque complètement effacés,
et il s'agit donc d'un des rares domaines dans lesquels ils sont prometteur pour la
génération.
\nl

J'ai par ailleurs précédemment suivi le cours de Sémantique et Vérification Formelle
donné par Xavier Rival et Jérôme Férêt ainsi que le cours d'Apprentissage pour les
Textes de la première période du MVA.
J'ai précédemment utilisé le langage lean pour construire un parser adapté à une
sémantique compositionnelle et monadique de l'anglais, lors de mon travail avec
Simon Charlow au département de Linguistique de Yale.
J'ai également participé aux travaux sur la
représentation par produit tensoriel du mécanisme d'attention du laboratoire de
linguistique computationnelle de Yale.

\vspace{1em}
\makeletterclosing
\end{document}
