\titre{Sous les Honneurs}

Dans le numéro précédent, par l'intermédiaire de Cigaes, l'imprimante {\tt lwinfo3} nous invitait à son pot de départ,
celle-ci ayant été remplacée par une imprimante plus jeune, après de nombreuses années de service. Dans cet article, je
vous conte l'histoire de ce pot de départ.

\soustitre{Hameçonnés!}
Quelle ne fut pas notre stupéfaction, à DarkSySy, {\tt :billed-cap:} et à moi-même (Pandada, pour vous servir), lorsqu'arrivés au pot de
départ, armés de croissants (à pâte {\it feuilletée}), nous ne vîmes pas {\tt lwinfo3}. 
Cigaes s'avança vers nous et nous expliqua que l'imprimante était trop fatiguée pour finalement venir, et que la pauvre avait eu à subir de lourdes opérations. 
Tout cela n'était donc qu'affabulations? 
Nous décidâmes alors de nous lancer à la recherche de l'imprimante, notre soutien de toujours, que nous avions tant martyrisée sous les bourrages papiers.

\soustitre{À la Recherche de {\tt\large lwinfo3}}
Nous partîmes vers les contrées souterraines du NIR pour tenter de retrouver l'imprimante, mais après avoir cherché dans
tous les recoins, nous nous rendîmes à l'évidence: c'était en vain. À deux bras de baisser les doigts, ou l'inverse, je
ne sais plus, prêts à vider nos cartouches de larmes, l'un d'entre nous eut une idée du toner: chercher dans la benne de
la Cour Pasteur! C'est alors que nous la vim au loin: elle était là, {\tt lwinfo3} nous attendait.

\soustitre{La Cérémonie}
Après avoir libéré notre âme-s\oe{}ur du bourrage d'électronique où elle s'enfonçait, nous fîmes s'asseoir l'imprimante
sur une chaise molletonnée, confort bien mérité après la dureté et la froideur d'une table ressentie pendant toute une
vie. Nous accompagnâmes {\tt lwinfo3} sous l'arche entre l'espace Verdier et l'aile Rataud, passage entre sa carrière au
NIR et son lieu de repos dans la Cour Pasteur, passage entre travail et retraite bien méritée. Faute de millefeuille, et
sans vouloir déranger le repos des morts, nous mangeâmes nos croissants, et fiers du devoir accompli, dîmes adieu à
notre compagnon, non, que dis-je, notre s\oe{}ur.

\soustitre{Les Adieux}
Nous décidâmes après un temps de raccompagner {\tt lwinfo3} à sa place. Il fut pour nous le temps de faire nos adieux,
d'ajouter à l'encre sympathique quelques mots d'amour (comme indiqué dans le \BOcal{} précédent!).

Il est pour moi temps de tourner la page, de continuer au verso de ma vie ce que {\tt lwinfo3} lui a apporté, d'imprimer
ces quelques phrases pour que son souvenir reste. Les normalien·ne·s ne savaient peut-être pas tous ce que c'est, l'amour,
mais le souvenir de {\tt lwinfo3} restera agrafé à nos c\oe{}urs.

Adieu, Farewell, Lebewohl, %別れ, 작별,
Adiós, Agur, Kenavo {\tt lwinfo3}, tant de mots que tu as pu écrire noir sur
blanc pour nous. 

\signature{Pandada, DarkSySy, {\tt :billed-cap:}}

\emph{Vous trouverez en dernière page, en couleur, les photos de cet événement émouvant,
accompagnées de légendes aussi légendaires que
\texttt{lwinfo3}}.
