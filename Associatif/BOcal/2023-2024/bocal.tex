% Choisir la police entre 'Essays457BCE', 'LiberationSans', 'WorkSans', 'ZillaSlab'
\documentclass[minor=7, quiet, police=ZillaSlab]{BOcal}
\usepackage{xurl}
\usepackage{tikz}
% \usepackage{multicol}
% \usepackage{cancel}
% \usepackage{kibitzer} %Pour le bridge
% \usepackage{sudoku} % Pour le sudoku.  Incompatible avec les echecs.
% \usepackage{chess} %À commenter optionnellement s'il n'y a pas d'échecs
% cette semaine.
% \usepackage{verbatim}
% \usepackage{altverse} % Pour mettre en page des vers

\def\numeroBOcal{1216}
\def\nombrePages{4}
\def\dateBOcal{Mercredi 27 Mars -- Jour de la jonquille}

\input{signatures}

\usetikzlibrary{math}
\newcommand{\coussinet}[5]{
        \def\innerradius{#3}
        \def\outerradius{#4}
        \tikzmath{
            %
            \innerinner = max(\innerradius * .7, \innerradius - .3);
            \innerouter = max(\outerradius * .7, \outerradius - .3);
        }
        %
        \draw[pattoune_ext, fill = pattoune_ext, rotate = #5] (#1, #2) ellipse (\innerradius cm and \outerradius cm);
        \draw[pattoune_ext, fill = pattoune_int, rotate = #5] (#1, #2) ellipse (\innerinner cm and \innerouter cm);
        }
%    \definecolor{pattoune_ext}{HTML}{7f2b0a}
%    \definecolor{pattoune_int}{HTML}{bf9005}

    \newcommand{\Pquatrettoune}[1][cadreFond]{
    \colorlet{pattoune_ext}{#1}
    \coussinet{0}{0.5}{1}{1.3}{-15}
    \coussinet{-.25}{2.08}{.4}{.5}{10}
    \coussinet{0.2}{2.35}{.4}{.5}{-1.8}
    \coussinet{.3}{2}{.4}{.5}{-45}
    \coussinet{0.23}{2.3}{.4}{.5}{-25}}

    \newcommand{\PAttoune}[1][cadreFond]{
       \colorlet{pattoune_int}{#1}
       \Pquatrettoune
    }
    
    \newcommand{\pattoune}[1][1]{
        \begin{tikzpicture}[scale = #1, baseline = 0]
            \PAttoune
        \end{tikzpicture}
    }
    \newcommand{\revpattoune}[1][1]{
        \begin{tikzpicture}[scale = #1, baseline = 0]
            \begin{scope}[xscale = -1]
                \PAttoune
            \end{scope}
        \end{tikzpicture}
    }


\columns=4 % La première page a 4 colonnes
\begin{document}

%%%%% PAGE 1: La une, l'édito, l'événement%%%%%%%%%%%%%%%%%%%%%%%%%%%%%%%%%%%%%%%%%%%%%%%%%%%%%%%
%%%%%%%%%%%%%%%%%%%%%%%%%%%%%%%%%%%%%%%%%%%%%%%%%%%%%%%%%%%%%%%%%%%%%%%%%%%%%%%%%%%%%%%%%%%%%%%%%

%%%%% La manchette
\place 1[4]t \manchettelogo[logo][poiscaille]{Numéro \numeroBOcal{} - \nombrePages{} pages}{\dateBOcal}
%\place 1[4]t \manchetteAG[logo][poiscaille]{Numéro \numeroBOcal{} - \nombrePages{} pages}{\dateBOcal}
%\place 1[4]t \manchettebuscal[poiscaille]{Numéro \numeroBOcal{} - \nombrePages{} pages}{\dateBOcal}
%\place 1[4]t \manchetteween[logold]{Numéro {\bocal\huge\numeroBOcal{}} - \nombrePages{} pages}{\dateBOcal}
%\place 1[4]t \manchettenoel[logo][poiscaillenoel]{Numéro \numeroBOcal{} - \nombrePages{} pages}{\dateBOcal}
%%%
%
\input{premiere_page}

\columns=3 % Les autres pages sont sur 3 colonnes
\newpage

%%%%% PAGE 2 : la culture %%%%%%%%%%%%%%%%%%%%%%%%%%%%%%%%%%%%%%%%%%%%%%%%%%%%%%%%%%%%%%%%%%%%%%%%
%%%%%%%%%%%%%%%%%%%%%%%%%%%%%%%%%%%%%%%%%%%%%%%%%%%%%%%%%%%%%%%%%%%%%%%%%%%%%%%%%%%%%%%%%%%%%%%%%%

%% haut
\newbreve t {\gfr{C'était trop mignon! Il faut le commercialiser.} --- Jimmie}
%% bas
\newbreve b {\gfr{Je fais pas de l'ingérence, je fais du journalisme.} --- \gfr{Je veux bien que ce soit anonyme s'il te plaît}}
%
\input{page2}

\newpage
%%%%% PAGE 3 :conférences, clubs%%%%%%%%%%%%%%%%%%%%%%%%%%%%%%%%%%%%%%%%%%%%%%%%%%%%%%%%%%%%%%%%%%
%%%%%%%%%%%%%%%%%%%%%%%%%%%%%%%%%%%%%%%%%%%%%%%%%%%%%%%%%%%%%%%%%%%%%%%%%%%%%%%%%%%%%%%%%%%%%%%%%%

%% haut
\newbreve t {\gfr{Je suis un pendrillon} --- Un responuit trop traquenardé}
%% bas
\newbreve b {\gfr{Ça m'a pris 15s, il m'a suffit de réfléchir 2 minutes.} --- BuggyBaguette}
%
\input{page3}

\newpage
%%%%% DERNIÈRE PAGE : soirées, petites annonces, etc...%%%%%%%%%%%%%%%%%%%%%%%%%%%%%%%%%%%%%%%%%%%
%%%%%%%%%%%%%%%%%%%%%%%%%%%%%%%%%%%%%%%%%%%%%%%%%%%%%%%%%%%%%%%%%%%%%%%%%%%%%%%%%%%%%%%%%%%%%%%%%%

\input{derniere_page}


\end{document}

