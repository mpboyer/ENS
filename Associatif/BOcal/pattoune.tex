\documentclass{cours}
\usepackage{tikzlings}
\usepackage{lipsum}
\usetikzlibrary{shapes,snakes}
\usetikzlibrary{patterns, patterns.meta}

\newcommand{\coussinet}[5]{
        \def\innerradius{#3}
        \def\outerradius{#4}
        \tikzmath{
            %
            \innerinner = max(\innerradius * .7, \innerradius - .3);
            \innerouter = max(\outerradius * .7, \outerradius - .3);
        }
        %
        \draw[fill = pattoune_ext, rotate = #5] (#1, #2) ellipse (\innerradius cm and \outerradius cm);
        \draw[fill = pattoune_int, rotate = #5] (#1, #2) ellipse (\innerinner cm and \innerouter cm);
    }
    \newcommand{\Pquatrettoune}[1][7f2b0a]{
    \definecolor{pattoune_int}{HTML}{\colorInt}
    \definecolor{pattoune_ext}{HTML}{#1}    
    \coussinet{0}{0.5}{1}{1.3}{-15}
    \coussinet{-.25}{2.08}{.4}{.5}{10}
    \coussinet{0.2}{2.35}{.4}{.5}{-1.8}
    \coussinet{.3}{2}{.4}{.5}{-45}
    \coussinet{0.23}{2.3}{.4}{.5}{-25}}

    \newcommand{\PAttoune}[1][bf9005]{
        \def\colorInt{#1}
        \Pquatrettoune
    }
    
    \newcommand{\pattoune}[1][1]{
        \begin{tikzpicture}[scale = #1]
            \PAttoune
        \end{tikzpicture}
    }
    \newcommand{\revpattoune}[1][1]{
        \begin{tikzpicture}[scale = #1]
            \begin{scope}[xscale = -1]
                \PAttoune
            \end{scope}
        \end{tikzpicture}
    }


\begin{document}
\begin{center}

    \pgfdeclarelayer{background}
    \pgfsetlayers{background,main}

    \tikzset{
        % Add more line modifications here:
        papyrus line/.style={ line width=1pt, color = black }
    }

    %Usage: \drawRole{}{}{}
    \def\drawRole#1#2#3{%
        \begin{scope}[yscale=#2,scale=0.6*#3]
            % Draw role on the left
            \draw[papyrus line] (A.#1 west) .. controls +(0,1) and +(0,1) .. +(-1,0) .. controls +(0,-1) and +(0,-1) .. +(-.2,0) .. controls +(0,.7) and +(0,.7) .. +(-.8,0) .. controls +(0,-.5) and +(0,-.5) .. +(-.5,0) -- +(-.2,0);
            \draw[papyrus line, fill = blue] (A.#1 west) +(-.6,-.75) -- +(0,-.75) ;
            \draw[papyrus line, fill = blue] (A.#1 west) +(-.65,-.375) -- +(-.25,-.375) ;


            % Draw right corner and vertical line
            \draw[papyrus line] (A.#1 east) .. controls +(0,.7) and +(.4,0) .. +(-.5,0.751) -- ($(A.#1 west)+(-.5,.751)$);
        \end{scope}
    }
    %Usage: \papyrus[]{}
    \newcommand\papyrus[2][1]{%
        \tikz{
            \node[inner xsep=1em, inner ysep=0.5em] (A) {\color{vulm}#2};  % Draw the text of the node
            \begin{pgfonlayer}{background}  % Draw the shape behind
                
                \drawRole{north}{1}{#1}
                % \drawRole{south}{-1}{#1}
                \begin{scope}[yscale = -1, scale = 0.6*#1]
                    %\draw[papyrus line] (A.south west) .. controls +(0,.7) and +(.4,0) .. +(0,0.751) -- ($(A.south west)+(0,.751)$);
                    % \draw[papyrus line] (A.south west) -- ($(A.south west) + (0,.751)$);
                    \draw[papyrus line] (A.south east) .. controls +(0,.7) and +(.4,0) .. +(-.5,0.751) -- ($(A.south west)+(.5,.751)$) -- (A.south west);
                \end{scope}

                %\draw ([papyrus line] A.north west) -- (A.north east);
                \draw[papyrus line, fill = blue] (A.north east) -- (A.south east);
                %\draw[papyrus line] (A.south east) -- (A.south west);
                \draw[papyrus line, fill = blue] (A.south west) -- (A.north west);
                %\fill[white]   (A.north west) -- (A.north east) -- (A.south east) -- (A.south west) -- (A.north west);
            \end{pgfonlayer}}
    }

    \papyrus{\parbox{.9\textwidth}{\lipsum[90]}}
\end{center}
\end{document}