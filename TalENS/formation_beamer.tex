\documentclass{beamercours}
\title{Les Primitives}
\subtitle{Classe Inversée Formation}
\author{Matthieu Boyer}
\date{\today}


\begin{document}
    \maketitle

    \section{Préliminaires : Fonction, Fonction Dérivée}
            \begin{frame}
                \frametitle{Fonction}
                Une fonction $f$ est un objet mathématique qui prend une valeur d'entrée $x$, et renvoie une valeur de sortie $f(x)$. \\
                Dans notre cas, $x$ et $f(x)$ seront des nombres \textsl{réels}, comme $1, 2, 3.14, \pi, e, \sqrt{2}, 42, 469497.840019$\\
                On peut alors représenter une fonction par sa courbe.   
            \end{frame}

            \begin{frame}
                \frametitle{Dérivation}
                On appelle dérivée d'une fonction $f$, la fonction $f^{'}$ qui a $x$, associe la pente de la tangente à la courbe de $f$ en $(x, f(x))$.\\
                Cette définition implique que toute fonction n'est pas dérivable.\\
                On dispose de nombreuses règles de dérivation, les seules qui nous intéressent ici étant : 
                \begin{itemize} 
                    \item Si $c$ est une fonction constante, $c^{'} = \tilde{0}$, et ce sont les seules fonctions de dérivée nulle.
                    \item Si $f$ et $g$ sont des fonctions : $(f+g)^{'} = f^{'} + g^{'}$ (linéarité de la dérivation)
                \end{itemize}
            \end{frame}

    \section{Primitives}
        \begin{frame}
        \frametitle{Primitive d'une Fonction}
            On appelle Primitive de $f$ toute fonction $F$ telle que $F^{'} = f$.\\
            Une telle fonction n'est pas nécessairement unique : Si $F$ est une primitive de $f$, si $c$ est une constante, $F + c$ l'est aussi.\\
            On en déduit que deux primitives d'une même fonction sont égales à constante près.
        \end{frame}

        \begin{frame}
        \frametitle{Aire sous la Courbe}
            On note : $\int_{a}^{b} f(x) \mathrm{d}x$ l'aire sous la courbe de $f$ sur le segment $\left[a, b\right]$ i.e. entre $a$ et $b$.\\
            On appelle ce nombre l'intégrale de $f$ entre $a$ et $b$
            On ne peut définir ce nombre que pour certaines fonctions, dites continues par morceaux.\\
            On remarque par ailleurs que l'aire sous la courbe de $f$ sur le segment $\left[a, a\right]$ est nulle, et donc que l'intégrale de $f$ entre $a$ et $a$ est nulle. 
        \end{frame}

        \begin{frame}
            \frametitle{Primitives par l'Aire sous la Courbe}
            On peut montrer que la fonction \[F : 
                \begin{cases}
                    \mathbb{R} &\rightarrow \mathbb{R}\\
                    x &\mapsto \int\limits_{a}^{x} f(t) \mathrm{d}t 
                \end{cases}
            \]
            est une primitive de $f$, la seule qui vaut $0$ en $a$. \\
            Ceci montre que toute fonction continue a une primitive. On appelle ce résultat \textsl{Théorème Fondamental du Calcul Intégral}
        \end{frame}
        \begin{frame}
            \frametitle{Aire sous la Courbe par Primitive}
            On peut montrer réciproquement que si $F$ est une primitive de $f$, alors \[
                \int_{a}^{b} = \left[F\right]_{a}^{b} = F(b) - F(a)
            \]
            Ceci découle du fait que deux primitives d'une même fonction sont égales à constante près.
        
        \end{frame}

    
\end{document}
