\documentclass{cours}

\title{Formation TalENS}
\date{\today}
\author{Tiphaine de Gésincourt}

\begin{document}
\section*{Introduction}
\begin{enumerate}
    \item Planning Général
    \item Organisation
    \item Commentaires sur Livret
    \item Fiche à Rendre
    \item Livret et Restitution
    \item Sorties Culturelles 
    \item Moodle
    \item Contenu d'une Séance
\end{enumerate}

\section{Planning Général}
Voir sur le Drive/Moodle.\\


\section{Organisation}
Sessions au Lycée Colbert pour les Premières, entre 14h30 et 17h. Sessions aux Mines pour les Term.\\
Il y a des apéros après les séances \textbf{sans les élèves}. Groupe = SF2.

\section{Commentaires sur Livret}
Commentaires sur les élèves à rendre en Janvier, qui apparaissent sur le livret scolaire des élèves. \\
Essayer de ne pas toujours mettre le même commentaire, c'est un vrai travail à faire. \\
Préparer un carnet avec des notes sur chacun des tutorés, ils se ressemblent un peu tous. \\
Il y aura un drive dès la semaine prochaine, avec les commentaires des tuteurs et tutrices sur les élèves de première qu'ils ont eu en entretien. On demande aux jeunes ce qu'est pour eux l'égalité des chances. Travail syntaxique à faire. \\
Les commentaires ont un poids important par rapport à celui des professeurs, car ils obtiennent une mention \textsl{Cordées de la Réussite de PSL}.

\section{Fiche à Rendre}
Toutes les infos seront dans le drive. Elles sont à rendre le 1er février. \\
Fiches soit de méthodologie (surtout pour les premières), soit de préparation à l'enseignement supérieur, mises en ligne en Mars 2024 sur le moodle. \\
Fiche = 1 page, 2 pages jusqu'à plus si besoin. \\
Il faudra s'inscrire sur le drive pour le choix des fiches, histoire de ne pas avoir 500 personnes qui font la même fiche. 

\section{Livret et Restitution}
Texte à rendre pour montrer son engagement, raconter ce qui a été fait dans l'année. \textbf{NE PAS FAIRE DE PDF}\\
Sert à valider l'engagement et les crédits ECTS.\\
Toutes les fiches (de chaque niveau) sont compilées dans un livret disponible aux élèves.\\
Pareil, les anciens livrets sont sur le drive.
\\
Restitution préparée la séance précédente, le 27 avril, tout le monde revient à l'ENS.\\
Chaque groupe présente devant deux trois autres groupes (littéraires, scientifiques, premières, terminales) ce qu'ils ont fait dans l'année, sous la forme souhaitée (cours, film, théâtre, quiz, jeux...)\\
L'idée est de travailler l'oral, ce qui est engageant pour les élèves. De toute façon, personne n'est prêt, et ça se fait dans la \textsl{biENveIlLanCe}. Format à discuter.

\section{Moodle}
Les élèves y vont peu, et c'est pourtant là qu'on va tout mettre (fiches...), un lieu de discussion pour toute la promo. Tous les jeunes se connaissent déjà, ils se sont rencontrés lors du campus fin août. \\
Module de présence : trois états, \textit{A} pour absent, \textit{P} pour présent, \textit{E} pour excusé. Présence \textit{A} $\implies$ convocation au lycée. Nécessaire pour la responsabilité.\\


\section{Sorties Culturelles}
Bonne chose de se renseigner sur le lieu, mais il faut choisir un lieu un peu inconnu. \\
Première sortie : 45 mn dans le musée et après bah, parc.   

\end{document}