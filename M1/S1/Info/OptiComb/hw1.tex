\documentclass[math]{cours}
\title{Homework 1}
\author{Matthieu Boyer}

\def\bR{\bar{R}}
\def\tR{\tilde{R}}
\def\Ac{A^{\complement}}
\def\oc{\mathrm{oc}}

\begin{document}
\maketitle
\section{Question 1}
\subsection{Question 1(a)}
Initially, we have $d^{*} = 0$.
Moreover, we always have $d^{*} \geq 0$ and an increase of $d^{*}$ is caused by a relabeling.
Thus, $d^{*}$ can only increase $2n^{2}$ times (the maximum number of relabelings) and decrease as many times.\\
$\boxed{\text{There are thus at most } 4n^{2} \text{ phases.}}$

\subsection{Question 1(b)}
\begin{itemize}
	\item Relabeling $v$ causes $\bar{d}(v)$ to increase but cannot cause $\bar{d}(w)$ to increase if $w \neq v$.\\
		\boxed{\text{Thus, relabeling a node increases $\Phi$ by at most $\frac{n}{K}$.}}
	\item A saturating push creates at most one new active node.\\
		\boxed{\text{Thus, a saturating push increases $\Phi$ by at most $\frac{n}{K}$.}}
	\item A nonsaturating push across the edge $(u, v)$ deactivates node $u$ and might activate node $v$.
		Then we have $\bar{d}(v) \leq \bar{d}(u)$, and hence a nonsaturating push does not increase $\Phi$.

		\medskip

		During heavy phases, we execute $\rho > K$ nonsaturating pushes.
		Since $d^{*}$ is constant during the phase, all $\rho$ nonsaturating pushes must be from nodes at level $d^{*}$.\\
		Indeed, we choose nodes from the highest level, thus $d^{*}$.\\
		The phase terminates either when all nodes in level $d^{*}$ are deactivated or when relabeling moves a node to level $d^{*} + 1$.\\
		Level $d^{*}$ thus contains $\rho > K$ nodes (either active or inactive) throughout the phase.\\
		Hence, each nonsaturating push decreases $\Phi$ by at least one, since $\bar{d}(v) \leq \bar{d}(u) - 1$ for $(u, v)$ with $\abs{\left\{w\mid d(w) = d(u)\right\}} \geq K$.\\
		Finally, a heavy phase of non saturating push will decrease $\Phi$ by at least $\rho > K$.\\

		For light phases, the bound is easier: the number of nonsaturating pushes is bounded $K$.
\end{itemize}

\subsection{Question 1(c)}
The total increase of $\Phi$ is bounded by $\frac{(2n^{2} + 2nm)n}{K}$ and so the total decrease cannot be more than that (since $\Phi \geq 0$).
Therefore, the number of nonsaturating push cannot be more than $\frac{2n^{3} + 2n^{2}m}{K}$.
The number of non saturating pushes in both phases, is then bounded by:
\begin{equation*}
	\frac{2n^{3} + 2n^{2}m}{K} + 4n^{2}K
\end{equation*}
since $4n^{2}$ is the number of phases (and thus more than the number of light phases).

Finally, since $n = \O(m)$ (the graph being connex $m \geq n - 1$ and $n \leq m + 1$), taking $K = \sqrt{m}$ we get a complexity in $\O(n^{2}\sqrt{m})$.

\medskip
\hrule
\hrule
\medskip

\section{Question 2}
We will use Ford-Fulkerson's theorem on an appropriate graph to prove this property.
Let us write $U = \left\{a_{1}, \ldots, a_{n} \right\}$.
We define a set of vertices $V$ by:
\begin{equation*}
	V = \left\{s, \bar{S_{1}}, \ldots, \bar{S_{k}}, \bar{a_{1}}, \ldots, \bar{a_{n}}, \tilde{a_{1}}, \ldots \tilde{a_{n}}, \bar{T_{1}}, \ldots, \bar{T_{n}}, t\right\}
\end{equation*}
That is, we have $4$ families of vertices that we will connect:
\begin{enumerate}
	\item $\mS = \left\{\bar{S_{j}} \forall j \leq k \right\}$
	\item $\bR = \left\{\bar{a_{i}} \forall i \leq n \right\}$
	\item $\tR = \left\{\tilde{a_{i}}\forall i \leq n \right\}$
	\item $\mT = \left\{\bar{T_{j}}\forall j \leq k \right\}$
\end{enumerate}
Then, we add edges with a capacity function $c: E \to \R^{+} \cup \{+\infty\}$ and a minimum flow function $l: E \to \R^{+}$:
\begin{itemize}
	\item An edge $(s, \bar{S_{i}})$ with capacity $1$ for each $j \in \onen{k}$.
	\item An edge $(s, \tilde{a_{i}})$ with capacity $0$ for each $i \in \onen{n}$.
	\item An edge $(\bar{S_{j}}, \bar{a_{i}})$ with infinite capacity for $i, j$ such that $a_{i} \in S_{j}$.
	\item An edge $(\bar{a_{i}}, \tilde{a_{i}})$ with capacity $1$ for each $i \in \onen{n}$.
	\item An edge $(\bar{a_{i}}, t)$ with capacity $0$ for each $i \in \onen{n}$.
	\item An edge $(\tilde{a_{i}}, \bar{T_{j}})$ with infinite capacity for $i, j$ such that $a_{i} \in T_{j}$.
	\item An edge $(\bar{T_{j}}, t)$ with capacity $1$ for each $j\in \onen{k}$.
	\item An edge $(s, t)$ with infinite capacity and an edge $(t, s)$ with minimum flow $k$ and infinite capacity.
\end{itemize}
Then having a feasible flow on this graph is equivalent to having a common SDR for $S$ and $T$,
	since necessarily we have $1$ flow going in each $S_{i}$ (from the fact that $\sum_{v \in \delta^{-}(s)} f(v, s) = f(t, s) \geq c(t, s) \geq k$)
	and thus need to choose an edge from $S_{i}$ to a certain $\bar{a_{j}}$ (such that $a_{j} \in S_{i}$)
	and that means $\tilde{a_{j}}$ leaves its flow to a $T_{l}$ such that $a_{j} \in T_{l}$.

Then, by Ford-Fulkerson's max-flow-min-cut theorem, this is possible if and only if, for all subset $A$ of the vertices, we have the following inequality:
\begin{equation}
	c(A, \Ac) = \sum_{e \in E \cap \left(A\times \Ac\right)} c(e) \geq
	\sum_{e \in E \cap \left(\Ac\times A \right)}l(e) = l(\Ac, A)
	\label{eqfordfulkerson}
\end{equation}

Here, we have multiple cases to more easily find a closed form for this inequality from the capacity and minimal flows:
\begin{enumerate}
	\item Either both $s$ and $t$ are in $A$ and then:
		\begin{align*}
			c(A, \Ac) = c\left(s, \mS\cap \Ac \right) + c\left(\mS\cap A, \bR\cap \Ac\right) + c\left(\bR\cap A, \tR \cap \Ac \right) + c\left(\tR\cap A, \mT \cap \Ac\right)\\
			l\left( \Ac, A \right) = l\left(\bR\cap \Ac, \tR\cap A \right) = 0
		\end{align*}
	\item Either both $s$ and $t$ are in $\Ac$ and then:
		\begin{align*}
			c\left(A, \Ac\right) = c\left(\mS \cap A,  \bR \cap \Ac\right) + c\left(\bR \cap A, \tR\cap \Ac \right) + c\left(\tR\cap A, \mT \cap \Ac \right) + c(\mT\cap A, t)\\
			l\left(\Ac, A \right) = l\left(\bR \cap \Ac, \tR\cap A \right) = 0
		\end{align*}
		Both of these cases are symmetrical.
	\item If $s \in \Ac, t\in A$ then $c(A, \Ac) \geq c(t, s) = \infty$ so the inequality is always verified.
	\item Finally, if $s \in A$ and $t \in \Ac$, we have:
		\begin{align*}
			c\left(A, \Ac \right) = c\left(s, \mS \cap \Ac \right) + c\left(\mS \cap A, \bR \cap \Ac \right) + c\left(\bR \cap A, \tR \cap \Ac \right) + c\left(\tR\cap A, \mT \cap \Ac\right) + c\left(\mT\cap A, t \right)\\
			l\left(\Ac, A \right) = l\left( \bR\cap \Ac, \tR\cap \A \right) + k = k
		\end{align*}
\end{enumerate}
However, the inequalities of the first two case are less strong than the equality of the last case since:
\begin{align*}
	c\left(\mT\cap A, t \right) = \abs{\mT \cap A} \leq k\\
	c\left(s, \mS\cap \Ac \right) = \abs{\mS\cap \Ac} \leq k
\end{align*}
Therefore, the inequality \ref{eqfordfulkerson} reduces to:
\begin{align*}
	\abs{\mS \cap \Ac} + c\left(\mS\cap A, \bR\cap \Ac \right) + c\left(\bR\cap A, \tR\cap \Ac \right) + c\left( \tR\cap A, \mT\cap \Ac \right) + \abs{\mT \cap A} \geq k\\
	\abs{\mT \cap \Ac} + \abs{\mS\cap A} \leq k + c\left(\mS \cap A, \bR\cap\Ac \right) + c\left(\bR \cap A, \tR\cap \Ac \right) + c\left(\tR\cap A, \mT\cap\Ac \right)
\end{align*}
So there is a feasible flow if the above inequality stands for all $A \subseteq V$.
This inequality is always true if either $\left(\mS\cap A, \bR\cap \Ac \right)$ or $\left(\tR\cap A, \mT\cap \Ac \right)$ is not empty.
If this is not true, then we know the outgoing edges of any node in $\mS \cap A$ go to nodes in $\bR \cap A$ and that all ingoing edges to nodes in $\mT\cap \Ac$ come from nodes in $\tR\cap\Ac$.
We then see that the right side is at its lowest point when taking $\bR \cap A = \delta^{+}\left(\mS \cap A\right)$ and $\tR\cap \Ac = \delta^{-}\left( \mT\cap \Ac \right)$.
Then, we get:
\begin{equation*}
	\abs{\mS \cap A} + \abs{\mT\cap \Ac} \leq n + c\left(\delta^{+}\left(\mS\cap A \right), \delta^{-}\left( \mT\cap B \right) \right)
\end{equation*}
This being true for all $A \subseteq V$, we can replace $\mS\cap A$ for any subset $X$ of $\mS$ or even $X \subseteq \onen{k}$, and similarly we replace $\mT\cap \Ac$ by any subset $Y$ of $\onen{k}$.
Then, the equation becomes:
\begin{equation*}
	\abs{X} + \abs{Y} \leq k + \sum_{I\left(X\right) \cap I\left(Y\right)} 1
\end{equation*}
which is exactly the result:
There is a common set of distinct representatives for $S = \left\{S_{1}, \ldots, S_{k} \right\}$ and $T = \left\{T_{1}, \ldots, T_{k} \right\}$ over a same ground set $U$ if and only if, for every $X, Y \subseteq \onen{k}$ the following inequality is verified:
\begin{equation*}
	\abs{X} + \abs{Y} \leq k + \abs{I\left(X\right) \cap I\left(Y\right)}
\end{equation*}

\medskip
\hrule\hrule
\medskip

\section{Question 3}
We will first prove that all edges in $M_{1}$ are in $E''$.
Since by duality we know that one can get a minimum \emph{integral} vertex cover from a maximum \emph{cardinality} matching on a non weighted graph (which is the case of $G'$ since all edges have the same positve non-zero weight), we know that a minimum integral vertex cover on $G'$ will have exactly one vertex per edge of $M_{1}$ and edges for unconnected vertices in $G'$.
Thus, $E''$ will contain all edges of $M_{1}$ plus the edges of $G$ of weight $2$ that are adjacent to an edge of $M_{1}$ and the edges of $G$ of weight $1$ that are not touched by the vertex cover, that is, edges that are at least one edge away from $M_{1}$.

Let $M$ be a matching in $G$.
Since $M_{1}$ is of maximum cardinality, $M$ cannot have more than $\abs{M_{1}}$ edges of weight $2$.
Moreover, when constructing $M_{2}$ by augmenting in $G''$, we maximise the matching size while respecting the vertex cover constraints and thus the matching constraints.
We have thus a number as large as possible of edges of weight $1$.
Therefore, $M$ cannot have have more edges than $M_{2}$ and thus $w(M) \leq w(M_{2})$.

\medskip
\hrule\hrule
\medskip

\section{Question 4}
First, we remind the Tutte-Berge's minimax formula on a network $\mathcal{G} = \left(V, E\right)$ in the formulation we will be using:
\begin{equation}
	\max_{M \text{ matching}} \abs{M} = \frac{1}{2}\min_{U \subseteq V} \left(\abs{U} - \oc\left(\mathcal{G} \setminus U\right) + \abs{V} \right)
\end{equation}
where $\oc(H)$ is the number of connex components of $H$ with an odd number of vertices (which will be denoted as \emph{odd components}).
We know then that a graph $\mathcal{G}$ has a perfect matching if and only if there is a matching of size $\frac{\abs{V}}{2}$, or, plugging it into the formula, if and only if the suppression of any subset $U$ creates at most $\abs{U}$ odd components, i.e., the following condition is true:
\begin{equation}
	\label{eq:tuttecondition}
	\forall U \subseteq V, \oc\left(\mathcal{G} \setminus U\right) \leq \abs{U}
\end{equation}

Now, if the following condition is verified for our graph $G = \left(A\sqcup B, E\right)$ where $\abs{A} = \abs{B} = n = \frac{\abs{V}}{2}$:
\begin{equation}
	\forall X \subseteq A, \abs{\Gamma\left(X\right)} \geq \abs{X}
	\label{eq:hallcondition}
\end{equation}

Consider the graph $H$ built from $G$ by adding edges to make $B$ a clique (complete graph), then if $H$ has a perfect matching, since edges from vertices in $A$ considered in $H$ must go to vertices in $B$ considered as vertices in $H$, then $G$ has a perfect matching.
We will now show that if $G$ satisfies \ref{eq:hallcondition} then $H$ has a perfect matching, meaning $G$ has a perfect matching.
To do so, we will show that $H$ verifies \ref{eq:tuttecondition}. Suppose there is a certain $U$ such that $\oc\left(H \setminus U\right) > \abs{U}$.
Since $B$ seen in $H$ forms a clique, there is a certain component $C$ of $H \setminus U$ which contains all of $B \setminus U$, and the other components are $k$ singletons from $A \setminus U$.
There are then two cases: either $\abs{C}$ is odd and $\oc\left(H \setminus U\right) = k + 1$ or $\abs{C}$ is even and $\oc\left(H \setminus U\right) = k$.
Since we know that $G$ verifies \ref{eq:hallcondition}, if we let:
\begin{equation*}
	S = \left\{x \in A\mid \Gamma(x) \subseteq B \cap U \right\}
\end{equation*}
we have
\begin{equation*}
	k = \abs{S} \leq \abs{B \cap U} \leq \abs{U}
\end{equation*}
and thus $\oc\left(H \setminus U \right) \leq \abs{U} + 1$.
However since $\abs{H} = 2n$ is even, $\oc\left(H \setminus T \right)$ has the same parity as $\abs{T}$.
Thus $k \leq \abs{U}$ or $\oc\left(H \setminus U \right) \leq \abs{U}$ for any $U \subseteq V$.
\boxed{\text{Finally, } H \text{ has a perfect matching per \ref{eq:tuttecondition} and thus } G \text{ has a perfect matching.}}



\end{document}
