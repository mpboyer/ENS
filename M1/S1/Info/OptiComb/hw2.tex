\documentclass[math, info]{cours}
\author{Matthieu Boyer}
\title{Homework 2}
\begin{document}
\maketitle
\section{Question 1}
\begin{notationT}
	For $I \subseteq E$ and $b \in B$, we will denote $I(b) = \left\{ a \in A \mid (a, b) \in I \right\}$ and by $I(X) = \left\{ a\in X \mid \exists b\in B, (a, b) \in I \right\}$.
\end{notationT}

We then define the matroids $\mathbb{A} = (E, \A)$, $\mathbb{B} = (E, \B)$ where:
\begin{align*}
	\A = \left\{ I \subseteq E \mid \abs{I(a)} \leq 1 \forall a \in A\right\}\\
	\B = \left\{ I \subseteq E \mid I(b) \in \mathcal{M}_{b} \forall b \in B\right\}
\end{align*}

We then see that $M \subseteq E$ is a $A$-perfect matching if and only if $\abs{M} = \abs{A}$ and
$M$ is an independent set of $\A$ and $\B$.
Thus, we will call sets in $\A \cap \B$ independent matchings.

Then, since $\abs{A} \geq \max_{I \in \A}{\abs{I}}$, from Edmonds' mini-max formula on matroid intersection, we just need to have $\min_{I\subseteq E} r_{\A}(I) + r_{\B}(E\setminus I) \geq \abs{A}$ to have the existence of a $A$-perfect matching.

We define $s: 2^{E} \to \N$ as:
\begin{equation}
  s(I) = \sum_{b \in B} rank_{M_{b}}(I(b) \cap N(b))
\end{equation}

We see that the rank set in $\B$ can be seen as the ranks on each component (by separating edges on the $b \in \B$ they are connected to).
Indeed, since $\B$ can be seen as a union of matroids (the $M_{b}$ seen as matroids on the edges connected to $b$) we have, for $I \subseteq E$:
\begin{equation*}
	r_{\B}(I) = \min_{T \subseteq I} \abs{I\setminus T} + s(T) = \min_{T \subseteq I} \abs{I} - \abs{T} + s(T)
\end{equation*}
Then plugging this into our main equation:
\begin{equation*}
	\begin{aligned}
		r_{\A}(E\setminus I)  + r_{\B}(I) =& r_{\A}(E\setminus I) + \min_{T} \abs{I} - \abs{T} + s(T)\\
		\geq & \min_{T} \abs{I} - \abs{T} + s(T)\\
		= & \min_{T} \abs{A} - \abs{T(A)} + s(T)
	\end{aligned}
\end{equation*}
But since this should be greater than $\abs{A}$ for all $T$ and all $I$, it is equivalent to being true for all possible $A' = T(A)$ (and modifying the \emph{type} of $s$ accordingly, which doesn't change anything) and thus:
\begin{equation*}
	\boxed{\max_{I\in \A \cap \B} \abs{I} = \abs{A} \Longleftrightarrow \forall A'\subseteq A, s(A') - \abs{A'} \geq 0}
\end{equation*}
which is the wanted result.

\section{Question 2}
Let $F = 2^{I}$ and let us denote by $g: 2^{\mathcal{F}} \to \R^{+}$ the function that to a family of sets gives their combined profit.
Clearly, $g$ is submodular.
Furthermore we denote by $X_{0}$ the empty set, and by $X_{i}$ the set of items taken after $i$ knapsacks were filled by our algorithm.
Since we apply the FPTAS $k$ times, and since $g$ is submodular, we have:
\begin{equation}
  g(X_{i}) - g(X_{i - 1}) \geq (1 - \epsilon)\frac{OPT - g(X_{i - 1})}{k}
	\label{eq:induction}
\end{equation}
for each $i$, where $OPT$ is the weight of an optimal solution.
Then, we have:
\begin{equation}
  g(X_{1}) - g(X_{0}) = g(X_{1}) \geq (1 - \epsilon)\frac{OPT}{k} = OPT(1 - \left(1 - \frac{1}{k}\right) - \epsilon) = OPT\left( 1 - \left(1 - \frac{1}{k}\right) - \O(\epsilon) \right)
	\label{eq:firststep}
\end{equation}
and then:
\begin{equation*}
  \begin{aligned}
    g(X_{2}) \geq (1 - \epsilon)\frac{OPT - g(X_{1})}{k} =& (1-\epsilon)OPT\left(1 - \left(1 - \frac{1}{k}\right) - \epsilon\right)\\
    =& OPT\left( 1 - \left(1 - \frac{1}{k}\right)^{2} - \epsilon\right) - OPT\times \epsilon\left(1 - \left(1 - \frac{1}{k}\right) - \epsilon\right)\\
    =& OPT\left(1 - \left(1 - \frac{1}{k}\right)^{2} - \O(\epsilon)\right)
  \end{aligned}
\end{equation*}
By induction:
\begin{equation*}
  g(X_{i}i) \geq OPT\left(1 - \left(1 - \frac{1}{k}\right)^{i} - \O(\epsilon)\right)
\end{equation*}
And thus:
\begin{equation*}
  g(X_{k}) \geq OPT\left(1 - \left(1 - \frac{1}{k}\right)^{k} - \O(\epsilon)\right) \geq OPT\left(1 - \frac{1}{e} - \O(\epsilon)\right)
\end{equation*}

\section{Question 3}
\subsection{Part 1}
I worked on this question with Mateo Torrents.
\smallskip

Let $\Delta_{k} = \Delta_{i \in \onen{k}} V_{f_{j}}$. We will consider increasing sets $A_{k}$  of vertices to prove by induction: $\Delta_{k} \cap A_{k} = U \cap A_{k} \text{ or } V\setminus U \cap A_{k}$ 
Let $\mathcal{H} = H - (f_{i})_{i \in \onen{t}}$. 
We will denote by $C_v$ the component containing $v$ in $\in \mathcal{H}$. We always have $C_{v} \subseteq U$ or $C_v \subseteq V \setminus U$. 
Let $A_{k}$ such that: 
\begin{itemize}
  \item $A_{1} = C_{v}$ for a certain $v \in V$
  \item $A_{k + 1} = A_{k} \cup C_v$ for a certain $v \in \delta(A_{k})$. We write $C_{k + 1} = C_{v}$
\end{itemize}
We order the $f_{i}$ such that the edge between $A_{k}$ and $C_{k + 1}$ is $f_{k}$.
We will now show the property by induction. It is clearly true for $k = 1$. 
Notice that $V_{f_{k + 1}}$ contains only one of $A_{k}$ and $C_{k + 1}$ and is disjoint from the other.
Let $f_{k} = (u, v)$: when going from $\Delta_{i}$ to $\Delta_{i + 1}$, with $i < k$, then $u, v$ stay in the same state (in or out of $\Delta_i$).  
Then, only when adding $V_{f_{k}}$ to the difference do $u$ and $v$ get treated differently.
Therefore, $\Delta_{k} \cap A_{k} \subseteq U$ if and only if $\Delta_{k} \cap C_{k} \subseteq V \setminus U$. 
But when going to $\Delta_{k + 1}$, either $u$ or $v$ changes side, and thus we get $\Delta_{k + 1} \cap A_{k} \subseteq U$ if and only if $\Delta_{k + 1} \cap C_{k + 1} \subseteq U$, hence keeping the proposition. 

\subsection{Part 2}
\begin{algorithm}
	\caption{Minimum Odd Size Cut}
	\begin{itemize}
		\item First, we build the Gomory-Hu tree of our graph.
		\item Then, for each edge in the tree we consider both components formed by removing the edge.
		\item For every odd-sized such component, we retrieve the cut size (the label of the edge in the Gomory-Hu tree), if it's less than one we return True.
			If none are of cut size $\leq 1$ then we return false.
	\end{itemize}
\end{algorithm}

This algorithm takes:
\begin{equation*}
	\O\left( \underbrace{(n-1)\times \text{max-flow}}_{\text{Gomory-Hu algorithm}} + \underbrace{n^{2}}_{\text{Check Sizes}} + \underbrace{n}_{\text{Retrieve Cut-size}} \right)
\end{equation*}

For correctness, we only need to show there is at least one $V_f$ of odd size. 
Indeed if $V_{f}$ is the minimum $u-v$ cut (for $u \in U$, $v \in V \setminus U$), then $w(V_{f}) \leq w(U)$ since $U$ is a $u-v$ cut which gives the result if $V_{f}$ is odd. 
Then to show one of the $V_{f}$ is odd, we only need to see that if all of the $V_{f}$ are even (as well as their complements), then both $V$ and $\Delta_{i \in \onen{t}} V_{f_{i}}$ are even since $\abs{A \Delta B} = \abs{A} + \abs{B} - 2\abs{A \cap B}$.
But since $U$ is odd and $V$ is even, $V \setminus U$ is odd and we have a contradiction.
Thus, at least one of the $V_{f}$ or $V \setminus V_{f}$ is odd.  

\end{document}
