\documentclass[math]{cours}
\title{Introduction à la Robotique}
\author{Justin Carpentier et Stéphane Caron}

\begin{document}
\bettertitle
\section*{Introduction}
L'objectif est de créer un framework général permettant de préparer à l'automatisation de tous les types de robots:
\begin{itemize}
	\item Manipulation
	\item Locomotion
	\item Hybrides
\end{itemize}
Un robot humanoïde étant beaucoup plus complexe qu'un quadripède, c'est le domaine qui fait avancer la robotique.\\

On peut voir la robotique comme la fabrique artificielle du mouvement (par opposition au vivant).
La robotique s'itnéresse au lien entre des choses qui bougent et le monde réel, et à son contrôle par ordinateur.
C'est aussi le miroir de l'humain et des manières dont nous bougeons naturellement.
On a une analogie entre les robots et le vivant.
\begin{center}\begin{tabular}{cc}
		Outil Inerte & Pierre\\
		Action sans Locomotion & Végétal\\
		Locomotion & Animal
\end{tabular}\end{center}
L'objectif est de construire des algorithmes pour des robots capables d'agir de manière autonome, notamment, qui tournent sur des ordinateurs portables.

\section*{Historique de la Robotique}
La robotique a seulement 60 ans~!
\begin{itemize}
	\item UNIMATE est le premier bras robotique industriel, vendu à partir de 1961.
		Il a une logique pneumatique.
	\item Les premières téléopérations assistées par ordinateur à la CEA (Commission de l'Énergie Atomique) ont commencées dans les années 1970
	\item Du côté de la recherche, le LAAS-CNRS a construit le robot HILARE dans les années 70 sous l'impulsion de Jean Lagasse.
	\item Les robots industriels se sont énormément développés entre les années 80 et 90.
	\item Les premiers robots à jambes sont nés dans les années 80 sous la direction de Mark Hebert (fondateur de Boston Dynamics).
		Les premiers bipèdes de l'Inria sont nés à partir des années 90.
\end{itemize}

\section{Le Contrôle du Mouvement}
L'objectif est de se mouvoir pour agir.
Le mouvement a lieu dans l'espace physique, espace commun entre l'objet et le robot.
On n'a accès qu'à une projection à travers les \textit{sens} du robot, \emph{son espace sensoriel} et on envoie ensuite des commandes à \emph{l'espace moteur}.
Il faut donc prendre en compte une boucle de contrôle:
\begin{category}
	& Physique\arrow[dl, bend right, "Percevoir"] & \\
Sensoriel \arrow[rr, bend right] & & Moteur\arrow[ul, bend right]
\end{category}

On crée un boucle de contrôle:
\begin{itemize}
	\item On définit une action
	\item On représente les éléments nécessaires à l'action et on contrôle les paramètres
	\item On calcule les réactions nécessaires
\end{itemize}
Une partie de l'intelligence du mouvement est dans le design.
Un bon design de robot facilite le mouvement.

Le principe fondamental de la locomotion:
On construit un espace de référence avec les positions des objets alentours.
On calcule le déplacement de notre robot et on calculer ainsi les déplacements des autres objets:
on compose des mouvements, ce qu'on fait donc mathématiquement en utilisant des crochets de Lie.

On va faire des bilans de force qui agissent physiquement sur le centre de masse.
Les robots humanoïdes sont, par le calcul, mus par des forces.
On va donc devoir résoudre, efficacement, les équations de la mécanique Lagrangienne.

Le roboticien a pour but de construire une fonction de contrôle pour passer de l'espace sensoriel à l'espace moteur.
Il y a deux manières principales:
\begin{description}
	\item[Le Contrôle Optimal] On essaie de calculer les commandes moteurs en passant pas une estimation optimale et un contrôle optimal:
		on a deux étapes de modélisation, le calcul de $x$ un état intermédiaire, et le calcul de $u$ l'ensemble des commandes moteurs.
		\begin{align*}
			\min_{x(.), u(.)} \int_{0}^{T}l_{t}\left(x(t), u(t)\right)\d t\\
			\dot{x}(t) = f_{t}\left(x(t), u(t)\right)\\
			x(0) = x_{0}
		\end{align*}
	\item[L'apprentissage par Politique] On veut calculer les commandes moteurs en passant par une simulation du résultat des consignes.
		Ici, $x$ est un état caché de simulation qu'on crée par intégration puis génération, et on regénère des données de mesures.
		On procède donc par apprentissage, et on cherche à résoudre:
		\begin{align*}
			\min_{\color{vulm}\theta}\mathbb{E}_{\xi}\left[\int_{0}^{T}l_{t}\left(x(t), {\color{vulm}\pi_{\theta}\left(x(t), \xi(t)\right)}\right)\d t\right]\\
			\dot{x}(t) = f_{t}\left(x(t), {\color{vulm}\pi_{\theta}(x_{t}, \dot{\xi}(t))}\right)\\
			x(0) \sim \xi_{0}
		\end{align*}
\end{description}


\end{document}
